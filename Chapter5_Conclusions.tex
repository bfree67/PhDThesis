\chapter{Conclusions and Recommendations}
This chapter provides a summary of the research findings, limitations, and recommendations for future work.

\section{Summary}

The research presented in the previous chapters addressed several key areas required to develop and maintain a comprehensive air management program effectively. The major novel academic contributions of this thesis are as follows:

\begin{itemize}
\item {Using synthetic sources and CALPUFF air dispersion modeling to determine common air mixing to identify manageable air quality zones.}
\item {Development of a stochastic model to compare exceedance criteria for zones with multiple air monitoring stations using a risk-based approach}
\item{Development of a stochastic model to quantify small and dispersed area sources for inclusion into national emissions inventories.}
\item{Use of a UAS to capture traffic imagery at signalized intersections and convert into 3D models to evaluate stacking patterns and fleet composition as inputs into mobile emission models}.
\item{Use of deep learning to predict near-term air pollution concentrations using only locally collected historical data to train and test a recurrent neural network with long short term memory.}
\end{itemize}

\subsection{Identifying air quality zones in coastal areas}

Over 1.4 billion people live in coastal communities where complex land-sea breezes impact air quality.  Mapping air quality zones (AQZs) for air quality management in the coastal communities are still done using crude methods based on zones drawn strictly on existing political boundaries. This paper presents a novel statistical risk-based method to identify air quality zones for coastal urban centers that evaluates the effects of local coastal wind patterns and spatial distribution of key urban air pollution sources on air pollutant concentration statistics.  This method uses modeled air dispersion results to calculate the $S-K$ statistics of air pollutant concentrations by placing virtual air pollution receptors within high-risk population centers of the study area.  When graphed against each other, the air quality statistics accurately classify the extent of coastal wind effect within a given zone.
 
This methodology supported the identification of three distinct air quality zones in Kuwait.  These AQZs are tools to classify areas in order to meet local air management strategy.  Inland zones were classified to represent  desert area with little to no residential populations.  Hydrocarbon production zones were included to account for petroleum industry operations taking place in large sections of the study area and managed directly by national oil company assets but supervised by the local regulatory authority.  Coastal zones were established that included the majority of residential and industrial areas (roughly 3.4 million people).  Once the extent of coastal effect mixing was estimated using the $S-K$ method, geophysical boundaries were approximated to facilitate air quality management. 

The model was validated by comparing the concentration and wind data generated using prognostic data with a send weather set. The MM5 and WRF models provided comparable results, especially given the computational differences. Selecting MM5 for the meteorological input to the air dispersion model was validated. The graphical classification approach was tested against a machine learning model. While the SVM provided similar and even better results when additional input features were included, the graphical method is a less complicated procedure.

The selection of individual zones was finally validated by comparing satellite data to exposure risk areas.  The $S-K$ zone classification method and satellite-based risk evaluation method can be employed by other developing nations with limited air monitoring resources and budgets to identify and establish air quality zones of similar air mixing characteristics.

\subsection{Exposure Risk from Zone Classification}

This study presented a new approach for evaluating air quality classification methods by breaking the distribution of historical pollution concentration into compliance and exceedance concentration PDFs. This approach allows the distribution of historical data from an air monitoring station to create ambient air quality profiles to test different compliance scenarios. Two separate cases were used to test the methods – a single station case where the data of only one station was used, and a multiple station case where two or more stations’ data are used. In the multiple station case example, three stations were used. Statistical methods were defined to determine normality of the distributions used for the CDI, and whether the different methods had statistically similar variances and means.

A Null hypothesis was proposed that stated the variance and means of both methods were not significantly different and could, therefore, be used based on non-statistically based preferences. Rejecting the Null showed that the methods did indeed have enough statistically significant differences that to choose one method over another would result in higher local exposure over the duration period. 

Using a Monte Carlos Analysis, it was shown that the two methods do indeed have statistically significant differences in their variances and means for the pollutants tested (NO$_{2}$ and O$_{3}$). While the Null hypothesis is rejected, the individual methods are not. The large sample size and degrees of freedom can reject Null hypothesizes due to extreme sensitivity invariances. Using a different approach based on Effect Size, in which the population size does not affect the statistic, the methods can be compared with better merit. Because the COV tests showed that the distributions were strongly Normal, making the mean and SD robust descriptive statistics that cannot be discounted. The choice of statistic used to compare the classification methods, the CDI, represents a relative risk of exposure, and not a risk of disease, injury or damage. 

Choosing to use one classification method over the other may involve other non-health related factors besides comparing CDI of chemicals. Other factors may include the availability of air monitoring data, economic impacts of declaring a zone non-compliant and the ability of the local air management authority to enforce an improvement program. Other reasons may be that the zone is sparsely populated such as a hydrocarbon production field or an inland desert that may not need the same level of exposure protection. Also, areas experiencing rapid growth would constantly be in non-compliance under a strict 3-strike method based classification whereas a 99\% Rule method would average some of the emission increases over time and look at the overall trend. Recommendations for using both methods are shown in Table \ref{tb18:recommmends}.

% 
\begin{table}[H]
\centering
\caption{Recommended uses for classification methods.}
\label{tb18:recommmends}
\resizebox{\columnwidth}{!}{%
%\begin{adjustbox}{width=1\textwidth}
\begin{tabular}{@{}cc@{}}
\toprule
\textbf{3-Strike Method} & \textbf{99\% Rule Method} \\ \midrule
One station in zone & Multiple stations in zone \\
Zone is primarily residential & Zone is primarily industrial or sparsely populated \\
Incomplete or data sets less than 12 months & Several years of data available \\
Static development & Changes in development and land uses \\ \bottomrule
\end{tabular}
} %end resize
\end{table}

The fact that one method of classification may cause higher exposure than another is not a reason to reject its use- but to use it with discretion. 

\subsection{Estimating annual emissions of distributed area sources}
Recreational smoking using $nargyla$ pipes is a popular pastime for many people in the Middle East and is growing in other parts of the world.  The major emissions are generated from the combustion of charcoal with a fractional contribution from the flavored $sheesha$ tobacco.  This study looked at the gaseous emissions associated with smoking $nargyla$ pipes that could contribute to the overall air pollution emissions inventory.  Emission rates of different gases were collected from previous research and published emission factors.  The emission factors were converted into triangle PDFs to account for variable rates.  Assuming that the many restaurants and cafes are serving $nargyla$ use the similar annual average amounts of charcoal and $sheesha$ tobacco, the Central Limit Theorem could be used to calculate annual feedstock consumption for the entire country.  A Monte Carlo Analysis was used to calculate the total emissions of individual pollutants.  Resulting values showed that GHG emissions were large, while carcinogenic PAH emissions were low overall.  Evaluating the risk associated with exposure to the individual emissions was not part of the study, but is covered in other research \citep{Fromme2009, Moon2015, Mulla2015}.

This investigation demonstrated that selecting the underlying distribution for the emissions factors was important and that Triangle and Pert distributions provide more conservative results and can, therefore, be used when the data set is unknown or limited.  The final $nargyla$ emissions results were significant when compared to the reported total emissions of 1996 reported in Kuwait’s initial correspondence report for the UNFCCC.  Even when the results were adjusted to the reduced population of 1996 (38\% of the current population in 2016), the results are still large compared to the total reported amount. While a conservative estimation was used for the emission factors during the IPCC report preparation, the actual amount of $nargyla$ smoked was not accounted for. Gaps in the inventory include emissions from home use and cafes that might not have been accounted for.

A more realistic assumption is that the 1996 baseline did not capture all the emissions. Using a relatively easy emission source such as $nargyla$ smoking can provide a rapid check of inventory totals to confirm whether estimation processes make sense and are generating expected results. It also highlights where the attention of different pollutants is placed.  Calculations of N$_{2}$O in the 1996 baseline need additional review, especially if $nargyla$ smoking represents 20\% of all reported N$_{2}$O emissions.  The Greenhouse Warming Potential (GWP) of N$_{2}$O is 289 times greater than CO$_{2}$ over 100 years (compared to 24 times for CH$_{4}$)  and is, therefore, an important GHG \citep{IPCC2007}.

While the main purpose of this study was to show how limited data can still be used to generate macro-scale results for purposed of national inventories, the study was nonetheless limited to the accuracies of the emission factors employed.  The published values may not have been accurate representations of the materials used, either overestimating or underestimating the emission rates. Our final figures, therefore, can only represent a range of possible solutions, rather than a definitive answer.  What the results of the study provide is an order of magnitude to compare other results and a basis to prepare more accurate studies.

\subsection{Estimating traffic density}

Traffic density characterization is a fundamental input to mobile source emission inventory estimation and traffic planning. Establishing fleet composition is usually completed by traffic counts and can be done using stationary cameras or human observers. Capturing vehicle spacing is more complicated when using a fixed camera without surface references. Our method, using a UAS and photogrammetry software to create a 3D model of a signaled interchange, allowed easy fleet composition evaluation as well as measuring $\delta_{s}$. By evaluating different intersections during different high volume traffic periods, we were able to establish that both fleet composition and $\delta_{s}$ distributions did not vary significantly and could be pooled to form a larger sample set. The $\delta_{s}$ datasets required logarithmic transformation to achieve normality and meet statistical testing requirements (p$<$0.05). This research is one of the first papers that address the expected stacking gaps between vehicle at SIs.

Our initial results show that traffic density at intersections, and by assumption, congested traffic moving at 5 km/h or less, stacks using a log-normal distribution with a mean of 2.2 m and a variance of 1 m. The $\delta_{s}$ did not depend on vehicle type, although significant differences were noticed for cars in the first lane and SUVs in the third lane. Large buses and HGVs had the shortest $\delta_{s}$, possibly due to driver positions near the vehicle's front bumper. 

\subsection{Forecasting environmental time series using deep learning}
This research is one of the first of its kind to use Deep Learning techniques to predict of air quality time series events. This new methodology produced very good results using our validation data set. A recurrent neural network with LSTM was trained on time series air pollutant and weather data from an air monitoring station in Kuwait to predict 8 hour average O$_{3}$ over different prediction horizons. Missing data and censored data were replaced using a first-order imputation technique that accounted for the sequential influence of previous readings for small gaps ($<$ 8) and seasonal effects for larger gaps. 

A decision tree was used to prioritize the most influential features for training by categorizing pollution exceedances using the input parameters. Prioritizing and removing less important features allows for real-time observations to be fed into the model without transforming large blocks of data as is required when using principal components or wavelets. New observations need only be scaled by normalizing or standardizing with the scaling values calculated from historical data sets. 

A sensitivity analysis of key parameters showed that the network could be tuned for optimal performance. Measurements of the performance, in terms of observed and predicted results, were consistent in form, but the RMSE was always biased higher than the MAE measurement. Either measurement would have produced the same conclusions based on observation of local minima and maxima regardless of the error value. Error increased with the complexity of the network, even with reduced features. This ``Curse of Dimensionality" led to overfitting of the model, reducing the ability to generalize if new data sets were introduced. Slight overfitting is not a problem for time series data that follow predictable cyclic patterns and the main output product of interest is when that pattern goes higher than a set limit.

While the results cannot be directly compared to other studies because different data sets were used, the results should not be dismissed either. Comparing the same data set results to other common forecasting models such as ARIMA and a multi-layer FFNN shows that the RNN does perform significantly better. The complexity of RNN implementation has been dramatically reduced with the use of the Keras developmental library, allowing non-computer scientists the ability to use DL without the coding overhead. The LSTM model provided very good results for this case and can be applied to other environmental time series challenges such as forecasting wide area pollution exceedances from multiple stations and multiple pollutants. LSTMs could also be effective in predicting individual source emissions or modeling source apportionment under different criteria. 

Reducing features and optimizing parameters assisted with lowering error of both training and test sets. Initial runs using local data showed excellent results compared to performance from FFNNs, even with the inclusion of complex pre-processing of input data and architecture of the model. The relative ease of model structure in the programming code is misleading though. The Keras and Theano libraries are some of the most advanced and complex libraries available in the Python community. 

The underlying errors within the model implementation may not be resolved or even quantified. However, they are still useful tools for rapid prototyping and architecture validation. Using the data sets of the sources listed in Table \ref{tb:compare} with our RNN would be a more direct way to prove which method works better.

\section{Limitations and Recommendations}

The following observations are presented within each task that identify key limitations within the presented research and recommendations for follow-on studies.

\noindent
\subsection*{Identifying air zones}

\noindent
\textbf{Observation:}\\
The two areas investigated, Kuwait and Qatar, share common topography, land use features and climate. Other coastal communities with more diverse topography and climates may have different dispersion profiles and be subject to different LSB effects. The methodology presented in this task should be applied to other coastal areas at different latitudes and topography.\\
\noindent\\
\textbf{Observation:}\\
Only SO$_{2}$ was used as a dispersant gas in the virtual sources. Other pollutants such as CO and NO$_{2}$ were not evaluated. Different emission gases should be used in modeling and compared to the dispersion of SO$_{2}$. Substituting different pollutants may be more appropriate for an area that has different source types.\\
\noindent\\
\textbf{Observation:}\\
Modeling outputs from WRF and MM5 prognostic meteorological data were compared, but the prognostic results were never compared to actual observations. Calibration of the models may be necessary to bring the results in line with surface observations available from local weather stations. No study was available that compares prognostic weather data to actuals. This study would be applicable to researchers and consultants using dispersion models in Kuwait.

\subsection*{Exposure risk from zone classification}
\noindent
\textbf{Observation:}\\
A composite subpopulation group was used to evaluate adult body weights instead of assessing individual group separately. These groups can be further broken down by gender, age, and work type. Some jobs require more time outside or long commutes from the place of residence to the place of work. Specific subgroups should be evaluated to determine their susceptibility to exposure risk.\\

\subsection*{Estimating annual emissions from distributed area sources}
\noindent
\textbf{Observation:}\\
A limited number of facilities were considered for evaluation. An exhaustive audit was not conducted, nor were personal sources considered. A more thorough audit of facilities should be conducted as well as surveys of managers to get individual consumption rates of charcoal and tobacco. \\
\noindent\\
\textbf{Observation:}\\
Emission factors from literature were used to estimate emissions. The accuracy of these factors is always questionable and errors can be assumed. Actual measurements of emissions from different charcoal and $sheesha$ tobacco should be conducted to better characterize the factors used.\\

\subsection*{Estimating traffic density}
\noindent
\textbf{Observation:}\\
Flights and data collection were taken only at two different locations at two different times. More data should be collected from different locations, different times of day,  different days of the week, and different weeks of the year to account for possible seasonal effects. \\
\noindent\\
\textbf{Observation:}\\
Measurements were only taken at 3 lane signalized intersections. Other traffic control features were not measured, such as 1 and 2 lane signalized intersections, traffic circles, merge lanes, entrance and exit ramps, and 4-way intersections.  Stacking distances and fleet compositions should be evaluated at these features, especially it some features are common, such as 1 lane signalized intersections.
\\

\subsection*{Forecasting environmental time series using deep learning}
\noindent
\textbf{Observation:}\\
A deep learning system was trained on the data from one AMS only. Most stations are part of a network that can provide additional predictive support due to weather patterns and nearest neighbor influence. Evaluation of results using features from different AMS's should be compared to the results of a single AMS.\\
\noindent\\
\textbf{Observation:}\\
Only 8 hr ave O$_{3}$ was used for training and evaluation. Other measured pollutants were not evaluated. Primary pollutants such as CO or SO$_{2}$  may different patterns as they are not as influenced by photochemical activity as O$_{3}$ is.\\
\noindent\\
\textbf{Observation:}\\
The deep learning network was trained on 1 hr intervals. Some pollutants such as PM$_{10}$ and PM$_{2.5}$ are measured on 24 hr scales. Different input intervals should be trained and tested.

\bigskip
%---------------------------------------------------------------
%------------------------End of Chapter----------------------

\begin{center}
END OF CHAPTER
\end{center}
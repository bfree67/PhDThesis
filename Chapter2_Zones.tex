\noindent \textbf{Transition to Chapter 2}

An apportionment entropy disorder index (AEDI), capturing both temporal and spatial variability of precipitation, was introduced as a new input parameter to an artificial neural networks (ANN) model to more accurately predict flow duration curves (FDCs) at ungauged sites. The ANN model was trained on the randomly selected 2/3 of the dataset of 147 gauged streams in Ontario, and validated on the remaining 1/3. Both location and scale parameters that define the lognormal distribution for the FDCs were highly sensitive to the driving climatic factors, such as, mean annual precipitation, mean annual snowfall, and AEDI. Of the long list of watershed characteristics, the location parameter was most sensitive to drainage area, shape factor and percent area covered by natural vegetation that enhanced evapotranspiration. However, scale parameter was sensitive to drainage area, watershed slope and the baseflow index. Incorporation of the AEDI in the ANN model improved prediction performance of the location and scale parameters by 7\% and 21\%, respectively. A case study application of the new model for the design of micro-hydropower generators in ungauged rural headwater streams was presented. This paper was published in the peer-reviewed Journal of the Air and Waste Management Association:

\begin{itemize}
	\item Freeman, B., Gharabaghi, J.  Th\'e, S. Faisal, M. Abdullah, and A. Al-Aseed (2017). Mapping air quality zones for coastal urban centers. Journal of the Air and Waste Management Association, Volume 67, Issue 5, December 2016, Pages 565-581.
\end{itemize}
\chapter{Identifying air quality zones in coastal areas}

\section{Introduction}
The use of air quality zones (AQZs) as a tool for air management was identified as early as the 1960s \citep{Breivogel1961, Holland1960} and was first mandated State Implementation Plans (SIPs) under the US Clean Air Act (CAA) of 1990 (USEPA, 1990). In the United States, AQZs are based on designated air quality control regions (AQCRs) that include defined municipalities and groups of intrastate and interstate counties (40CFR81, 1991). Currently there are 264 designated AQCRs with 121 in some form of non-compliance (or non-attainment) with the NAAQS (USEPA, 2016). In general, AQZs describe areas where air pollutants are below ambient standards (“in attainment” or “in compliance”), or exceeding them (“in non-attainment” or “in non-compliance”).  The United States Environmental Protection Agency (USEPA) uses ambient air monitoring data from state regulatory agencies measured in each county to determine non-attainment areas \citep{Carr2012}.  USEPA originally initially required states to identify zones of non-attainment for 8 hour ozone.  These regulatory requirements later included sulfur dioxide, Particulate Matter less than 10 microns (PM$_{10}$), and less than 2.5 microns (PM$_{2.5}$).  A consequence of these regulations was the installation of thousands of monitoring stations throughout the USA.  In 2014, 1,293 active ozone monitoring stations were in operation at areas considered to be at risk (USEPA, 2015).  Despite the widespread placement of monitoring equipment, designation of AQZs (or AQCRZs) are limited to areas defined by county lines \citep{Carr2012}.

The European Union initiated a zone management system for its member states in 2008 with Directive 2008/50/EC (EC, 2008).  In addition to designated AQZs, the directive requires at least one monitor for PM$_{2.5}$ in every 100,000 km$^{2}$.  Low emission zones (LEZs) were established in over 200 European cities that enforced restrictions on cars and heavy duty vehicles (Holman et al., 2015).  The drawback of LEZs is that the zones were established based on city boundaries and not by geophysical features \citep{Henschel2013}.

Turkey followed the European Directive to define AQZs for  eight different regional areas on its 81 provinces \citep{CYGM2010}.  The process used multiple methods and tools, including statistical analysis of monitoring station historical data, use of geodatabases and Geographic Information System (GIS) evaluations, air dispersion and local sources, as well as meteorological effects \citep{Karaca2012}.  Furthermore, the Turkish evaluation only employed average daily values of pollutants, instead of hourly averages, with daily values discarded if more than 30\% of hourly data was unavailable.  

China established Control Zones in 1995 that covered over 11.4\% of its territory to manage ever increasing levels of SO$_{2}$ \citep{Hao2000}.  Additional acid rain control zones covered almost 8.4\% of China.  Zones were established based on local meteorological, topographical and soil conditions that had low pH (acidic).  Other designation factors included precipitation pH values and ambient SO$_{2}$ measurements.

\subsection{Coastal Urban Centers}

Approximately 40\% of the population of the United States (over 100 million people) lives in coastal shoreline counties \citep{NOAA2013}.  An estimate of the world population living in coastal areas was approximately 1.4 billion people in 2015, with expected populations growing  to over 1.6 billion by 2024 \citep{Geohive2015}.  With the growth of coastal communities comes air management challenges \citep{Gamas2015}.  Local ambient air quality is impacted by the volume of emissions from industry, transportation, and residential activities, as well as meteorological and seasonal changes \citep{Fiore2015, Kimbrough2013}.  Coastal zones are also subject to land-sea breezes, caused by the diurnal differential heating/cooling of the sea and land \citep{Crosman2010, Cuxart2014, Tsai2011}.  Land-sea breezes (LSBs) continuously shift direction and speed up over the course of the day, sometimes extending as deep as 150 km inland depending on topography, and even deeper for desert regions \citep{Miao2015, Zhu2004}.  Variables that impact the extent and intensity of LSBs are largely geophysical, and include surface heat flux, wind patterns, atmospheric stability and moisture, dimensions of local water bodies, shoreline curvature, topography, and surface roughness \citep{Crosman2010, Lu1995}.  These shifting winds recirculate emissions, which have negative impacts for coastal communities \citep{Lu1996}.  Modeling and evaluating the impact of this recirculation has been the subject of several studies. \citep{Crosman2010, Levy2009, Wu2013, Zhu2004}.  A key objective of this study wais to determine the extent of LSBs along a developed coastline using a quantitative method that incorporated multi-year weather patterns. 

Statistical approaches to evaluating LSB have been applied by many researchers.  Levy et al. (2009) developed a spatio-temporal model to evaluate meso-scale recirculation using a Regional Atmospheric Modeling System HYbrid Particle and Concentration Transport (RAMS-HYPACT) modeling system and inert particles.  They used the recirculation potential index ($R$) to measure recirculation potential at each grid cell.  The $R$ index is a ratio of the wind’s vector and scalar sums over a 24 hour period.  A low numerical $R$ index indicates a constantly changing wind direction, while a high value indicates constant wind direction during the time.  Existing sources were used for model inputs and results were compared to readings from over 26 air monitoring station distributed throughout the area of evaluation.  

Wu et al. (2013) also used an $R$ index to identify LSB impacts as well as a Ventilation Index (VI) based on wind speeds at different altitudes.  Both Levy and Wu required measurements at different altitude using radio soundings or balloons to capture vertical wind speeds.  Neither team’s findings were incorporated into air zone management.  In addition to Levy and Wu, several studies effectively used neural networks and other machine learning-based techniques to analyze complex dispersion patterns in coastal areas \citep{Elangasinghe2014, Feng2015}, but these methods require large data sets for training. 

Classifying categories based on higher order moments, such as Skewness ($S$) and Kurtosis ($K$) statistics, were applied by Martins as early as 1965 to discriminate between beach and dune sand grains \citep{Martins1965}.  Other fields have more recently used $S-K$ statistics for pattern recognition \citep{Crosilla2013}, medical recovery \citep{Chi2008} and music genre classification (Seo and Lee, 2011).  Skewness is a 3rd order moment around the sample mean that measures the symmetry (or asymmetry) of the sample’s distribution. A normal distribution has an $S$ = 0. Kurtosis is a 4th order moment that measures the peakness of a distribution relative to a normal distribution. A normal distribution has a $K$ = 3. A highly skewed distribution with a high kurtosis will have a sharp peak and a long tail on one side of the mean \citep{NIST2016}.  The general equations for $S$ is

\begin{equation}
\label{eq:skewness}
S = \frac{\sum_{i=1}^{N}\left (x_{i}-\bar{x} \right )^{3}}{N\sigma^{3}}
\end{equation}

\noindent
and for $K$ is 

\begin{equation}
\label{eq:kurtosis}
K = \frac{\sum_{i=1}^{N}\left (x_{i}-\bar{x} \right )^{4}}{N\sigma^{4}}
\end{equation}

\noindent
where $x$ is an individual data point, $\bar{x}$ is the data mean, $\sigma$ is the standard deviation, and $N$ is the number of data points \citep{Cristelli2012}.    Some Kurtosis formulas apply a correction term of ``-3" to Eq \ref{eq:kurtosis} (called excess Kurtosis) in order to bias the equation to obtain K = 0 for a normal distribution.  Kurtosis may also be calculated as shown in eq \ref{eq:xskurtosis}.

\begin{equation}
\label{eq:xskurtosis}
K = \frac{N(N+1)}{(N-1)(N-2)(N-3)} \frac{\sum_{i=1}^{N}\left (x_{i}-\bar{x} \right )^{4}}{\sigma^{4}}
\end{equation}

The first term, $(N(N+1))/((N-1)(N-2)(N-3))$, is approximately 1/N for large values of N (N$>$200), thus reducing eq \ref{eq:xskurtosis} down to eq \ref{eq:kurtosis} \citep{Cox2010}.  Difference of distributions with various $S$ and $K$ values are shown below in Figure \ref{fig:SKcurves}.  The blue, solid line curve shows a normal distribution with $S$ = 0 and $K$ = 3. The green dot-dash line shows a Weibull distribution with $S$ = 1.7 and $K$ = 7.4.  The red dash line shows a log-normal distribution with $S$ = 4 and $K$ = 41. 
%
\begin{figure}
\includegraphics[width=\linewidth,height=22.1cm,keepaspectratio]{images/aqz1.png} 
\caption{Distributions representing various $S$ and $K$ statistics.}
\label{fig:SKcurves}
\end{figure}
%
Using the higher order moments amplifies the error of the sample points from the sample mean as well as its variance to allow better classification \citep{Seo2011}.  Crosilla et al. (2013) applied $S$ and $K$ statistics to remote sensor images in order to group terrain into homogeneous categories as based on datasets collected from Light Detection and Ranging (LiDAR).  Alberghi et al. (2002) used Kurtosis to identify vertical velocity components in convective boundary layers associated with LSBs.  They used measured data and data from air dispersion models to identify correlations between Skewness and Kurtosis \citep{Alberghi2002}.  This is very similar to our requirement, in which we seek to classify areas of similar wind fields into among two separate categories – inland and coastal.  While multiple parameters impact the overall mixing and distribution of pollutant concentrations, our hypothesis assumes that long-term pollutant concentrations in coastal regions can be categorized using $S-K$ statistics from inland regions due to the high frequency of multi-directional winds compared to the high frequency of prevailing winds in inland regions.  This assumption is based on previously stated examples in literature and further demonstrated by our modeling approach.  Observed data collected from earth observation satellites validate our assumptions.

Our other hypothesis is that identification of LSB areas can be can be used to establish AQZs.  AQZ determination methods mentioned previously did not take terrain and local weather patterns into consideration to establish zones, even though meteorological impacts have long been recognized as key contributors to air management planning \citep{Leavitt1960}.  Using existing political borders for air zones is convenient when a state or county is small.  A larger jurisdiction may have different terrain, land use, features and microclimates that lead to unequal ambient exposures for receiving populations.  Basing AQZ boundaries strictly on existing political boundaries penalizes stakeholders located outside heavily polluted areas but inside the designated zone \citep{Carr2012}.  One way to incorporate local terrain and weather patterns into the AQZ designation process is to use modern air dispersion models that utilize terrain and weather as input parameters to generate spatial pollutant concentration distributions \citep{Ghannam2013b, Irwin2014}. 

\section{Materials and Methods}
A simple method to estimate LSB extent is a key component to proposing defined air quality zones suitable for compliance enforcement for countries that have a significant industrial and population presence on or near coastlines.  This study presents a novel approach that incorporates LSB into the zone designation process, employing air dispersion modeling results from virtual sources placed in high-risk receptor areas.  A virtual source with the same emission parameters was placed at several designated areas of concern and modeled using a dispersion model over several individual years.  The resulting dispersion patterns over a 20 km x 20 km grid with the virtual source at the center was converted to a histogram and descriptive statistics calculated.  The $S$ and $K$ statistics were used to classify individual areas as inland wind effect impacted or coastal wind effect impacted.

The methodology is summarized below:
\begin{itemize}
\item Scope the overall area, accounting for coast, terrain, and land use.
\item Identify high risk receptor areas.
\item Model dispersion of by a common virtual sources at selected risk areas over 1 year (8,760 hours) for multiple years.
\item Compute $S-K$ statistics for individual years.
\item Determine Coastal/Inland classification based on majority of runs at each site.
\end{itemize}

The following subsections illustrate the new approach as applied by the United Nations Development Program (UNDP) Kuwait Integrated Environmental Management System Project to support the definition of AQZs in Kuwait \citep{Freeman2013}.  

\subsection{Modeling Land-Sea Breeze patterns}

LSBs are the primary dispersal agent of generated emissions for sources located near the sea \citep{Cuxart2014}.  Coastal areas often contain large concentrations of people and industries, thus making them sensitive receptor areas.  In Kuwait, the overwhelming majority of people and industry are located in coastal areas.  The first step of the process is to determine how far LSB patterns encroach inland.  The diurnal nature of LSB is particularly strong for countries in the study area \citep{Zhu2004}.  Effects also vary seasonally with the sea breezes lasting longer in the summer months than the winter months.  Zhu reported maximum penetration along the Gulf coast countries of sea breezes in excess of 250 km.  Of the multiple variables reported by Conklin, Zhu suggests that the main factors affecting LSB are the land-sea heat flux, ambient wind patterns, and local topography.  Figure \ref{fig:LSBwinds} displays examples of coastal wind distributions for southern Kuwait captured at ground level by an air monitoring station.  Similar patterns showing wind directions distributed around the compass with a prevailing component are common for coastal areas \citep{Lozano2013}.

%
\begin{figure}
\includegraphics[width=\linewidth,height=22.1cm,keepaspectratio]{images/aqz2.png} 
\caption{(a) Typical day time and (b) night time ground level LSB winds in southern Kuwait for the year 2011.}
\label{fig:LSBwinds}
\end{figure}
% 
\subsection{Puff Modeling of Virtual Urban Air Pollution Sources}

This new AQZ delineation methodology requires investigation to assess how far inland coastal effect winds travel, by evaluating wind patterns throughout the area.  Virtual emission sources were placed in areas considered to have high receptor and source risks.  These may included dense residential areas, industrial zones, and concentrated urban environments.  The same virtual source is inserted at each site in order to compare the dispersion patterns over the same time frames. Source parameters were selected to represent typical industrial operations and published sources \citep{Chusai2012}. Table \ref{tb:puffmodel} contains parameters used for each virtual source configuration.  The source was designed to represent a typical boiler burning high sulfur diesel fuel.

\begin{table}[]
\centering
\caption{Puff model virtual source properties.}
\label{tb:puffmodel}
\begin{tabular}{@{}cc@{}}
\toprule
\textbf{Parameter}      & \textbf{Value}           \\ \midrule
Modeling period         & 1 Jan 2009 – 31 Dec 2011 \\
Averaging Period        & 8,760 hours              \\
Emission rate           & SO$_{2}$ 200 kg/hr       \\
Stack height            & 20 m                     \\
Stack internal diameter & 1 m                      \\
Gas exit velocity       & 30 m/s                   \\
Gas exit temperature    & 1000 K                   \\ \bottomrule
\end{tabular}
\end{table}

Air dispersion patterns were modeled using CALPUFF 5.8.4 and CALMET 5.8.4  (USEPA, 2014) set to 1 km x 1 km receptor grids in a 20 km x 20 km area at ground level for a total of 400 points per run.  CALMET can accommodate the complex wind fields associated with coastal environments and has been used for modeling land sea breezes for other similar flat terrain \citep{Mangia2010}. Lakes Environmental generated the mesoscale numerical weather prediction (NWP) model (MM5) data to compute wind fields for three individual years (2009 to 2011).  The model calculated the annual average (8,760 hours) for each year at each virtual source location. 

CALPUFF was chosen because it is a non-steady puff model that calculates complex wind fields typically found in coastal zones \citep{Ghannam2013a, Indumati2009, USEPA2014, Weiss2014}.  Digitized local terrain and geophysical data were incorporated into the process during the development of the CALMET model using the Terrain Elevation Data Processing (TERREL), and other preprocessors \citep{Scire2000}.  Digital terrain files were downloaded from the Shuttle Radar Topography Mission (SRTM3) Global Coverage (~90 m) Version 2 database \citep{USGS2000}.  The Global Land Cover Characteristics (GLCC) database maintained by the US Geological Survey (USGS) provided land use data \citep{USGS2008}.  Coast line features were further processed using shoreline data from the Global Self-consistent, Hierarchical, High-resolution Shoreline Database (GSHHS) provided by NOAA's National Centers for Environmental Information \citep{NOAA2015}.  Default values were used for other settings.

Sulfur Dioxide (SO$_{2}$) was use as a pollutant for the virtual sources due to the extensive data already collected by local air monitoring stations and heavy use of sulfur rich fuels in local combustion processes \citep{Al-Awadhi2014, Al-Rashidi2005}.  SO$_{2}$ was also used for other studies involving air control zones \citep{Hao2000, Henschel2013, Pereira2007}.  While photochemistry is an important process in secondary formation of pollutants, it is not important to the evaluation of dispersion and recirculation of pollutants in the AQZ.  Therefore, this study did not consider it as part of the modeling. 

\subsection{Prognostic Weather Set-up}

MM5 is a limited-area, non-hydrostatic, terrain-following sigma-coordinate model designed to simulate or predict mesoscale atmospheric circulation developed by Pennsylvania State University (PSU) and National Center for Atmospheric Research (NCAR) \citep{Grell1994}.  It is used widely to generate local wind fields and weather data for air dispersion models \citep{Ghannam2013a, Lee2009, Tsai2011, Zhu2004}.  Zhu and Atkins evaluated observed and modeled weather conditions at Kuwait International Airport and reported good results for daily and monthly observations using MM5 data.  100 x 100 km data for the study years were prepared in 4 km cells. Vertical layers for the wind fields were calculated at 0, 20, 40, 80, 160, 320, 640, 1200, 2000, 3000 and 4000 m. Concentration results from the air dispersion modeling were analyzed using StatTools and @RISK 7.1.3.1 Industrial Edition (www.palisades.com) to generate histograms and comparative statistics. 

\subsection{Case Study Site Description}

Kuwait is a small country of approximately 17,818 km$^{2}$  located on the northwest corner of the Persian Gulf, between longitudes 46.56$^{o}$ – 48.37$^{o}$ East and latitudes 28.51$^{o}$ - 30.16$^{o}$ North with over 499 km of coastline \citep{CIA2015}.  The country is classified as a desert zone with the highest altitude reaching only 300 meters above sea level.  In 2011, approximately 3.1 million people lived in Kuwait \citep{CSB2016} with over 64\% of its annual gross domestic product (GDP) coming from the production of hydrocarbons\citep{KAMCO2013}.  Other industries in Kuwait includes power generation and water desalination using heavy oil and natural gas at five sites, steel making using electrical induction furnaces, and food preparation.  The country has over 6,600 km of paved roads and over 1.8 million cars in service \citep{OICA2014}.  Kuwait has 6 governorates divided into 81 districts.  Figure \ref{fig:kuwait} shows the map of Kuwait and its location in the Gulf region.  Recent studies demonstrate that air quality has deteriorated along the Kuwaiti coastline, where most of the population is clustered\citep{Al-Awadhi2014, Al-Yakoob2012}.  The resulting concern from local citizens has led to protests and negative press coverage\citep{Carlisle2010}.
 
%
\begin{figure}
\includegraphics[width=\linewidth,height=22.1cm,keepaspectratio]{images/aqz3.png} 
\caption{Location of Kuwait.}
\label{fig:kuwait}
\end{figure}
%
\section{Results and Discussions}

\subsection{Air dispersion model results and histograms}

Virtual sources were placed at key areas to represent different risk areas within Kuwait.  The selected sites include industrial and urban areas as well as oil production and remote desert areas. These sites are distributed among the main industrial and living areas to reflect the main land uses in Kuwait; mainly industrial, residential, oil production and open desert.  The sites are listed in Table \ref{tb:virtualsites} and shown in Figure \ref{fig:virtual-locations}. A partial 10 km coastal buffer is also shown in Figure \ref{fig:virtual-locations} that incorporates two major roads running north/south (40 Highway) and east/west (6th Ring Road).

\begin{table}[!htb]
\centering
\caption{Virtual Site locations.}
\label{tb:virtualsites}
\begin{tabular}{@{}lccl@{}}
\toprule
\textbf{Location}        & \textbf{Longitude} & \textbf{Latitude} & \multicolumn{1}{c}{\textbf{Comments}} \\ \midrule
Ahmadhi                  & 48.0915$^{o}$      & 28.9933$^{o}$     & Dense heavy industrial area           \\
Burgan Field             & 47.9644$^{o}$      & 28.9879$^{o}$     & Oil production operations             \\
Egalia                   & 48.1032$^{o}$      & 29.1393$^{o}$     & Dense residential area                \\
Jahra City               & 47.6671$^{o}$      & 29.3321$^{o}$     & Dense urban area                      \\
Jahra Inland             & 47.7378$^{o}$      & 29.2418$^{o}$     & Sparse population                     \\
Kuwait City              & 47.9874$^{o}$      & 29.3580$^{o}$     & Dense urban area                      \\
Kabd                     & 47.7366$^{o}$      & 29.1967$^{o}$     & Sparse population                     \\
Kuwait Intl Airport      & 47.971$^{o}$       & 29.2440$^{o}$     & Sparse population                     \\
Salmiya                  & 48.069$^{o}$       & 29.3180$^{o}$     & Dense residential area                \\
Shuaibah Industrial Zone & 48.1428$^{o}$      & 28.9944$^{o}$     & Dense heavy industrial area           \\
Subhan                   & 48.0064$^{o}$      & 29.2113$^{o}$     & Dense heavy industrial area           \\ \bottomrule
\end{tabular}
\end{table}
%
%
\begin{figure}
\includegraphics[width=\linewidth,height=22.1cm,keepaspectratio]{images/aqz4.png} 
\caption{Location of virtual sources for air dispersion analysis shown with partial 10 km buffer and main perimeter roads.}
\label{fig:virtual-locations}
\end{figure}
%

Figure \ref{fig:virtual-dispersions} displays examples of individual ground level concentration outputs from the virtual SO$_{2}$ sources described in Table \ref{tb:puffmodel}, while Figure \ref{fig:virtual-histogram} shows histograms of the SO$_{2}$ concentrations calculated by the model to show the distribution of concentrations within the model area.  Concentration color gradients were plotted on the same scale for comparison purposes.  The model is based only on runs accomplished with SO$_{2}$ over an annual average of 8,760 hours.  
%
\begin{figure}
\includegraphics[width=\linewidth,height=22.1cm,keepaspectratio]{images/aqz5.png} 
\caption{Ground level air dispersion patterns from Inland and Coastal virtual SO$_{2}$ sources averaged over 8,760 hours (annual); (a) Jahra Inland dispersion (Inland) - Year 2010; (b) Kuwait City dispersion (Coastal) - Year 2009}
\label{fig:virtual-dispersions}
\end{figure}
%
%
\begin{figure}
\includegraphics[width=\linewidth,height=22.1cm,keepaspectratio]{images/aqz6.png} 
\caption{Histograms of ground level concentrations from Inland and Coastal virtual SO$_{2}$ sources averaged over 8,760 hours (annual); (a) Jahra Inland histogram (Inland) - Year 2010; (b) Kuwait City histogram (Coastal) - Year 2009.}
\label{fig:virtual-histogram}
\end{figure}
%

Different studies using ambient air measurements over time assumed normal distributions of measurements and included statistics such as the mean and standard deviation \citep{Henschel2013, Rivera2015,Suresh2005}.  For highly skewed datasets, traditional statistics cannot effectively describe the distribution.  As described in the introduction section, skewness and kurtosis best describe the different concentration values generated by the puff highly skewed dispersion model runs.  

Zones were identified visually at first in order to establish baseline results.  The calculated statistics for each site and annual run are shown in Table 3 along with distance to shore.  The runs properly classified as inland $(S < 3, K < 20)$ or coastal $(S > 3, K > 20)$ are classified as strong.   Most of the misclassified runs using the $S-K$ method are sites that are clearly situated in an inland or coastal area, but are not properly classified (a site in an inland area classified as coastal and vice versa).  The Shuaibah Industrial Zone - Year 2010, is located 2.1 km from the coast and is considered coastal due to its physical location and that 2 out of 3 trials classified the zone as coastal. Figure \ref{fig:SKplotKuwait} plots the Skewness vs. Kurtosis values from Table \ref{tb:skewdist}. The Subhan site is 9.3 km from the coast and near the border of a coastal zone and a production zone. The Kabd site is clearly inland (16.8 km from the coast), but shows many coastal wind pattern features – suggesting that LSB effect extends deep into this region of Kuwait due to the flat topography and relatively smooth surfaces existing as a result of limited development.
%
\begin{table}[!htb]
\centering
\caption{Skewness, Kurtosis and distance to shore statistics for virtual sites.}
\label{tb:skewdist}
\begin{tabular}{@{}lccccc@{}}
\toprule
\textbf{Site} & \textbf{Zone} & \textbf{Year} & \textbf{Skewness} & \textbf{Kurtosis} & \textbf{Distance (km)} \\ \midrule
Ahmadhi & Coastal & 2009 & 4.18 & 30.98 & 7.1 \\
 &  & 2010 & 4 & 25.43 &  \\
 &  & 2011 & 4.37 & 30.13 &  \\
Burgan Field & Inland & 2009 & 1.84 & 8.32 & 19.2 \\
 &  & 2010 & 2.34 & 11.11 &  \\
 &  & 2011 & 2.74 & 13.47 &  \\
Egalia & Coastal & 2009 & 8.16 & 115.4 & 2.7 \\
 &  & 2010 & 7.33 & 94.55 &  \\
 &  & 2011 & 6.74 & 77.61 &  \\
Jahra City & Coastal & 2009 & 5.07 & 48.94 & 4.9 \\
 &  & 2010 & 4.38 & 37.47 &  \\
 &  & 2011 & 6.46 & 70.95 &  \\
Jahra Inland & Inland & 2009 & 1.74 & 6.52 & 12.7 \\
 &  & 2010 & 2.15 & 9.26 &  \\
 &  & 2011 & 2.29 & 9.31 &  \\
Kabd & Inland & 2009 & 2.89 & 12.96 & 16.8 \\
 &  & 2010 & 3.19 & 16.33 &  \\
 &  & 2011 & 3.55 & 18.85 &  \\
Kuwait City & Coastal & 2009 & 4.71 & 32.72 & 2.2 \\
 &  & 2010 & 4.45 & 30.18 &  \\
 &  & 2011 & 5.22 & 41.33 &  \\
Kuwait Intl Airport & Inland & 2009 & 1.88 & 9.98 & 11.9 \\
 &  & 2010 & 1.56 & 6.56 &  \\
 &  & 2011 & 2.77 & 16.23 &  \\
Salmiya & Coastal & 2009 & 5.19 & 40.12 & 2.2 \\
 &  & 2010 & 5.76 & 48.53 &  \\
 &  & 2011 & 9.17 & 123.69 &  \\
Shuaibah Industrial Zone & Coastal & 2009 & 3.08 & 22.59 & 2.1 \\
 &  & 2010 & 2.33 & 12.56 &  \\
 &  & 2011 & 3.29 & 22.87 &  \\
Subhan & Coastal & 2009 & 2.8 & 12.06 & 9.3 \\
 &  & 2010 & 3.02 & 13.07 &  \\
 &  & 2011 & 3.68 & 26.72 &  \\ \bottomrule
\end{tabular}
\end{table}
%
%
\begin{figure}
\includegraphics[width=\linewidth,height=22.1cm,keepaspectratio]{images/aqz7.png} 
\caption{$S-K$ plot of virtual sources in Kuwait.}
\label{fig:SKplotKuwait}
\end{figure}
%

Regions are highlighted showing inland classification areas (orange) and coastal (blue). A linear trend line of the plotted points, $K$ = 11.901$S$ – 15.732, shows strong correlation ($R^{2}$ = 0.943).

To confirm the procedure, random points were modeled in the nearby State of Qatar. An inland point was taken at Abu Naklah Air Base (ANAB - Lat 25.1208$^{o}$ East/Long 51.3177$^{o}$ North) and a coastal point taken at Masaeed Industrial City (MIC - Lat 24.9911$^{o}$ East/Long 51.5506$^{o}$ North) for the same years (2009-2011) using the same virtual source parameters as shown in Table \ref{fig:virtual-locations}. The points are approximately 570 km from Kuwait International Airport as shown in Figure \ref{fig:qatarlocs}. 

%
\begin{figure}
\includegraphics[width=\linewidth,height=22.1cm,keepaspectratio]{images/aqz8.png} 
\caption{Location of Qatar virtual sites in relation to Kuwait.}
\label{fig:qatarlocs}
\end{figure}
%

The methodology properly classified each location as inland and coastal.  The $S-K$ points are shown in Figure \ref{fig:SKplotQatar} along with a trend line using the parameters from Figure \ref{fig:SKplotKuwait}.

 
Fig 9. $S-K$ plot of virtual sources in Qatar.
%
\begin{figure}
\includegraphics[width=\linewidth,height=22.1cm,keepaspectratio]{images/aqz9.png} 
\caption{$S-K$ plot of virtual sources in Qatar.}
\label{fig:SKplotQatar}
\end{figure}
%

Figure \ref{fig:flowchart} shows a flowchart which summarizes the process to develop this coastal/inland determination.  The process requires at least 3 trial years to allow a majority winner-take-all classification if one year returns a misclassification.

Fig 10. Coastal/Inland Determination Process Flow Chart.
%
\begin{figure}
\includegraphics[width=\linewidth,height=22.1cm,keepaspectratio]{images/aqz10.png} 
\caption{Coastal/Inland determination process flow chart.}
\label{fig:flowchart}
\end{figure}
%

\subsection{Identifying Air Quality Zones for Kuwait}

 The delineation and categorization of the AQZs for Kuwait was completed to assign appropriate control technologies to local air emission processes in order to improve air quality in zones that did not meet air quality standards and prevent deterioration in zones that did \citep{Carr2012}. This approach assumes that areas with predominantly coastal effect winds have different exposure risks than areas with predominantly inland winds, and that they should therefore be managed differently.  In Kuwait, most of the population, industries, power generation, and downstream hydrocarbon processes are located within 10 km of the coast.  The Bay of Kuwait is a relatively shallow body of water that never exceeds a depth of 20 meters.  Seasonal surface water temperatures vary from 12.4 degrees Celsius to 34.8 degrees Celsius \citep{Al-Mutairi2014}.  The presence of this body of water provides a significant source of energy that impacts local weather patterns \citep{Mizak2007,Panin2005}. 

After coastal effect limits were estimated using the $S-K$ method, zones were further defined by incorporating existing roads and expressways.  Where practical, these physical features were taken into account to allow for easier management.  As a result, the Coastal / Inland boundaries follow a main Kuwaiti ring road up to the northeast end of Kuwait Bay.  Municipal district boundaries were also used to promote familiarity when transitional areas were close to estimated boundaries.  The final AQZs are shown in Figure \ref{fig:kuwaitzones}.

%
\begin{figure}
\includegraphics[width=\linewidth,height=22.1cm,keepaspectratio]{images/aqz11.png} 
\caption{Kuwait Air Quality Zones.}
\label{fig:kuwaitzones}
\end{figure}
%

The pink area in Figure \ref{fig:kuwaitzones} shows the buffer area from Figure \ref{fig:virtual-locations} and the dark border show final zone borders. Statistics for the air quality zones are shown in Table \ref{tb:aqzstats}.

\begin{table}[!htb]
\centering
\caption{Air Quality Zone Statistics.}
\label{tb:aqzstats}
\begin{tabular}{@{}clccccc@{}}
\toprule
\textbf{Item} & \textbf{Zone} & \textbf{Zone Type} & \textbf{Area (km$^{2}$)} & \textbf{\% of Total Land} & \textbf{Population} & \textbf{\% of Total Population} \\ \midrule
1 & Northern Coastal & Coastal & 365 & 2.10\% & 392,598 & 11.50\% \\
2 & Southern Coastal & Coastal & 796 & 4.60\% & 737,997 & 21.60\% \\
3 & Central Coastal & Coastal & 426 & 2.50\% & 2,280,773 & 66.70\% \\
4 & Bubiyan & Coastal & 847 & 4.90\% & NA &  \\
5 & Southern Inland & Inland & 2,700 & 15.60\% & NA &  \\
6 & Jahra Inland & Inland & 10,556 & 61.00\% & 9,205 & 0.30\% \\
7 & Wafra & Production & 269 & 1.60\% & NA &  \\
8 & West Kuwait & Production & 375 & 2.20\% & NA &  \\
9 & North Kuwait 1 & Production & 267 & 1.50\% & NA &  \\
10 & North Kuwait 2 & Production & 105 & 0.60\% & NA &  \\
11 & Burgan & Production & 592 & 3.40\% & NA &  \\
 & TOTAL &  & 17,298 & 100\% & 3,420,573 & 100\% \\ \bottomrule
\end{tabular}
\end{table}

The Kuwaiti coastal air zones include 14.1\% of the land area and 99.8\% of the population.  The inland zones include 76.6\% of the land area and the production zones represent 9.3\%.  Descriptions of the individual zone types are provided below:

\textbf{Coastal Zones.} The four coastal zones also include Bubiyan Island and the smaller islands off the coast of Kuwait including Failaka and Kubar Islands.  While comprising only 9.2\% of the land area, coastal zones are home to 99.7\% of all people living and working in Kuwait. Over 88\% of people live in the Southern and Central zones alone.  With additional construction of residential areas and industrial facilities, these are the primary zones for air quality management. 

\textbf{Inland Zones.} The two inland zones are mostly farms, camping areas, military base and free range land comprising over 76\% of the land area in Kuwait. The area is sparsely populated (0.3\% of the population).  

\textbf{Production Zones.} Designated oil and gas fields operated by the national oil companies require different compliance activities due to their industrial processes and site security.  production zones were established to allow the oil industry to apply their own regulatory program under local regulatory guidance and supervision. The zones follow the fence lines of the secure areas and represent over 9.3\% of the land area with no residential population and a sparsely distributed work force. Additionally, larger production zones were found to have internal micro-climates compared to the neighboring inland zones during the modeling phase.  This is due in large part to over-grazing by herd animals outside the secured areas.  The resulting change in land cover clearly impacts albedo and surface heat capacity. Perimeters around production zones are fenced and provide a continuous boundary.

\subsection{Validation of Methodologies}

In addition to applying the method to a secondary area in Qatar, three different methodologies were used to test the validity of the model and the final AQZ selection. The first test was a comparison of prognostic weather data sets to determine if the air dispersion model component is robust. The second test used a support vector machine (SVM) to classify the data sets and compare to the less computationally intensive method we presented. The final test evaluated the regions identified as a result of the LSB classification against satellite based data.

\subsection{Prognostic weather data comparison between MM5 and WRF}

Another common option for prognostic weather data is the Weather research and forecasting (WRF) model also developed by NCAR \citep{Skamarock2008}. WRF has been used in the region for different climatology studies looking at wind forecasts for different applications \citep{Amjad2015}. Direct comparisons of the two models, MM5 and WRF show that the results are statistically similar, with topography playing a large role in predictive accuracy \citep{Gsella2014, Henmi2004}. Comparing ground level wind patterns generated by both the MM5 and WRF models over a common 100 x 100 km area using 4 km grid cells, we found that wind directions were very similar, with wind fields approximately 10\% higher for the MM5 results. When applied to the CALPUFF model, the lower WRF wind speeds provided higher ground level concentrations, however, the final results support our method using MM5. Wind roses for three locations are shown in Figure \ref{fig:12windcompare}.
 	 
%
\begin{figure}
\includegraphics[width=\linewidth,height=22.1cm,keepaspectratio]{images/aqz12.png} 
\caption{Comparison of wind patterns using MM5 and WRF prognostic data.}
\label{fig:12windcompare}
\end{figure}
%

\subsection{Classification by SVM}

SVMs are a class of supervised learning based models that classify input data sets by mapping into high-dimensional feature spaces. While SVMs have been used since 1995 \citep{Cortes1995}, they have only relatively recently been applied to air pollution concentration predictions \citep{Lu2005, Luna2014, Moazami2016}. A full description of the SVM is beyond the scope of this paper but several excellent references and online tutorials are available. A linear SVM was trained with the data set in Table \ref{tb:skewdist} using the Classification Learner module in Matlab 2015a (www.mathworks.com) with a linear kernel function \citep{Yang2011}. The input data was standardized by the software prior to training. Inland classification was converted to a -1 while coastal classification was designated with a 1. The SVM was initially trained only with $S$ and $K$ results. The resulting confusion matrix could achieve an overall accuracy of 87.9\% - comparable to the results of graphing the $S$ and $K$ values.  

%
\begin{figure}
\includegraphics[width=\linewidth,height=22.1cm,keepaspectratio]{images/aqz13.png} 
\caption{Confusion matrix of trained SVM with $S$ and $K$ inputs.}
\label{fig:13confusionSK}
\end{figure}
%

The use of the SVM allows additional features to be included so that multiple inputs can be used to improve accuracy. Adding the coastal distance improved overall accuracy to 97\% as shown in Figure \ref{fig:14confusionSKdist}. The trained SVM was tested using the Qatar data set and achieved 100\% accuracy. 

%
\begin{figure}
\includegraphics[width=\linewidth,height=22.1cm,keepaspectratio]{images/aqz14.png} 
\caption{Confusion matrix of trained SVM with S, K and distance to coast inputs.}
\label{fig:14confusionSKdist}
\end{figure}
%
 
\subsection{Satellite validation of mixing areas}

In order to evaluate the effectiveness of the new zone boundaries, satellite derived data from NASA’s Ozone Monitoring Instrument (OMI) on the Aura satellite was obtained and mapped for NO$_{2}$ over Longitudes 46.125$^{o}$ – 49.125$^{o}$ East and Latitudes 28.375$^{o}$ – 30.375$^{o}$ North \citep{Boersma2011, Strawa2013}.  NO$_{2}$ was used to map actual concentrations due to the higher concentrations measured in Kuwait by the OMI sensor instead of SO$_{2}$.  This provided higher contrasts of readings for visual analysis. Ground monitoring stations have also confirmed that SO$_{2}$ concentrations are not significant while concentrations of NO$_{2}$ are \citep{Al-Awadhi2014}.

Data was downloaded from the Giovanni web based system \citep{Acker2007} and converted to Excel spreadsheets.  The data resolution from the OMI was limited to 0.25 degrees per cell, representing approximately 312 km$^{2}$.  In addition, the data represented the average measurement of a vertical column extending from the surface of the planet to the satellite sensor in a sun synchronous orbit approximately 705 km above sea level.  The satellite repeats its orbit every 16 days (OMI, 2012). 

Annual mean and standard deviations for NO$_{2}$ (measured in 1015 molecule/cm$^{2}$) for the Kuwait area were plotted as shown in Figure \ref{fig:15meanNO2} and Figure \ref{fig:16stdNO2} respectively.  Highest concentrations are located on the Central and Southern coasts corresponding to heavy urban development and industrial activities. The pattern roughly follows the central and southern coastal zones.

%
\begin{figure}
\includegraphics[width=\linewidth,height=22.1cm,keepaspectratio]{images/aqz15.png} 
\caption{Annual mean concentration of NO$_{2}$ in Kuwait for 2011.}
\label{fig:15meanNO2}
\end{figure}
%
%
\begin{figure}
\includegraphics[width=\linewidth,height=22.1cm,keepaspectratio]{images/aqz16.png} 
\caption{Annual Standard Deviation of NO$_{2}$ Concentration as percentage of annual concentration in Kuwait for 2011.}
\label{fig:16stdNO2}
\end{figure}
%

Annual average concentrations of NO$_{2}$ show that the highest concentrations are clustered on the coast. Other zones may contribute to the ambient concentration but exposure risks are predominately located in the Central and Southern Coastal zone areas.  Looking at the standard deviation of the annual average NO$_{2}$ concentrations shows that different areas vary at different rates throughout the year. Areas of high variance occur in the Central Coast and Southern Inland zones while the Southern Coastal zone stays relatively constant.  A zone with high ambient concentrations that does not fluctuate has higher exposure risk. 

\subsection{Air Quality Exposure Risk Scores}

To identify where high exposure risk areas are, a risk scoring method was developed that incorporated annual average concentration ($\mu$) and annual concentration variance ($\sigma^{2}$).  High values of annual concentrations and annual variance were defined as annual concentrations greater than one standard deviation from the mean ($> \mu + \sigma$) or approximately 1.64 times the mean. This value accounts for the highest 5\% of values. Low values were assumed to be less than one standard deviation from the mean ($< \mu - \sigma$) or approximately 0.836 times the mean.

The following assumptions of risk were used:
\begin{itemize}
\item Areas with high annual concentrations and low annual variation had the most exposure risk due to constant high concentration exposure throughout the year.
\item Areas with high annual average concentrations and high annual variations were also considered high-exposure risks.
\item Areas with low annual exposure concentrations and high annual variation had moderate exposure risk due to fluctuating concentration levels throughout the year.
\item Areas with low annual exposure concentrations and low annual variation had low risk.
\end{itemize}

Datasets used for the annual concentration and annual standard deviations were normalized by taking the average of each cell as shown in eq \ref{eq:normconcentrate}

\begin{equation}
\label{eq:normconcentrate}
CN_{i,j}=\frac{C_{i,j}}{\bar{C}}
\end{equation}

\noindent
where $CN_{i,j}$ is the normalized annual concentration value of the cell, $C_{i,j}$ is the annual concentration value for the cell, and $\bar{C}$ is the average concentration of all cells.  A similar procedure was done to normalize the standard deviation of each cell.  Risk values ranged from 0 (low) or 1 (high) for concentration and 0 (low) to 0.5 (high) for standard deviation as shown in eq 5. 

\begin{equation}
\label{eq:riskscore}
Risk = CN_{risk} + StDev_{risk}
\end{equation}

\noindent
where $CN_{risk}$ is given as

\begin{equation}
\label{eq:riskCN1}
CN_{risk} = \left\{\begin{matrix}
1, CN > 1.6\\ 
0, CN \leq 1.6
\end{matrix}\right.
\end{equation}

\noindent
and $StDev_{risk}$ is given as
\begin{equation}
\label{eq:riskCN2}
StDev_{risk} = \left\{\begin{matrix}
0.5, StDev > 16\%\\ 
0, StDev \leq 16\%
\end{matrix}\right.
\end{equation}

Risk scores are shown in Table \ref{tb:expriskscores} and Figure 17.  High and medium risk areas are predominantly in the coastal areas with some medium risk areas extending inland where heavy oil production takes place.

\begin{table}[!htb]
\centering
\caption{Exposure Risk Scores.}
\label{tb:expriskscores}
\begin{tabular}{@{}cc@{}}
\toprule
\textbf{Score} & \textbf{Risk} \\ \midrule
1.5            & High          \\
1              & Medium - High \\
0.5            & Medium        \\
0              & Low           \\ \bottomrule
\end{tabular}
\end{table}

%
\begin{figure}
\includegraphics[width=\linewidth,height=22.1cm,keepaspectratio]{images/aqz17.png} 
\caption{NO$_{2}$ Exposure risk based on annual average concentration and standard deviation. Areas of high population are shown as blue dots.}
\label{fig:176riskNO2}
\end{figure}
%

The use of satellite data confirms the high exposure risk areas fall within identified AQZs.  These areas should have priority in regards to access of pollution control technology and exposure reduction programs.  Zones that do not have high exposure risks can be managed differently such as providing additional control technology for processes in zones near high exposure risk areas to prevent cross-zone transport of emissions. 

\section{Conclusions and Recommendations}
Over 1.4 billion people live in coastal communities where air quality is impacted by complex land-sea breezes.  Mapping air quality zones (AQZs) for air quality management in the coastal communities is still done using crude methods based on zones drawn strictly on existing political boundaries. This paper presents a novel statistical risk-based method to identify air quality zones for coastal urban centers that evaluates the effects of local coastal wind patterns and spatial distribution of key urban air pollution sources on air pollutant concentration statistics.  This method uses modeled air dispersion results to calculate the $S-K$ statistics of air pollutant concentrations by placing virtual air pollution receptors within high risk population centers of the study area.  When graphed against each other, the air quality statistics accurately classify the extent of coastal wind effect within a given zone.
 
This methodology supported the identification of three distinct air quality zones in Kuwait.  These AQZs are tools to classify areas in order to meet local air management strategy.  Inland zones were classified to represent primarily desert area with little to no residential populations.  Hydrocarbon production zones were included to account for petroleum industry operations taking place in large sections of the study area and managed directly by national oil company assets but supervised by the local regulatory authority.  Coastal zones were established that included the majority of residential and industrial areas (roughly 3.4 million people).  Once the extent of coastal effect mixing was estimated using the $S-K$ method, geophysical boundaries were approximated to facilitate air quality management. 

The model was validated by comparing the concentration and wind data generated using prognostic data with a send weather set. The MM5 and WRF models provided comparable results, especially given the computational differences. Selecting MM5 for the meteorological input to the air dispersion model was validated. The graphical classification approach was tested against a machine learning model. While the SVM provided similar and even better results when additional input features were included, the graphical method is a less complicated procedure.

The selection of individual zones was finally validated by comparing satellite data to exposure risk areas.  The $S-K$ zone classification method and satellite based risk evaluation method can be employed by other developing nations with limited air monitoring resources and budgets to identify and establish air quality zones of similar air mixing characteristics.

Additional research is recommended to confirm the application of this method to other areas.  Kuwait has relatively homogeneous topography, a very hot climate and is located by a shallow body of water.  This method may directly lend itself to other countries in the Gulf region, as shown with sites in Qatar, but should be studied further to evaluate its ability to predict coastal effect mixing for regions with more rugged terrain and more temperate weather.






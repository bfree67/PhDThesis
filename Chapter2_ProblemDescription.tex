\chapter{Problem Description}
The purpose of this research is to advance the state of understanding for key areas in air management programs. Gaps and incomplete knowledge areas identified in the previous chapter show that specific areas should be explored. In addition to identifying knowledge gaps within literature and open access sources, practical understanding of real world needs were identified during the development of an air management program for Kuwait under the UNDP Kuwait Integrated Environmental Management Systems (KIEMS) project \citep{KIEMS2012}.  While the motivation to develop these tools were to assist regulatory agencies in developing countries, the procedures developed as a result of this research can be used to augment and reinforce mature air management programs as well.  

The main challenges identified during the KIEMS project can be decomposed into five main tasks that are tangentially related and build off each other. The first task, identification of air quality zone through discrimination of coastal and inland wind patterns led to assignment of zones to improve local management of generators. The second task of establishing classification criteria for the zones allows for long term management of the zones. The third task supports the completion of an emissions inventory by estimating traffic densities as inputs to vehicle emission models. The fourth task provides a method to estimate small common emission sources distributed throughout an area that may be overlooked during emission inventories. The last task, predicting air quality exceedances, allows for short term forecasts of air quality conditions within the zones. 

The relationship of each task to the overall framework of air quality management is shown in Figure \ref{fig:tasks}.
%
\begin{figure}[!htbp]
\centering
\includegraphics[width=.75\textwidth]{images/tasks}  %assumes jpg extension
\caption{Relationship of tasks in the air quality management framework.}
\label{fig:tasks}
\end{figure}
%
The individual tasks are shown within the heavy borders.

In the rest of this chapter, research problem statements are defined for each task based on their impact to the overall research and need within the research community. Each section includes scoping of the problem and major assumptions.

\section{Research Needs}

Air quality management for an area begins with one point. That point is usually a sampling tube mounted at around 10m above the ground and (hopefully) far from busy roads, exhaust vents, and smokestacks. The sample of the ambient air is therefore assumed to be well mixed and represent the prevailing blend of different upwind emissions generators, and not weighted by a nearby source. The sample is split and drawn into sensors that measure the concentration of a specific parameter (or family of parameters) over a measurement averaging period (usually every five minutes). The measurement period is aggregated into larger time weighted averages, such as 1 hr, 3 hrs, 8 hrs, or 24 hrs, for each parameter, depending on local air quality standards. 

If the average measured concentration is below the standard, the air quality is acceptable. If it is above the standard, it is an exceedance. If we accept the reasonable assumption that air quality standards are based on solid toxicological studies and represent an acceptable balance between allowable pollution releases and human health, ambient air quality concentrations become the measure of performance for air quality.

Just as one sample cannot adequately represent a complex population in statistics, one air monitoring station cannot adequately represent a country with complex topographies, urban centers, industrial activities, transportation networks, weather patterns, and proximity to neighbors with their own complex environments.  In short, the air quality in one part of the country will most likely be different from the other parts. For regulatory agencies with limited staff and budget, prioritizing areas with poor air quality is a more efficient use of resources than applying the same enforcement standards uniformly. Identifying areas for enforcement jurisdiction is therefore an important basis of an air management program. As shown in the previous chapter, initial zones were usually established using existing political boundaries without consideration of micro-climate effects based on geography and land use.  

\section{Establishing the program}
\subsection{Air zone mapping}

In this research, we develop a novel procedure to identify common dispersion areas using the assumption that land-sea breezes (LSBs) provide different circulatory wind patterns than inland wind. As part of the methodology, we only consider dispersion patterns at ground level, and not at different altitudes.

\subsection{Attainment classification}
Once zones are established, there needs to be a method in place to convert the measured concentration data into management action. Any monitoring station will register high readings that exceed the established standard for the parameter. Assuming the sensor is calibrated and operating properly, the question becomes how many exceedances are allowable before direct action is taken to reduce generator emissions - thereby impacting economic production by cutting operations, expending capital to improve processes, or changing transportation habits by limiting private vehicle operation. These exceedances may be only 3 exceedances per year, as is the case for O$_{3}$ in the USA, or 2.5 million premature deaths, as is the case in India \citep{Landrigan2017}.

In this research, we developed a novel method to evaluate different ways to determine if a zone met its air quality expectations, or presented a health risk based on high exposure. Methods were developed to evaluate an area with only one monitoring station and areas with multiple stations. Only two classifications were evaluated, although any classification approach can use it. The USEPA's Human Health Risk Assessment (HHRA) approach was used, along with assumptions on body weight, inhalation rates, and exposure duration. 

\subsection{Emission inventories}
Once management zones are identified and a method of evaluating the overall condition of the zone is established, the next step to managing air quality is to understand where the pollutants come from in order to impact the processes that generate them. This requires an emissions inventory of sources that includes rates of emission for individual analytes, physical parameters of the release point such as height, release temperature, release velocity, and area of release, as well as the geospatial location of the emission source. It also helps to include an industrial code of the source in order to evaluate processes within a sector, and an emission source code to group common emission processes \citep{The2008}.

Collection of emission rate data is often challenging as the rates are time dependent based seasonal, diurnal, weekly and production schedules. Emissions are also impacted by the quality of feedstock (\% of sulfur or ash in the material). Accurate collection of real-time emissions using continuous emission monitoring systems (CEMS) are limited only to the analytes being measured and to large sources that can afford to install and operate these systems. Other techniques use some form of parametric emission factor equation - whether local factors that include operational data from a process historian capturing fuel rates, operating temperatures, and combustion efficiencies, or published emission factors such as USEPA's AP-42 that use an input feedstock \cite{USEPA1995}. While some of the AP-42 factors are rated as high quality (B or better), they are nonetheless outdated (published in 1995) and incomplete (despite over 12,000 published factors).

\subsection{Distributed area sources}
While most EFs were developed to characterize stationary sources, many EFs also provide values for area sources. The area sources are assumed to be local and associated with a specific organization. In many non-industrial or commercial scenarios, small sources exist that aggregate to a significant annual amount when taken together.  These sources may be spread throughout the country (or zones), such as smoking or grilling.

In this research, we look at the impact of emission contributions by $nargyla$ water pipes in restaurants and caf\'es. We assume that there are a set number of establishments and only consider non-personal uses.  Monte Carlo Analysis is used to represent ranges of uncertainty and variations within data sets. Emission factors were selected from published literature.

\subsection{Traffic estimating}
In order to inventory mobile source emission contributions, we need to quantify the amount and type of vehicles on the road and their driving patterns. In this research, we present a two step approach that uses the near-real time traffic status updates on the Google Maps Traffic Overlay (GMTO) combined with a Monte Carlo analysis of vehicle types expected to be on the road. We assume that the GMTO is a fair representation of actual traffic conditions, and that the vehicle distributions are also accurate. For the inventory of vehicles using the GMTO, we only looked a t major highways, freeways, arterial, feeders and roundabouts. Smaller surface streets were not evaluated but can easily play a large part in local air quality.

\section{Forecasting conditions}

Predicting air pollution concentrations is important to reduce exposure from at risk populations and also to prevent over exceedances by limiting emission generating activities. In this research machine learning techniques were used to prepare raw air monitoring data by replacing missing and outlier data points and identifying features for training. A deep learning model was used for time series prediction at one monitoring station for 8 hour averaged O$_{3}$ only.  Data from 3 years was used with 70\% reserved for training, 15\% used for validating parameters within the deep learning model, and the remainder used to test the trained model.

\section{A working air quality program}

The goal of this research was to provide tools for air quality managers to strengthen the effectiveness of their program given limited budget and staff. Aside from the air monitoring stations, their maintenance and operation, and computing resource that any modern office should have, the outputs of these tools require very little capital cost. One necessary expense is prognostic weather data (WRF or MM5) required to identify air zone micro climates. This may cost around \$1,895 US Dollars for five years of data for a 100 km x 100 km set with 4km grids. The MM5 data used in this research was originally purchased by the Integrated Environmental Solutions Company, who graciously allowed its use. Another expense is a Geographic Information System (GIS) license such as ESRI's ArcGIS Desktop Advanced which can cost over \$4,200 USD (depending on which country it's purchased in).

Other software costs can be considered if full licenses are included. Palisade's @Risk software used for the Monte Carlo analysis work costs around \$800 US Dollars per license and Lakes Environmental CALPUFF View costs \$2,575 US Dollars. If a department did not want to make these purchases, they could either use the free 30 day trials, or use a no-cost option. CalPuff is a free download but requires a significant investment in training to use. Random number generators are available in Microsoft Excel or Python's Numpy library. A breakdown of major software and products used in this research is shown in Table \ref{tb:projcosts}.

\begin{table}[]
\centering
\caption{Estimated software costs for research tasks.}
\label{tb:projcosts}
\begin{tabular}{@{}lcc@{}}
\toprule
\textbf{Research task} & \textbf{Software Used} & \textbf{License Fee} \\ \midrule
Identifying Air Zones & CALPUFF View & \$2,950.00 \\
 & Prognostic weather data & \$1,895.00 \\ \midrule
Classifying attainment & @Risk & \$2,575.00 \\ \midrule
Distributed area sources & @Risk & (same as above) \\ \midrule
Estimating traffic density & ArcGIS Desktop Advanced & \$4,200.00 \\
 & Python & - \\ \midrule
Forecast concentration & Python & - \\
 & \multicolumn{1}{r}{\textbf{Total}} & \textbf{\$11,620.00} \\ \bottomrule
\end{tabular}
\end{table}
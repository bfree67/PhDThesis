\chapter{Problem Description}
The purpose of this research is to advance the state of understanding for key areas in air management programs. Gaps and incomplete knowledge areas identified in the previous chapter show that specific areas should be explored. In addition to identifying knowledge gaps within literature and open access sources, practical understanding of practical needs were identified during the development of an air management program for Kuwait under the UNDP Kuwait Integrated Environmental Management Systems (KIEMS) project but are relevant to other regulatory agencies in the region. 

The main challenges identified during the KIEMS project can be decomposed into five main tasks that are tangentially related and build off each other. The first task, identification of air quality zone through discrimination of coastal and inland wind patterns led to assignment of zones to improve local management of generators. The second task of establishing classification criteria for the zones allows for long term management of the zones. The third task supports the completion of an emissions inventory by estimating traffic densities as inputs to vehicle emission models. The fourth task provides a method to estimate small common emission sources distributed throughout an area that may be overlooked during emission inventories. The last task, predicting air quality exceedances, allows for short term forecasts of air quality conditions within the zones. 

The relationship of each task to the overall framework of air quality management is shown in Figure \ref{fig:tasks}.
%
\begin{figure}[!htbp]
\centering
\includegraphics[width=.75\textwidth]{images/tasks}  %assumes jpg extension
\caption{Relationship of tasks in the air quality management framework.}
\label{fig:tasks}
\end{figure}
%
The individual tasks are shown within the heavy borders.

In the rest of this chapter, research problem statements are defined for each task based on their impact to the overall research and need within the research community. Each section includes scoping of the problem and major assumptions.

\section{Air zone mapping}
Air quality for an entire region begins with one point. That point is usually a sampling tube mounted at around 10m above the ground and (hopefully) far from busy roads, exhaust vents, and smokestacks.

The first step to managing air quality is to understand where these pollutants come from in order to prioritize actions to impact the processes that generate them. This requires an emissions inventory of sources that includes rates of emission for individual analytes, physical parameters of the release point such as height, release temperature, release velocity, and area of release, as well as the geospatial location of the emission source, the industrial code of the source, and the source classification code (SCC) of the process \cite{The2008}.

Collection of emission rate data is often challenging as the rates are time dependent based seasonal, diurnal, weekly and production schedules. Emissions are also impacted by the quality of feedstock (\% of sulfur or ash in the material). Accurate collection of real-time emissions using continuous emission monitoring systems (CEMS) are limited only to the analytes being measured and to large sources that can afford to install and operate these systems. Other techniques use some form of parametric emission factor equation - whether local factors that include operational data from a process historian capturing fuel rates, operating temperatures, and combustion efficiencies, or published emission factors such as USEPA's AP-42 that use an input feedstock \cite{USEPA1995}. While some of the AP-42 factors are rated as high quality (B or better), they are nonetheless outdated (published in 1995) and incomplete (despite over 12,000 published factors).

\section{Attainment classification}


\section{Distributed area sources}


\section{Traffic estimating}


\section{Concentration forecasting}

\chapter{Problem Description}
The purpose of this research is to advance the state of understanding for key areas in air management programs. Gaps and incomplete knowledge areas identified in the previous chapter show that specific areas should be explored. In addition to identifying knowledge gaps within literature and open access sources, practical understanding of real world needs were identified during the development of an air management program for Kuwait under the UNDP Kuwait Integrated Environmental Management Systems (KIEMS) project \citep{UNDP2012}.  While the motivation to develop these tools were to assist regulatory agencies in developing countries, the procedures developed as a result of this research can be used to augment and reinforce mature air management programs as well.  

The main challenges identified during the KIEMS project can be decomposed into five main tasks that are tangentially related and build off each other. The first task, identification of air quality zone through discrimination of coastal and inland wind patterns led to assignment of zones to improve local management of generators. The second task of establishing classification criteria for the zones allows for long term management of the zones. The third task supports the completion of an emissions inventory by estimating traffic densities as inputs to vehicle emission models. The fourth task provides a method to estimate small common emission sources distributed throughout an area that may be overlooked during emission inventories. The last task, predicting air quality exceedances, allows for short term forecasts of air quality conditions within the zones. 

The relationship of each task to the overall framework of air quality management is shown in Figure \ref{fig:tasks}.
%
\begin{figure}[H]
\centering
\includegraphics[width=.75\textwidth]{images/tasks}  %assumes jpg extension
\caption{Relationship of tasks in the air quality management framework.}
\label{fig:tasks}
\end{figure}
%
The individual tasks are shown within the heavy borders.

In the rest of this chapter, research problem statements are defined for each task based on their impact to the overall research and need within the research community. Each section includes scoping of the problem and major assumptions.

\section{Research Needs}

Air quality management for an area begins with one point. That point is usually a sampling tube mounted at around 10m above the ground and (hopefully) far from busy roads, exhaust vents, and smokestacks. The sample of the ambient air is therefore assumed to be well mixed and represent the prevailing blend of different upwind emissions generators, and not weighted by a nearby source. The sample is split and drawn into sensors that measure the concentration of a specific parameter (or family of parameters) over a measurement averaging period (usually every five minutes). The measurement period is aggregated into larger time weighted averages, such as 1 hr, 3 hrs, 8 hrs, or 24 hrs, for each parameter, depending on local air quality standards. 

If the average measured concentration is below the standard, the air quality is acceptable. If it is above the standard, it is an exceedance. If we accept the reasonable assumption that air quality standards are based on solid toxicological studies and represent an acceptable balance between allowable pollution releases and human health, ambient air quality concentrations become the measure of performance for air quality.

Just as one sample cannot adequately represent a complex population in statistics, one air monitoring station cannot adequately represent a country with complex topographies, urban centers, industrial activities, transportation networks, weather patterns, and proximity to neighbors with their own complex environments.  In short, the air quality in one part of the country will most likely be different from the other parts. For regulatory agencies with limited staff and budget, prioritizing areas with poor air quality is a more efficient use of resources than applying the same enforcement standards uniformly. Identifying areas for enforcement jurisdiction is therefore an important basis of an air management program. As shown in the previous chapter, initial zones were usually established using existing political boundaries without consideration of micro-climate effects based on geography and land use.  

\subsection{Area of Study}
Most of the data used in this research was collected in the State of Kuwait over the period of 2012 - 2017. Kuwait is a small country of 17,818 km$^{2}$ located on the northwest corner of the Persian Gulf, between longitudes 46.56$^{o}$ – 48.37$^{o}$ East and latitudes 28.51$^{o}$ - 30.16$^{o}$ North with over 499 km of coastline \citep{CIA2015}. It is bordered by Iraq to the north and Saudi Arabia to the south and west. The country is classified as a desert zone with the highest altitude reaching only 300 meters above sea level.   In 2011, approximately 3.1 million people lived in Kuwait \citep{CSB2017} with over 64\% of its annual Gross Domestic Product (GDP) coming from the production of hydrocarbons \citep{KAMCO2013}.  Other industries in Kuwait include power generation and water desalination using heavy oil and natural gas at five sites, steel making using electrical induction furnaces, and food preparation.  The country has over 7,400 km of paved roads and over 1.8 million cars in service \citep{CSB2014}.  Over 98\% of the population lives within 10 km of the coast and are subject to coastal effect winds, caused by the diurnal differential heating/cooling of the sea and land \citep{Crosman2010, Cuxart2014}.  The land-sea breezes (LSB) shift direction and speed over the course of the day, re-circulating pollution back and forth from land sources.

The Kuwait Environment Public Authority (KEPA) is responsible for monitoring environmental conditions and enforcing compliance with national environmental law. It operates 15 each fixed site AMS’s located along the coast as shown in Fig \ref{fig1:amskuwait}.  The KEPA stations measure and report 1 hour averaged air pollutants and meteorological conditions. The bulk of the AMSs are located within two air quality zones (Central and Southern Coast) which correspond to areas of high population and industrial centers. Access to monitoring data from additional stations operated by the Kuwait Oil Company (KOC) provides an air monitoring density of approximately 1 monitoring station per 163,150 people. This density is similar to Vancouver, Canada which has approximately 1 per 160,000 people, as compared to around 1 per 440,000 people in the South Coast of California \citep{Marshall2008}.

%  
\begin{figure}[H]
\centering
\includegraphics[width=.75\textwidth]{images/risk1.png} 
\caption{Location of air monitoring stations in Kuwait.}
\label{fig1:amskuwait}
\end{figure}

\section{Establishing an air management program}
\subsection{Air zone mapping}

In this research, we develop a novel procedure to identify common dispersion areas using the assumption that land-sea breezes (LSBs) provide different circulatory wind patterns than inland wind. As part of the methodology, we only consider dispersion patterns at ground level, and not at different altitudes.

\subsection{Attainment classification}
Once zones are established, there needs to be a method in place to convert the measured concentration data into management action. Any monitoring station will register high readings that exceed the established standard for the parameter. Assuming the sensor is calibrated and operating properly, the question becomes how many exceedances are allowable before direct action is taken to reduce generator emissions - thereby impacting economic production by cutting operations, expending capital to improve processes, or changing transportation habits by limiting private vehicle operation. These exceedances may be only 3 exceedances per year, as is the case for O$_{3}$ in the USA, or 2.5 million premature deaths, as is the case in India \citep{Landrigan2017}.

In this research, we developed a novel method to evaluate different ways to determine if a zone met its air quality expectations, or presented a health risk based on high exposure. Methods were developed to evaluate an area with only one monitoring station and areas with multiple stations. Only two classifications were evaluated, although any classification approach can use it. The USEPA's Human Health Risk Assessment (HHRA) approach was used, along with assumptions on body weight, inhalation rates, and exposure duration. 

\subsection{Emission inventories}
Once management zones are identified and a method of evaluating the overall condition of the zone is established, the next step to managing air quality is to understand where the pollutants come from in order to impact the processes that generate them. This requires an emissions inventory of sources that includes rates of emission for individual analytes, physical parameters of the release point such as height, release temperature, release velocity, and area of release, as well as the geospatial location of the emission source. It also helps to include an industrial code of the source in order to evaluate processes within a sector, and an emission source code to group common emission processes \citep{The2008}.

Collection of emission rate data is often challenging as the rates are time dependent based seasonal, diurnal, weekly and production schedules. Emissions are also impacted by the quality of feedstock (\% of sulfur or ash in the material). Accurate collection of real-time emissions using continuous emission monitoring systems (CEMS) are limited only to the analytes being measured and to large sources that can afford to install and operate these systems. Other techniques use some form of parametric emission factor equation - whether local factors that include operational data from a process historian capturing fuel rates, operating temperatures, and combustion efficiencies, or published emission factors such as USEPA's AP-42 that use an input feedstock \citep{USEPA1995}. While some of the AP-42 factors are rated as high quality (B or better), they are nonetheless outdated (published in 1995) and incomplete (despite over 12,000 published factors).

\subsection{Distributed area sources}
While most EFs were developed to characterize stationary sources, many EFs also provide values for area sources. The area sources are assumed to be local and associated with a specific organization. In many non-industrial or commercial scenarios, small sources exist that aggregate to a significant annual amount when taken together.  These sources may be spread throughout the country (or zones), such as smoking or grilling.

In this research, we look at the impact of emission contributions by $nargyla$ water pipes in restaurants and caf\'es. We assume that there are a set number of establishments and only consider non-personal uses.  Monte Carlo Analysis is used to represent ranges of uncertainty and variations within data sets. Emission factors were selected from published literature.

\subsection{Vehicle density estimation in congested environments}

A key element of estimating total mobile source emission contributions to the emissions inventory is identifying the types and numbers of vehicles using road segments at particular times of the day in order to prepare a reasonable model of activity that can then be used to calculate emissions. Evaluating traffic at an intersection allows for discrete analysis of vehicles in terms of fleet composition, as well as providing another input to mobile emissions in regards to emissions generated during idling and accelerating from rest.

Using a commercial drone and DEM generating software, 3D models will be generated of traffic at signalled intersections. The IVG and vehicle type will be extracted from the model in order to build a stochastic model for fleet composition and vehicle density. 

\section{Forecasting air quality conditions}

Predicting air pollution concentrations is important to reduce exposure from at risk populations and also to prevent over exceedances by limiting emission generating activities. In this research machine learning techniques were used to prepare raw air monitoring data by replacing missing and outlier data points and identifying features for training. A deep learning model was used for time series prediction at one monitoring station for 8 hour averaged O$_{3}$ only.  Data from 3 years was used with 70\% reserved for training, 15\% used for validating parameters within the deep learning model, and the remainder used to test the trained model.


%\appendix

\chapter{Published Papers}

The following papers were prepared and published (or submitted for review) during the course of this research

\section{First paper}
\noindent
\textbf{Estimation of Mixed Traffic Densities in Congested Roads using Monte Carlo Analysis}. Freeman, B., Gharabaghi, J.  Th\'e, (2015). \textit{EM}, April 2015, pages 8-13. 

\vspace{5mm}
\noindent
\textbf{Context}\\
\noindent
The authors of this article developed a novel stochastic Monte Carlo approach to estimate the number and type of vehicles on congested road sections. This method assumes that each vehicle occupies road space based on its length and inter-vehicle gap during congested traffic. The inter-vehicle spacing is subject to variability due to driver behavior and average speed. By assigning vehicle spaces on a road based on the average speed, each vehicle can be treated as an independent variable using Monte Carlo simulation to identify ranges of possible outcomes. Under multiple sampling, the most likely number of mixed vehicles in a 1-km unit road length can be represented by mean of the distribution of results due to the Central Limit Theorem. Spacing at different speeds are calculated and vehicle density curves estimated for combined traffic and individual vehicle types.

\vspace{5mm}
\noindent
\textbf{Contributions}\\
\noindent
This paper presents a novel model to estimate vehicle densities using Monte Carlo Analysis. National fleet characteristics are used to create the initial distribution of vehicles on the road with the number of vehicles in a kilometer segment based on independent variables that are a function of the average traffic speed. 

\noindent
\section{Second paper}
\noindent
\textbf{Mapping air quality zones for coastal urban centers}. Freeman, B., Gharabaghi, J.  Th\'e, S. Faisal, M. Abdullah, and A. Al-Aseed (2017). \textit{Journal of the Air and Waste Management Association}, Volume 67, Issue 5, December 2016, Pages 565-581, DOI: 10.1080/10962247.2016.1265025.

\vspace{5mm}
\noindent
\textbf{Context}\\
\noindent
This paper presents a novel method that incorporates modern air dispersion models allowing local terrain and land–sea breeze effects to be considered along with political and natural boundaries for more accurate mapping of air quality zones (AQZs) for coastal urban centers. This method uses local coastal wind patterns and key urban air pollution sources in each zone to more accurately calculate air pollutant concentration statistics. The new approach distributes virtual air pollution sources within each small grid cell of an area of interest and analyzes a puff dispersion model for a full year’s worth of 1-hr prognostic weather data. The difference of wind patterns in coastal and inland areas creates significantly different skewness (S) and kurtosis (K) statistics for the annually averaged pollutant concentrations at ground level receptor points for each grid cell. Plotting the S-K data highlights grouping of sources predominantly impacted by coastal winds versus inland winds. The application of the new method is demonstrated through a case study for Kuwait by developing new AQZs to support local air management programs. The zone boundaries established by the S-K method were validated by comparing MM5 and WRF prognostic meteorological weather data used in the air dispersion modeling, a support vector machine classifier was trained to compare results with the graphical classification method, and final zones were compared with data collected from Earth observation satellites to confirm locations of high-exposure-risk areas. The resulting AQZs are more accurate and support efficient management strategies for air quality compliance targets effected by local coastal microclimates.

\vspace{5mm}
\noindent
\textbf{Contributions}\\
\noindent
This paper presents a novel method to determine air quality zones in coastal urban areas is introduced using skewness (S) and kurtosis (K) statistics calculated from grid concentrations results of air dispersion models.  The method identifies land-sea breeze effects that can be used to manage local air quality in areas of similar micro-climates. Additionally, the paper uses a novel risk based method to identify areas within zones that have high exposure risks based on satellite-derived imagery.

\section{Third paper}
\noindent
\textbf{Evaluation of air quality zone classification methods based on ambient air concentration exposure}. Freeman, E. McBean, B., Gharabaghi, J.  Th\'e, (2017). \textit{Journal of the Air and Waste Management Association}, Volume 67, Issue 5, December 2016, Pages 550-564, DOI: 10.1080/10962247.2016.1263585.

\vspace{5mm}
\noindent
\textbf{Context}\\
\noindent
Air quality zones are used by regulatory authorities to implement ambient air standards in order to protect human health. Air quality measurements at discrete air monitoring stations are critical tools to determine whether an air quality zone complies with local air quality standards or is non-compliant. This study presents a novel approach for evaluation of air quality zone classification methods by breaking the concentration distribution of a pollutant measured at an air monitoring station into compliance and exceedance probability density functions (PDFs) and then using Monte Carlo analysis with the Central Limit Theorem to estimate long-term exposure. The purpose of this paper is to compare the risk associated with selecting one ambient air classification approach over another by testing the possible exposure an individual living within a zone may face. The chronic daily intake (CDI) is utilized to compare different pollutant exposures over the classification duration of 3 years between two classification methods. Historical data collected from air monitoring stations in Kuwait are used to build representative models of 1-hr NO$_{2}$ and 8-hr O$_{3}$ within a zone that meets the compliance requirements of each method. The first
method, the “3 Strike” method, is a conservative approach based on a winner-take-all approach common with most compliance classification methods, while the second, the 99\% Rule method, allows for more robust analyses and incorporates long-term trends. A Monte Carlo analysis is used to model the CDI for each pollutant and each method with the zone at a single station and with multiple stations. The model assumes that the zone is already in compliance with air quality standards over the 3 years under the different classification methodologies. The model shows that while the CDI of the two methods differs by 2.7\% over the exposure period for the single station case, the large number of samples taken over the duration period impacts the sensitivity of the statistical tests, causing the null hypothesis to fail. Local air quality managers can use either methodology to classify the compliance of an air zone, but must accept that the 99\% Rule method may cause exposures that are statistically more significant than the 3 Strike method. 

\vspace{5mm}
\noindent
\textbf{Contributions}\\
\noindent
This paper presents a novel method to directly compare different air standard compliance classification methods using the Central Limit Theorem and Monte Carlo Analysis to estimate the chronic daily intake of pollutants. This method allows air quality managers to rapidly see how individual classification methods may impact individual population groups, as well as evaluate different pollutants based on dosage and exposure when complete health impacts are not known.

\section{Fourth paper}
\noindent
\textbf{Estimating Annual Air Emissions from $nargyla$ Water Pipes in Caf\'es and Restaurants Using Monte Carlo Analysis}. Freeman, B., Gharabaghi, J.  Th\'e, (2017). \textit{International Journal of Environmental Science and Technology}. DOI: 10.1007/s13762-018-1662-6

\vspace{5mm}
\noindent
\textbf{Context}\\
\noindent
Smoking \textit{sheesha} with a \textit{nargyla} water pipe is a popular pastime in many Arab countries and increasingly in use around the world.  The process uses charcoal to heat flavored tobacco which is then drawn through a water filter by the user.  Emissions come primarily from the combustion of the charcoal and the flavored tobacco.  Many studies have been conducted to determine the health effects on users, but no studies have looked at the cumulative effect of burning charcoal in caf\'es and restaurants on the overall national emissions inventory and carbon footprint.  This presentation looks at the combined generation of greenhouse gases and hazardous air pollutants using previous studies and published emission factors to provide annual estimates for inclusion into national emission inventory reports.  This new approach employs Monte Carlo Analysis to account for uncertainty and variance in input parameters, and used by the authors to estimate annual emissions from cafés and restaurants in Kuwait.  The results of this evaluation showed that selecting the underlying distribution for the emission factors significantly impacted the annual totals.  Additionally, the resulting totals of emissions from \textit{nargyla} smoking represent a significant portion (in the case of nitrous oxide, over 50\%) of the total emissions inventory when compared with official totals submitted under the United Nations Framework Convention on Climate Change.

\vspace{5mm}
\noindent
\textbf{Contributions}\\
\noindent
This paper uses a novel way to evaluate small, distributed area sources using Monte Carlo Analysis. In addition to looking at the air emissions generated from $nargyla$ smoking, the paper uses the output to compare with national greenhouse gas inventories in order to evaluate the impact of overlooked sources.

\section{Fifth paper}
\noindent
\textbf{Forecasting Air Quality Time Series Using Deep Learning}. Freeman, G. Taylor, B., Gharabaghi, J.  Th\'e, (2018).  \textit{Journal of the Air and Waste Management Association}, Volume 67, Issue 5, December 2016, Pages 550-564, DOI: 10.1080/10962247.2016.1263585.

\vspace{5mm}
\noindent
\textbf{Context}\\
\noindent
This paper presents one of the first applications of advanced Deep Learning techniques in the prediction of air quality time series events and classification of patterns within specific measurement windows. Air quality management relies extensively on time series data captured at air monitoring stations as the basis of identifying population exposure to airborne pollutants and determining compliance with local ambient air standards. In this paper, 8 hour averaged surface ozone (O$_{3}$) concentrations were predicted using Deep Learning (DL) consisting of a recurrent neural network (RNN) with long short-term memory (LSTM). Hourly air quality and meteorological data were used to train and forecast values up to 72 hours with very low error rates. The LSTM was able to forecast the duration of continuous O$_{3}$ exceedances as well. Prior to training the network, the dataset was reviewed for missing data and outliers. Missing data were imputed using a novel technique that averaged gaps less than 8 time steps with incremental steps based on first order differences of neighboring time periods. Data were then used to train decision trees in order to evaluate input feature importance over different time prediction horizons. The only processing of data prior to network training was to normalize each variable between 0 and 1. The number of features used to train the LSTM model was reduced from 25 features to 5 features, resulting in improved accuracy as measured by Mean Absolute Error (MAE). Parameter sensitivity analysis identified look-back nodes associated with the RNN proved to be a significant source of error if not aligned with the prediction horizon. Overall, MAE's less than 2 were calculated for predictions out to 72 hours. 

\vspace{5mm}
\noindent
\textbf{Contributions}\\
\noindent
This paper uses deep learning techniques to train an 8-hour averaged ozone forecast model. Missing data and outliers within the captured data set were replaced using a new imputation method that generated calculated values closer to the expected value based on the time and season. Decision trees were used to identify input variables with the greatest importance. The novel methods presented in this paper allow air managers to forecast long range air pollution concentration while only monitoring key parameters and without transforming the data set in its entirety, thus allowing real time inputs and continuous prediction.

\section{Sixth paper}
\noindent
\textbf{Using Unmanned Aerial Systems to Estimate Vehicle Stacking at Signalled Intersections}.  Freeman,  B., J. Al Matawah, M. Al Najat, Gharabaghi, J.  Th\'e, (2018). Submitted to \textit{International Journal of Transportation Science and Technology}, In Review.

\vspace{5mm}
\noindent
\textbf{Context}\\
\noindent
Fleet composition and vehicle spacing on roads are important inputs to mobile source emission models and traffic planning. In this paper, we present a novel method that uses an unmanned aerial system (UAS) to capture imagery of stationary vehicle formations at two different intersections and times of day. The imagery is processed using photogrammetric software to generate 3 dimensional models of the formations that allow measurement of the stacking gaps and identification of individual vehicle types for fleet composition evaluation. Statistical tests were performed on the different flight results to determine if the traffic behavior (both composition and gaps) were similar and can be pooled. In both cases, the variation of fleet composition and gaps were similar, however, the stationary headway gaps followed a logarithmic distribution and had to be transformed prior to pooling. The final results of the fleet composition measured varied significantly from the estimated mix based on registered vehicles, while the average vehicle spacing was approximately 2.2 m and did not depend on vehicle types. These results were used to prepare a Monte Carlo Analysis model to estimate the total number and types of vehicles on a 1 km road section. The model was extended from stationary traffic to traffic moving up to 20 km/h by assuming a linear increase of the spacing gap. This paper is one of the first of its kind to study the stacking spaces of mixed fleets at signalled intersections and shows that spacing is dependent more on individual driver behavior than vehicle type.

\vspace{5mm}
\noindent
\textbf{Contributions}\\
\noindent
This paper demonstrated the novel use of a UAS for flexible data collection of stationary traffic and photogrammetry methods to produce a measurable model for data collection. The results of the research demonstrated that common fleet distributions exist at similar intersections distributed in different locations and at different times of the day share similar variance and can be pooled for aggregated results. The major contribution is that results can be applied to other similar intersections to estimate fleet distribution and stacking space for vehicle density estimation.



\clearpage

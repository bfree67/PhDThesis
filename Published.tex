%\appendix

\chapter{Published Papers}

\section{First paper}
\textbf{Estimation of Mixed Traffic Densities in Congested Roads using Monte Carlo Analysis}. Freeman, B., Gharabaghi, J.  Th\'e, (2015). \textit{EM}, April 2015, pages 8-13. 

\subsection{Context}
Knowing the amount and type of vehicles on a segment of road is essential to evaluating mobile source emissions, traffic network planning, and emergency response. If intelligent traffic systems are not available to monitor and track traffic, a model is needed to estimate these parameters.

\subsection{Contributions}
This paper presents a novel model to estimate vehicle densities using Monte Carlo Analysis. National fleet characteristics are used to create the initial distribution of vehicles on the road with the number of vehicles in a kilometer segment based on independent variables that are a function of the average traffic speed. 

\section{Second paper}

\textbf{Mapping air quality zones for coastal urban centers}. Freeman, B., Gharabaghi, J.  Th\'e, S. Faisal, M. Abdullah, and A. Al-Aseed (2017). \textit{Journal of the Air and Waste Management Association}, Volume 67, Issue 5, December 2016, Pages 565-581, DOI 10.1080/10962247.2016.1265025.

\subsection{Context}
Air quality zones are a key element to air management programs and usually assigned based on existing political boundaries without accounting for atmospheric effects as a function of land use and geographical features. Accurately defined air zones can assist compliance activities by focusing on areas with low air quality and preventing deterioration of areas with good air quality.

\subsection{Contributions}
This paper presents a novel method to determine air quality zones in coastal urban areas is introduced using skewness (S) and kurtosis (K) statistics calculated from grid concentrations results of air dispersion models.  The method identifies land-sea breeze effects that can be used to manage local air quality in areas of similar micro-climates.

\section{Third paper}

\textbf{Evaluation of air quality zone classification methods based on ambient air concentration exposure}. Freeman, E. McBean, B., Gharabaghi, J.  Th\'e, (2017). \textit{Journal of the Air and Waste Management Association}, Volume 67, Issue 5, December 2016, Pages 550-564, DOI 10.1080/10962247.2016.1263585.
\subsection{Context}
Once air zones are designated, a method is needed to evaluate whether the zone meets air quality standards or does not. A single measurement value of a specific pollutant will not represent the air quality of an entire area, especially if there are multiple stations within the zone. 
\subsection{Contributions}
This paper presents a novel method to directly compare different air standard compliance classification methods using the Central Limit Theorem and Monte Carlo Analysis to estimate the chronic daily intake of pollutants. This method allows air quality managers to rapidly see how individual classification methods may impact individual population groups, as well as evaluate different pollutants based on dosage and exposure when complete health impacts are not known.

\section{Fourth paper}

\textbf{Estimating Annual Air Emissions from $nargyla$ Water Pipes in Cafes and Restaurants Using Monte Carlo Analysis}. Freeman, B., Gharabaghi, J.  Th\'e, (2017). \textit{International Journal of Environmental Science and Technology}. Accepted 16 Oct 2017.
\subsection{Context}
Small, distributed area sources are often not reported or under-reported in national inventories of hazardous air pollutants and greenhouse gases. 
\subsection{Contributions}
This paper uses a new way to evaluate small, distributed area sources using Monte Carlo Analysis. In addition to looking at the air emissions generated from $nargyla$ smoking, the paper uses the output to compare with national greenhouse gas inventories in order to evaluate the impact of overlooked sources.

\section{Fifth paper}

\textbf{Forecasting Air Quality Time Series Using Deep Learning}. Freeman, G. Taylor, B., Gharabaghi, J.  Th\'e, (2017). Submitted to \textit{Journal of the Air and Waste Management Association}.
\subsection{Context}
Air pollutant concentrations are continuous and dependent time series processes that are often impacted by precursor analytes and meteorological conditions. Deep learning techniques using recurrent neural networks can utilize the historical patterns within air quality concentrations as well as prepare data sets for input by replacing missing or censored data, and identifying outliers.

\subsection{Contributions}
This paper uses deep learning techniques to train an 8-hour averaged ozone forecast model. Missing data and outliers within the captured data set were replaced using a new imputation method that generated calculated values closer to the expected value based on the time and season. Decision trees were used to identify input variables with the greatest importance. The novel methods presented in this paper allow air managers to forecast long range air pollution concentration while only monitoring key parameters and without transforming the data set in its entirety, thus allowing real time inputs and continuous prediction.

\clearpage

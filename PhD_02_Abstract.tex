\begin{abstract}
\pagestyle{empty} \addtocounter{page}{-1}
%\renewcommand\thepage{\roman{page}}

\noindent Brian S. Freeman \hfill Advisory Team:\\
\noindent University of Guelph, 2018 \hfill Bahram Gharabaghi, Jesse Th\'e\\

Low ambient air quality is a leading cause of premature deaths for over 92\% of the world's population. Much of this at-risk population live and work in urban centers on or near coastlines and in developing nations.  Coastal communities face enhanced air quality problems due to the continuous circulation of land-sea breezes that return previously emitted primary and secondary air pollutants into airsheds already burdened by continuous emissions from industrial and mobile sources.  For many nations impacted by concentrated air pollution, regulatory agencies do not have the resources to develop comprehensive air management programs similar to the mature programs in the USA and Europe.  This research addressed specific air management issues faced by regulatory agencies to allow better oversight given constrained budgets and technical staff. Data sets were collected from different air monitoring stations in Kuwait to allow development, testing and validation of the different techniques described with the research.\\

The novel methods developed as a result of this research combined ambient air monitoring station time series data with statistical testing, stochastic methods and machine learning processes. The fundamental questions looked at were:

\begin{itemize}
\item{What areas within an airshed have common mixing and exposure patterns?}
\item{How can a designated zone be classified as exceeding or not exceeded ambient air standards given multiple air monitoring stations?}
\item{How to quantify small dispersed area sources for inclusion into emission inventories?}
\item{How to predict air pollutant concentrations based on historical and prevailing ambient conditions?}
\item{How can mobile sources be quantified in order to provide input to emission models?}
\end{itemize}

The first question was addressed by modelling virtual sources throughout a region using an advanced Lagrangian puff air dispersion model and prognostic meteorological data.  Evaluating the distribution of 1 year concentration averages showed that relationships between the distributions’ skewness and kurtosis statistics could predict whether the dispersion patterns were influenced by land-sea breezes or the inland prevailing winds. By identifying the limit of land-sea breeze effect, air zone boundaries could be established to better manage individual sources.  The resulting skewness and kurtosis statistics, along with distance from the shore were used to train a support vector machine model, which could discriminate between coastal and inland dispersion effects with a 98\% accuracy. \\

The second question regarding evaluation of a designated air management zone in regards to limits exceedance was addressed by addressing random exposure over three years based on local exceedance standards.  Uncertainty within exposure variables was managed using Monte Carlo Analysis that allocated an unlimited number of air monitoring stations located within the zone to contribute to the overall evaluation result and allow different characterization methods to be compared.  The Central Limit Theorem was used to simplify calculations.\\

Small dispersed area sources include smoking, BBQs, and open burning. In the Middle East,\textit{ nargyla}, or hookah smoking is common.  A model using Monte Carlo Analysis was developed that also employed the Central Limit Theorem to account for wide variation of individual caf\'es and restaurants where \textit{nargyla} smoking takes place.  One result of the analysis revealed that not including these small individual sources significantly impacts overall annual emissions for small countries, like Kuwait, where smoking was calculated to represent over 20\% of all nitrous oxide produced in 1995 based on Kuwait’s Initial Communication to the United Nations Framework for Climate Change Convention. \\

Predicting future air pollution concentrations was accomplished using a deep learning technique consisting of a recurrent neural network with long short term memory.  Datasets from a single air monitoring station was used to train and test the model. Prior to training, the data had to be cleaned using a novel first-order imputation technology to estimate missing data and outliers.  A decision tree method was adopted to estimate importance of individual features while a scaling method was used to limit input values to between 0 and 1.  The resulting model was able to predict 8 hour averaged ozone out to 72 hours with a Mean Absolute Error of less than 2 ppb, outperforming traditional methods using multi-layer feed forward networks and ARIMA methods.\\

Lastly, estimating mobile source emissions is one of the biggest challenges for air managers. Many models exist to calculate annual amounts but each require local inputs based on the number and types of vehicles on the road. Collecting these values is a time consuming and expensive process that must be update annually to account for different road use.  The developed method to determine vehicle fleet composition and vehicle density on a segment of road employed an unmanned aerial system (drone) to capture images of cars stacked at a signalled intersection.  The collected images were processed using photogrammetry software to create a digital elevation model that allowed measurement of stacking distances and identification of individual vehicles.  Using data collected from two similar intersections at different times of the day, both fleet composition and stacking distances were found to not vary significantly (p$>$0.05).  The stacking distance followed a log-normal distribution and was transformed prior to significance testing.  The data was assumed to have equal variance and could be pooled and used as parameters for mobile source emission models.\\

As a result of these novel techniques, key requirements of an air quality management program can be rapidly and economically implemented.  This research recommends further evaluation of each technique with real world scenarios in order to further validate their effectiveness. 


\normalsize
\end{abstract}

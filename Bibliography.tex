\setlength\parindent{0pt}

\chapter*{References}

\newenvironment{hangingpar}[1]
  {\begin{list}
          {}
          {\setlength{\itemindent}{-#1}%%'
           \setlength{\leftmargin}{#1}%%'
           \setlength{\itemsep}{0pt}%%'
           \setlength{\parsep}{\parskip}%%'
           \setlength{\topsep}{\parskip}%%'
           }
    \setlength{\parindent}{-#1}%%
    \item[]
  }
  {\end{list}}


\begin{hangingpar}{2em}
Akaike, H., 1973. Information theory as an extension of the maximum likelihood principle. In: Petrov, B.N., Csaki, F. (Eds.), Second International Symposium on Information Theory, Akademiai Kiado, Budapest, 267-281.

Arnold Jr., C.L., Gibbons, G.C., 1996. Impervious surface coverage: the emergence of a key environmental indicator. J. Am. Plan. Assoc., 62 (2), 243-259.

Akinremi, O., Mcginnand , S., Cutforth, h., 1999. ``Precipitation trends in the Canadian Prairies''. Journal of Climatology 12: 2996–3003
ASCE, 2000. Artificial neural networks in hydrology I: Preliminary concepts. J. Hydrol. Eng., 5(2), 115-123.

Asselman, N.E.M., 2000. Fitting and interpretation of sediment rating curves. J. Hydrol. 234 (3), 228-248.

Abrahart, R.J., White, S.M., 2001. Modeling sediment transfer in Malawi: comparing backpropagation neural network solutions against a multiple linear regression benchmark using small data sets. Phys. Chem. Earth (B) 26 (1), 19-24.

Alp, M.H., Cigizoglu, K., 2007. Suspended sediment load simulation by two artificial neural network methods using hydrometeorological data. Environ. Model. Softw. 22, 2-13. 

Akaike, H., 1973. Information theory as an extension of the maximum likelihood principle. In: Petrov, B.N., Csaki, F. (Eds.), Second International Symposium on Information Theory, Akademiai Kiado, Budapest, 267-281.

Arnold Jr., C.L., Gibbons, G.C., 1996. Impervious surface coverage: the emergence of a key environmental indicator. J. Am. Plan. Assoc., 62 (2), 243-259.

Akinremi, O., Mcginnand , S., Cutforth, h., 1999. ``Precipitation trends in the Canadian Prairies''. Journal of Climatology 12: 2996–3003
ASCE, 2000. Artificial neural networks in hydrology I: Preliminary concepts. J. Hydrol. Eng., 5(2), 115-123.

Asselman, N.E.M., 2000. Fitting and interpretation of sediment rating curves. J. Hydrol. 234 (3), 228-248.

Abrahart, R.J., White, S.M., 2001. Modeling sediment transfer in Malawi: comparing backpropagation neural network solutions against a multiple linear regression benchmark using small data sets. Phys. Chem. Earth (B) 26 (1), 19-24.

Alp, M.H., Cigizoglu, K., 2007. Suspended sediment load simulation by two artificial neural network methods using hydrometeorological data. Environ. Model. Softw. 22, 2-13.

Ashraf, M., Kahlown, M. A., and Ashfaq, A., 2007. Impact of small dams on agriculture and groundwater development: A case study from Pakistan.  Agric. Water Manage., 92(1-2), 90-98.

Aytek, A., Kisi, O., 2008. A genetic programming approach to suspended sedimentmodeling. J. Hydrol. 351 (3), 288-298

Alcazar, J., Palaua,A., Vega-Garcı , C., 2008.  A neural net model for environmental flow estimation at the Ebro River Basin, Spain.  J. Hydrol., 349, 44-55.

Adamowski, K., Prokoph, A., Adamowski, J., 2009. ``Development of a new method of wavelet aided trend detection and estimation''. Hydrological Processes 23 (18): 2686-2696

Archfield, S., Vogel, R., 2010. Map correlation method: selection of a reference stream gauge to estimate daily stream flow at ungauged catchments. Water Resour. Res. 46 (10), W10513.

Arrigoni, A.S., Greenwood, M.C., Moore, J.N., 2010. Relative impact of anthropogenic modifications versus climate change on the natural flow regimes of rivers in the Northern Rocky Mountains, United States. Water Resour. Res., 46(12), W12542.

Archfield, S., Vogel, R., Steeves, P., et al., 2010. The Massachusetts Sustainable‐Yield Estimator: A decision‐support tool to assess water availability at ungauged sites in Massachusetts, U.S. Geological Survey Scientific Investigations Report, 2009-5227.

Archfield, S., Vogel, R., 2010. Map correlation method: Selection of a reference stream gauge to estimate daily stream flow at ungauged catchments. Water Resour. Res., 46 (10), W10513.

Assani, A. A., Landry, R., Daigle, J., Chalifour, A., 2011. Reservoirs effects on the interannual variability of winter and spring streamflow in the St-Maurice River Watershed (Quebec, Canada). Water Resour. Manage., 25, 3661-3675.

Azamathulla, H., 2012. Gene expression programming for prediction of scour depth downstream of sills. J. Hydrol., 460-461, 156-159. 

Archfield, S.A. Steeves, P.A, Guthrie, J.D., and Rie, K.G., 2013.  Towards a publicly available, map-based regional software tool to estimate unregulated daily streamflow at ungauged rivers. Geosci. Model Dev., 6, 101-115.

Adamowski, J. Adamowski, K., Prokoph, A., 2013. ``Quantifying the spatial temporal variability of annual streamflow and meteorological changes in eastern Ontario and southwestern Quebec using wavelet analysis and GIS''. Journal of Hydrology 499:  27-40.

Ahmed, S., Singh, A., Rudra, R., Gharabaghi, B., 2013. ``Comparison of CANWET and HSPF for water budget and water quality modeling in rural Ontario''.   Water Quality Research Journal of Canada 49 (1): 53-71.

Atlas of Canada, 2014. Natural Resources Canada. \\$<$http://atlas.nrcan.gc.ca/site/english/index.html$>$ (accessed January, 2014). 

Abril, M., Muñoz, I., Casas-Ruiz, J.P., Gómez-Gener, L., Barceló, M., Oliva, F., Menéndez, M., 2015. Effects of water flow regulation on ecosystem functioning in a Mediterranean river network assessed by wood decomposition. Sci. Total Environ., 517, 57-65.

Atieh, M., Gharabaghi, B., and Rudra, R., 2015a. Entropy based neural networks model for flow Duration Curves at ungauged sites. J. Hydrol., 529 (3), 1007-1020.

Atieh, M., Mehltretter, S., Gharabaghi, B., and  Rudra, R., 2015b. Integrative neural networks model for prediction of sediment rating curve parameters for ungauged basins. J. Hydrol.,  531(3), 1095-1107.

Azmat, M., Laio, F., Poggi, D., 2015. Estimation of water resources availability and mini-hydro productivity in high-altitude scarcely-gauged watershed. Water Resour. Manage., 29(14), 5037-5054.

Asnaashari, A., Gharabaghi, B., McBean, E., Mahboubi, A. A., 2015. ``Reservoir Management under Predictable Climate Variability and Change''.  Journal of Water and Climate Change 6 (3): 472-485.

Al-Juboori, A.M., Guven, A.	, 2016. Hydropower Plant Site Assessment by Integrated Hydrological Modeling, Gene Expression Programming and Visual Basic Programming.Water Reso. Mngmt., 30 (7): 2517-2530

Atieh, M., Rudra , R., Gharabaghi,B., Golmohammadi, G.,  Mohammadi, K.2016. Addressing irrigation water deficiency in Lebanon. International Journal of Water Resources Development. Accepted June 2016.

Burmaster, D., Anderson, P. 1994. Principles of good practice for the use of Monte Carlo techniques in human health and ecological risk assessment. Risk Anal., 14, 477-481

Benjamin, L., Van Kirk, R., 1999. Assessing instream flows and reservoir operations on a Eastern Idaho River. J. Am. Water Resour. Assoc. 35(4), 899-909. 

Bowden G. J., Dandy G. C., Maier H.R., 2005.  Input determination for neural network models in water resources applications. Part 1-background and methodology. J. Hydrol., 301, 75-92.

Boyer, C., Chaumont, D., Chartier, I., Roy, A.G., 2010. Impact of climate change on the hydrology of St. Lawrence tributaries. J. Hydrometeorol., 384 (1–2), 65-83.

Besaw, L.E., Donna M.R., Bierman, P.R., William, R.H., 2010.  Advances in ungauged streamflow prediction using artificial neural networks. J. Hydrol., 386 (1-4), 27-37

Brown, L. R., Bauer, M. L., 2010. Effects of hydrologic infrastructure on flow regimes of California’s Central Valley Rivers: Implications for fish populations.  River Res. Appl. 26, 751-765.

Boyer, C., Chaumont, D., Chartier, I., Roy, A.G., 2010. ``Impact of climate change on the hydrology of St. Lawrence tributaries''. Journal of Hydrometeorology 384 (1–2): 65-83.

Biemans, H., Haddeland, I., Kabat, P., Ludwig, F., Hutjes, W.A., Heinke, J., et al., 2011. Impact of reservoirs on river discharge and irrigation water supply during the 20th century. Water Resour. Res., 47(3), W03509. 

Baker, D.W., Bledsoe, B.P., Albano, C.M., Poff, N.L., 2011. Downstream effects of diversion dams on sediment and hydraulic conditions of Rocky Mountain streams. River Res. Appl., 27(3), 388-401.

Bengio, Yoshua, Frédéric Bastien, Arnaud Bergeron, Nicolas Boulanger-Lewandowski, Thomas M. Breuel, Youssouf Chherawala, Moustapha Cisse, et al. 2011. ``Deep Learners Benefit More from Out-of-Distribution Examples.'' In AISTATS, 164-72.

Booker, D., Snelder, T., 2012. Comparing methods for estimating flow Duration Curves at ungauged sites. J. Hydrol. 434-435, 78-94. 

Betts, A. R., Gharabaghi, B., McBean, E. A., 2014.  ``Salt Vulnerability Assessment Methodology for Urban Streams''.   Journal of Hydrology, 517: 877-888.

Betts, A. R., Gharabaghi, B., McBean, E. A., Levison, J., and Parker, B., 2015. ``Salt vulnerability assessment methodology for municipal supply wells''. Journal of Hydrology 531 (3): 523-533.

Betts, A., Gharabaghi, B., McBean, E., Levison, J., 2015.  ``Salt Vulnerability Assessment Methodology for Municipal Supply Wells''.  Journal of Hydrology 531(3):523-533.

Burn, D.H., Whitfield, P.H., Sharif, M. , 2016. Identification of changes in floods and flood regimes in Canada using a peaks over threshold approach. Hydrological Processes

Castellarin, A. Galeati,G., Brandimarte, L., 2004. Regional flow duration curves: Reliability for ungauged basins. Adv. Water Resour., 27 (10), 953-965. 

Cigizoglu, H.K., 2004. Estimation and forecasting of daily suspended sediment data by multi-layer perceptrons. Adv. Water Resour. 27, 185-195.

Cheng, Q., Koa, C., Yuana, Y., Gea, Y., and Zhang, S., 2006. GIS modeling for predicting river runoff volume in ungauged drainages in the Greater Toronto Area, Canada. Comput. Geosci., 32 (8), 1108-1119.

Coulibaly, P., 2006. Spatial and temporal variability of Canadian seasonal precipitation (1900-2000). Adv. Water Resour., 29 (12), 1846-1865.

Cigizoglu, H.K., 2006. Application of the generalized regression neural networks in intermittent flow forecasting and estimation. J. Hydrol. Eng. 10 (4), 336-341.

Castellarin, A., Camorani, G., and Brath, A., 2007. Predicting annual and long-term flow-duration curves in ungauged basins. Adv. Water Resour., 30 (4), 937-953.

Crowder, D.W., Demissie, M., Markus, M., 2007. The accuracy of sediment loads when log-transformation produces nonlinear sediment load–discharge relationships. J. Hydrol. 336 (3), 250-268.

Cobaner, M., Unal, B., Kisi, O., 2009. Suspended sediment concentration estimation by an adaptive neuro-fuzzy and neural network approaches using hydrometeorological data. J. Hydrol. 367 (1), 52-61.

Castellarin, A.2014. Regional prediction of flow duration curves using a three dimensional kriging. J.of Hydrol. 513,179-191. 

Casadei, S., Liucci, L., Valigi, D., 2014. Hydrological uncertainty and hydropower: New methods to optimize the performance of the plant.  Energy Procedia, 59, 263-269.

Chen, Q., Zhang, X., Chen, Y., Li, Q., Qiu, L., Liu, M., 2015. Downstream effects of a hydropeaking dam on eco-hydrological conditions at subdaily to monthly time scales. Ecol. Eng., 77, 40-50.

CDIAC (Carbon Dioxide Information Analysis Center), 2016. United States Historical Climatology Network.\\ $<$http://cdiac.ornl.gov/epubs/ndp/ushcn/ushcn\_map\_interface.html$>$ (accessed January, 2016).

Cui, H., Singh, V. 2016. ``Maximum entropy spectral analysis for streamflow forecasting''. Physica A: Statistical Mechanics and its Applications 442: 91-99.

Coch, A., Mediero, L.	2016. Trends in low flows in Spain in the period 1949-2009		Hydrological Sciences Journal, 1-17

Dugdal,J.S., 1996. Entropy and its Physical Meaning. Taylor and Francis publishers.

Daqing Y, Baisheng Y., Douglas L. K, 2004. Streamflow changes over Siberian Yenisei River Basin. J.  Hydr., 296, 59-80.

Das, S., Rudra, R., Gharabaghi, B., Gebremeskel, S., Goel, P. K., Dickinson, W.T., 2008.  ``Applicability of AnnAGNPS for Ontario conditions''.   Canadian Biosystems Engineering, 50 (1): 1-11.

Demirel, M. Venancio, A. and Kahya, E., 2009. Flow forecast by SWAT model and ANN in Pracana basin, Portugal.  Adv. Eng. Soft., 40 (7), 467-473.

Demirci, M., Baltaci, A., 2013. Prediction of suspended sediment in river using fuzzy logic and multi-linear regression approaches. Neural Comput. Appl. 23 (Suppl. 1), 145-151.

Disley, T., Gharabaghi, B., Mahboubi, A., \& McBean, E., 2015.  ``Predictive equation for longitudinal dispersion coefficient''.   Hydrological Processes 29 (2): 161-172.

ECDE, 2012. Environment Canada Data Explorer: HYDAT version 1.0. Available at https://ec.gc.ca/rhc-wsc/default.asp?lang=En\&n=0A47D72F-1 . Accessed September 2013.

EC, 2012. Environment Canada Climate Normals (1981-2010). Available at \\http://climate.weather.gc.ca/climate\_normals/index\_e.html. Accessed September 2014

EC, 2014. Environment Canada Adjusted Precipitation Data. Available at \\http://climate.weather.gc.ca/climate\_normals/index\_e.html . Accessed November 2014.

Eng.,K., Carlisle,D.M., Wolock,D.M., Falcone,J.A., 2013a. Predicting the likelihood of altered streamflows at ungauged rives across the conterminous United States. River Res. Appl.,  29(6),781-791.

Eng., K., Wolock,D.M.,Carlisle,D.M., 2013b.Riverflow changes related to land and water management practices across the conterminous United States. Sci. Total Environ., 463-464, 412-422.

El-Hakeem, M., Sattar, A. M., 2015. An entrainment model for non-uniform sediment. Earth Surf. Process. Landf., 40(9), 1216-1226.

Horowitz, A.J., 2003. An evaluation of sediment rating curves for estimating suspended sediment concentrations for subsequent flux calculations. Hydrol. Process. 17 (17), 3387-3409.

Fotiadi. A.K., Metaxas, D.A. Bartzokas, A.,1999.'' A Statistical study of precipitation in Northwest Greece''. International Journal of Climatology 19: 1221-1232. 

Ferreira, C., 2001. Gene expression programming: a new adaptive algorithm for solving problems. Complex Systems, 13 (2), 87-129.

Finney, K., Gharabaghi, B., McBean, E. A., Rudra, R. P., and MacMillan, G., 2010. ``Compost Biofilters for Highway Stormwater Runoff Treatment''.   Water Quality Research Journal of Canada, 45 (4): 391-402.

Fan, X., Shi, C., Zhou, Y., Shao, W., 2012. Sediment rating curves in the Ningxia-Inner Mongolia reaches of the upper Yellow River and their implications. Quatern. Int. 282, 152-162.

Farmer, H. W., Vogel, R., 2013.  Performance-weighted methods for estimating monthly streamflow at ungauged sites. J. Hydrol., 477, 240-250.

Graf , W.L. , 2006. Downstream hydrologic and geomorphic effects of large dams on American rivers Geomorphology 79, 336-360.

Guven, A., Gunal, M., 2008. Genetic programming approach for prediction of local scour downstream of hydraulic structures. J. Irrig. Drain Eng., 134(2), 241-249.

Gao,S., Wang,Y.P., 200).Changes in material fluxes from the Changjiang River and their implications on the adjoining continental shelf ecosystem. Continental Shelf Research, 28, 1490-1500.

Guo, H., Hu, Q., Zhang, Q., Feng, S., 2012. Effects of the Three Gorges Dam on Yangtze River flow and river interaction with Poyang Lake, China: 2003-2008. J. Hydrol., 416-417, 19-27.

Ghanbarpour, M.R., Zolfaghari, S., Geiss, C., Darvari,Z, 2013. Investigation of river flow alterations using environmental flow assessment and hydrologic indices: Tajan River Watershed, Iran. Int. J.River Basin Manage., 11(3), 311-321.

Giafagna, C.C., Johnson, C.E., Chandler, D.G., Hofmann, C., 2015.Watershed area ration accurately predicts daily streamflow nested in catchments in the Catskills, Newyork. J.Hydrol.: Regional studies, 4, 583-594.

Gonzalez del Tanago M., Bejarano, M.D., Garcia de Jalon, D., Schmidt, J.C., 2015. Biogeomorphic responses to flow regulation and fine sediment supply in Mediterranean streams (the guadalete River, southern Spain). J. Hydrol., 528, 751-762. 

Gazendam, E., Gharabaghi, B., Ackerman, J.,and Whiteley, H., 2016.  ``Integrative Neural Networks Models for Stream Assessment in Restoration Projects''.  Journal of Hydrology 536: 339-350. 

Holmes, M.G.R., Young, A.R., Gustard, A., Grew, R., 2002. A region of influence approach to predicting flow Duration Curves within ungauged catchments. Hydrol. Earth Syst. Sci. 6 (4), 721-731.

Hu, T.S., Wu, F.Y., Zhang, X., 2007. Rainfall–runoff modeling using principal component analysis and neural network. Nordic Hydrol. 38 (3), 235-248

Hao, P. and Singh, V.P., 2011. Single-site monthly stream flow simulation using entropy theory. Water Resour. Res., 47 (9), W09528.

Hu, B., Wang, H., Yang, Z., Sun, H., 2011. Temporal and spatial variations of sediment rating curves in the Changjiang (Yangtze River) basin and their implications. Quatern. Int. 230, 34-43.

Hwang, Y., Clark, M., Rajagopalan, B.,  Leavesley, G., 2012. ``Spatial interpolation schemes of daily precipitation for hydrologic modelling''. Stochastic Environmental Research Risk Assessment 26:295-320.

Hrachowitz, M., Savenije, H.H.G., Blöschl, G., McDonnell, J.J., et al., 2013. A decade of predictions in ungauged basins (PUB)—a review. Hydrol. Sci. J., 58 (6), 1198-1255.

Heng, S., Suetsugi, T., 2013. Using artificial neural network to estimate sediment  load in ungauged catchments of the Tonle Sap River Basin, Cambodia. J. Water Res. Prot. 5, 111-123.

Hrachowitz, M., Savenije, H.H.G., Blöschl, G., McDonnell, J.J., et al., 2013. A decade ofpredictions in ungauged basins (PUB)—a review. Hydrol. Sci. J. 58 (6), 1198-1255.

Heng, S., Suetsugi, T., 2014. Comparison of regionalization approaches in parameterizing sediment rating curve in ungauged catchments for subsequent instantaneous sediment yield prediction. J. Hydrol. 512, 240-253.

Haghighi, A.T., Marttila, H., Klove, B., 2014. Development of a new index to assess river regime impacts after dam construction. Glob. Planet. Chang., 122, 186-196

Hannun, A., Case, C., Casper, J.,  Catanzaro, B., Diamos, G., Elsen, E., Prenger, R., et al. 2014. Deep Speech: Scaling up End-to-End Speech Recognition. arXiv [cs.CL]. arXiv. http://arxiv.org/abs/1412.5567.

Hashmi, M., and Shamseldin, A., 2014. Use of Gene Expression Programming in regionalization of flow duration curve. Adv. Water Resour., 68, 1-12.

Heng, S., Suetsugi, T., 2015. Regionalization of sediment rating curve for sediment yield prediction in ungauged catchments. Hydrol. Res. 46 (1), 26-38.

Hiltner, U.,  Brauninga A.,  Gebrekirstos, A., Huth, A.,  Fischer, R., 2016. ``Impacts of precipitation variability on the dynamics of a dry tropical montane forest''. Ecological Modelling 320: 92-101.

IPCC, 2007. In: Solomon, S., Qin, D., Manning, M., Chen, Z., Marquis, M., Averyt, K.B.,Tignor, M., Miller, H.L. (Eds.), Climate Change 2007: Working Group I Report: the Physical Science Basis. Cambridge University Press, Cambridge, United Kingdom and New York, NY, USA.

Isik, S., Kalin L., Schoonover, J.E., Srivastava, P.  Lockab, B., 2013.  Modeling effects of changing land cover/cover on daily streamflow: An artificial neural network and curve number based hybrid approach.  J. Hydrol. 485, 103-112

Isik, S., 2013. Regional rating curve models of suspended sediment transport for Turkey. Earth Sci. Inf. 6 (2), 87-98.

Jobson, J.D., 1992. Applied Multivariate Data Analysis, Categorical and Multivariate Methods, vol. II. Springer-Verlag, New York.

Jain, S.H., 2001. Development of integrated sediment rating curves using ANNs. J. Hydraul. Eng. 127, 30-37.

Kawachi, T., Maruyama, T., Singh, V.P., 2001. Rainfall entropy for delineation of water resources zones in Japan. J. Hydrol. 246, 36-44

Kisi, O., 2004a. Daily suspended sediment modelling using a fuzzy differential evolution approach. Hydrol. Sci. J. 49 (1), 183-197. 

Kisi, O., 2004b. Multi-layer perceptrons with Levenberg-Marquardt training algorithm for suspended sediment concentration prediction and estimation. Hydrol. Sci. J. 49 (6), 1025-1040.

Kingston, G. B., Lambert, M. F., Maier H. R., 2005. Bayesian training of artificial neural networks used for water resources modeling. Water Resour. Res., 41 (12), W12409.

Khan, M., Coulibaly, P., 2006. Bayesian neural network for rainfall-runoff modeling. Water Resour. Res., 42 (7), W07409.

Kaldellis J. K., 2007.  The contribution of small hydro power stations to the electricity generation in Greece: technical and economic considerations. Energy Policy, 35 (4), 2187-2196.

Kisi, O., 2008. Constructing neural network sediment estimation models using a data-driven algorithm. Math. Comput. Simul. 79 (1), 94-103.

Kumar, S., Merwade, V., Kam, J., Thurner, K., 2009.'' Streamflow trends in Indiana: effects of long term persistence, precipitation and subsurface drains''. Journal of Hydrology 374 (1–2), 171-183.

Khan, M., Coulibaly, P., 2010. Assessing hydrologic impact of climate change with uncertainty estimates: Bayesian neural network approach. J. Hydrometeor., 11, 482-495. 

Khalil, B., Ouarda T.B., St-Hilaire, A., 2011. Estimation of water quality characteristics at ungauged sites using artificial neural networks and canonical correlation analysis. J. Hydrol., 405 (3-4), 277-287.

Khosravi, A., Nahavandi, S. Creighton, D. Atiya, A., 2011. Comprehensive Review of Neural Network-Based Prediction Intervals and New Advances. IEEE Trans. Neural Netw., 22 (9), 1341-1356.

Kisi, O., Shiri, J., 2012. River suspended sediment estimation by climatic variables implication: comparative study among soft computing techniques. Comput. Geosci. 43, 73-82.

Kisi, O., Ozkan, C., Akay, B., 2012. Modeling discharge-sediment relationship using neural networks with artificial bee colony algorithm. J. Hydrol. 428, 94-103.

Kalteh, M. (2013). Monthly river flow forecasting using artificial neural network and support vector regression models coupled with wavelet transform. Comp. and Geosc., 54, 1-8.

Kim, D., Kaluarachchi, J., 2014. Predicting streamflows in snowmelt-driven watersheds using the flow Duration Curves method. Hydrol. Earth Syst. Sci., 18, 1679-1693.

Karim, F., Dutta, D., Marvanek, S., Petheram, C., Ticehurst, C., Lerat, J., Kim, S., Yang, A., 2015. Assessing the impacts of climate change and dams on floodplain inundation and wetland connectivity in the wet-dry tropics of northern Australia. J. Hydrol., 522, 80-94.

Liong, S.Y., Gautam, T.R., Muttil, N. and Khu, S.T., 2000. ``Runoff forecasting with genetic programming'', Hydroinformatics 2000, IOWA, USA.

Lajoie, F., Assani, A. A., Roy, A. G., Mesfioui, M. , 2007. Impacts of dams on monthly flow characteristics. The influence of watershed size and seasons. J.  Hydrol., 334, 423-439.

Lohani, A.K., Goel, N.K., Bhatia, K.K.S., 2007. Deriving stage–discharge–sediment concentration relationships using fuzzy logic. Hydrol. Sci. J. 52 (4), 793-807.

Liu, Q., Yang, Z., Cui, B., 2008.'' Spatial and temporal variability of annual precipitation during 1961–2006 in Yellow River Basin, China''.  Journal of Hydrology 361(3-4): 330-338. 

Lemmen D, Warren F, Lacroix J, 2008.  From impacts to adaptation: Canada in a changing climate2007. Government of Canada, Ottawa,ON p. 1-20

Li, M., Shao, Q., Zhang, L., Chiew, F., 2010. A new regionalization approach and its application to predict flow Duration Curves in ungauged basins. J. Hydrol. 389, 135-145.

Liu, Q., Cui, B., 2011. ``Impacts of climate change/variability on the streamflow in the Yellow River Basin, China''. Ecological modelling 222 (2):268-274.

Lu, H., Li, S.S., Guo, J., 2013. Modeling Monthly Fluctuations in Submersion Area of a Dammed River Reservoir: A Case Study. J. Am. Water Resour. Ass., 49(1), 90-102.

Lyons, M., Lubitz, D., 2013.  Archimedes screws for mircro-hydropower generation. Engineering and Technology Conference. Proceedings of the ASME 2013 7th International Conference on Energy Sustainability and 11th Fuel Cell Science. 

Lafadani, E.K., Niab, A.M., Ahmadi, A., 2013. Daily suspended sediment load prediction using artificial neural networks and support vector machines. J. Hydrol. 478, 50-62. 

Liu, Y., Yang, W., Yu, Z., Lung, I., Yarotski, J., Elliot, J., Tiessen, K., 2014. Assessing effects of small dams on stream flow and water quality in an agricultural watershed. J. Hydrol. Eng., 19(10), 05014015-1 - 05014015-14.

Li, R., Chen, Q., Tonina, D., Cai, D.,  2015. Effects of upstream reservoir regulation on the hydrological regime and fish habitats of the Lijiang River, China. Ecol. Eng., 76, 75-83.

Liu, T., Huang, H. Q., Shao, M., Yao, W., Gu, J., Yu, G., 2015. Responses of streamflow and sediment load to climate change and human activity in the Upper Yellow River, China: a case of the Ten Great Gullies Basin. Water Sci. Tech., 71(12), 1893-1900.

Liu, Y., Yang, W., Yu, Z., Lung, I., Gharabaghi, B. 2015. ``Estimating Sediment Yield from Upland and Channel Erosion at a Watershed Scale Using SWAT''.   Water Resources Management 29 (5): 1399-1412.

Li, J., Li, G., Zhou, S., Chen, F.,2016. ``Quantifying the Effects of Land Surface Change on Annual Runoff Considering Precipitation Variability by SWAT''. Water Resources Management 30 (3): 1071-1084.

Magilligan, F.J., Nislow,K.H., 2001.Hydrologic alteration in a changing landscape: effects of impoundment in the Upper Connecticut River Basin, USA. J. Am. Water Ass. 37, 1551-1569.

Merz, R., Bloschl, G. 2004.  Regionalisation of catchment model parameters.  J. Hydrol., 287 (1-4), 95-123.

Magilligan, F.J., Nislow, K.H., 2005. Changes in hydrologic regime by dams. Geomorphology 71,, 61-78.

Metcalfe R.A., Chang C., Smakhtin V., 2005. Tools to support the implementation of environmentally sustainable flow regimes at Ontario’s waterpower facilities. Can. Water Resour. J., 30 (2), 97-110.

Moyle P.B., Mount, J.F., 2007.  Homogenous rivers, homogenous faunas. Proc. Nat. Acad. Sci. U.S.A., 104, 5711-5712.

Mohamoud, Y.M., 2008. Prediction of daily flow Duration Curves and streamflow for ungauged catchments using regional flow Duration Curves. Hydrol. Sci. J., 53 (4), 706-724.

Makkearson, A. Chang, N.B. Zhou, X. (2008). Short term streamflow forecasting with gobal climate change implications- A comparative study between genetic programming and neural network models. J. of Hydrol., 352, 336-354

Mohamoud, Y. M., 2008. Prediction of daily flow duration curves and streamflow for ungauged catchments using regional flow duration curves. Hydrol. Sci. J., 53(4), 706-724.

Mishra A.K, Ozger M., Singh, V.P., 2009. An entropy-based investigation into the variability of precipitation. J. Hydrol., 370 (1-4), 139–154.

Matteau, M., Assani, A.A., Mesfioui, M., 2009. Application of multivariate statistical analysis methods to the dam hydrologic impact studies. J. Hydrol., 371, 120-128.

Mishra, A.K., Ozger, M., Singh, V.P., 2009. An entropy-based investigation into the variability of precipitation. J. Hydrol. 370 (1–4), 139-154.

Mishra, A.K., Coulibaly, P., 2010. Hydrometric network evaluation for Canadian watersheds. J. Hydrol. 380 (3-4), 420-437.

Mishra, A.K., Singh, V.P., 2010. ``Changes in extreme precipitation in Texa''s. Journal of Geophysical Research 115  (D14), D14106.

Mishra, I., Uhlenbrook, S., Maskey, S., Ahmad, M.D., 2010. Regionalization of a conceptual rainfall–runoff model based on similarity of the flow duration curve: a case study from the semi-arid Karkheh basin, Iran. J. Hydrol. 391, 188-201.

Masih, I., Uhlenbrook, S., Maskey, S., and Ahmad, M.D., 2010. Regionalization of a conceptual rainfall–runoff model based on similarity of the flow duration curve: A case study from the semi-arid Karkheh basin, Iran. J. Hydrol., 391 (1-2), 188-201

Maier, H., Jain, A., Dandy, G., Sudheer, K., 2010. Methods used for the development of neural networks for the prediction of water resource variables in river systems: Current status and future directions. Env. Model. Soft., 25 (8), 891-909.

Moramarco, T., Singh, V.P. 2010. Formulation of the entropy parameter based on hydraulic and geometric characteristics of river cross sections. J. Hydrol. Eng. 15 (10), 852-858. 

Maier, H., Jain, A., Dandy, G., Sudheer, K., 2010. Methods used for the development of neural networks for the prediction of water resource variables in river systems: current status and future directions. Environ. Model. Softw. 25, 891-909. 

Mount, N.J., Abrahart, R.J., 2011. Load or concentration, logged or unlogged? Addressing ten years of uncertainty in neural network suspended sediment prediction. Hydrol. Process. 25, 3144-3157. 

Melesse, A.M., Ahmad, S.M., McClain, E., et al., 2011. Suspended sediment load prediction of river systems: an artificial neural network approach. Agric. Water Manage. 98 (5), 855-866. 

Mustafa, M., Isa, M.H., Rezaur, R.B., 2011. Comparison of artificial neural network for prediction of suspended sediment discharge in river – a case study in Malaysia.World Acad. Sci. Eng. Technol. 81, 372-376. 

McManamay, R.A., Orth, D.J., Dolloff, C.A., Frimpong, E.A., 2011. A regional classification of unregulated streamflows: spatial resolution and hierarchical frameworks. River Res. Appl., 28(7), 1019-1033.

Meijer KS, Van der Krogt WNM, Van Beek, E.,  2012.  A new approach to incorporating environmental flow requirements in water allocation modeling. Wat Resour Mgmt 26(5):1271-1286

Moore, J.N., Arrigoni, A.S., Wilcox, A.C., 2012. Impacts of dams on flow regimes in three headwater subbasins of the Columbia River Basin, United States. J. Am. Water Res. Ass., 48(5), 925-938.

Mondal, S., Jana, S., Majumder , S.  Roy, D., 2012. A comparative study for prediction of direct runoff for a river basin using geomorphological approach and artificial neural networks. Appl. Water Sci., 2 (1), 1-13. 

Mcmanamay, R., Orth, D., \& Dolloff, C., 2012. Revisiting the homogenization of dammed rivers in the southeastern US. J. Hydrol., 424-425, 217-237.

Mendicino, G., Senatore, A., 2013. Evaluation of parametric and statistical approaches for the regionalization of flow duration curves in intermittent regimes. J. Hydrol., 480, 19-32.

MNR, 2013. Ontario Flow Assessment Tool version III manual. Ministry of Natural Resources. 

Muller, M. F., Dralle, D. N., Thompson S. E., 2014. Analytical model for flow duration curves in seasonally dry climates. Water Resour. Res., 50 (7), 5510-5531.

Mohamoud, Y.M., 2014. Time series separation and reconstruction technique to estimate daily suspended sediment concentrations. J. Hydrol. Eng. 19 (2), 328-338.

Mittal, N., Bhave, A.G., Mishra, A., Singh, R., 2015. Impact of Human Intervention and Climate Change on Natural Flow Regime.Water Res. Manage. 30(2), 685-699.

Masinde, M., 2014. ``Artificial neural networks models for predicting effective drought index: Factoring effects of rainfall variability''. Mitigation and Adaptation Strategies for Global Change 19(8): 1139-1162.

Mattar, M. A., Alazba, A. A., Alblewi, B., Gharabaghi, B., and Yassin, M. A., 2016. ``Evaluating and Calibrating Reference Evapotranspiration Models Using Water Balance under Hyper-Arid Environment''. Water Resou. Mngmt, 1-23. 

Mao, T., Wang, G., Zhang, T.	, 2016. Impacts of Climatic Change on Hydrological Regime in the Three-River Headwaters Region, China, 1960-2009. Water Reso. Mngmt,	30 (1):115-131.

Nash, J.E., Sutcliffe, J.V., 1970. River flow forecasting through conceptual models part I—a discussion of principles. J. Hydrol. 10 (3), 282-290.

Newcombe, R.G., 1998. Two sided confidence intervals for the single proportion: comparison of seven methods. Stat. Med., 17 (8), 857-872

Nislow, K.H., Magilligan, F.J., Fassnacht, H., Bechtel, D., Ruesink, A., 2002. Effects of hydrologic alteration on flood regime of natural floodplain communities in the upper Connecticut River. J.  Am.Water Resour. Ass. 38, 1533-1548.

Newcombe, C.P., 2003. Impact assessment model for clear water fishes exposed to excessively cloudy water. J. Am. Water Resour. Assoc. 39 (3), 529-544. 

Neff, BP., Day, S.M., Piggot, A.R., Fuller, L.M., 2005. Baseflow in the Great Lakes Basin. National Water Research Institute, Environment Canada, Burlington, Ontario, Canada 

Nilsson, C., Reidy, C.A., Dynesius, M., Revenga, C., 2005. Fragmentation and flow regulation of the World’s large river systems. Science, 308, 405-408.

Nourani, V., Alami, M.T., Aminfar, M.H., 2009. A combined neural-wavelet model for prediction of Ligvanchai watershed precipitation. Eng. Appl. Artif. Intell. 22, 466-472. 

Nalley, D. Adamowski, J., Khalil, B., 2012. ``Using discrete wavelet transforms to analyse trends in streamflow and precipitation in Quebec and Ontario (1954–2008)''. Journal of Hydrology 475: 204-228.

Najafzadeh, M,, Sattar, AM., 2015. Neuro-Fuzzy GMDH Approach to Predict Longitudinal Dispersion in Water Networks. Water Resour. Manag., 29(7),  2205-2219.

Norman, L.M., Brinkerhoff, F., Gwilliam, E., Guertin, D.P., Callegary, J., Goodrich, D.C., Nagler, P.L., Gray, F., 2016. Hydrologic response of streams restored with check dams in the Chiricahua Mountains, Arizona. River Res. Appl., 32(4), 519-527.

NeuralTools Version 6 by Palisade Corporation (http://www.palisade.com/neuraltools/).

Oreskes, N., Shrader-Frechette, K., Belitz, K., 1994.  Verification, Validation, and Confirmation of Numerical Models in the Earth Sciences. Science, 263 (5147): 641-646.

Ongley, E.D., 1996. Control of Water Pollution from Agriculture – Irrigation and Drainage Paper 55. Food and Agricultural Organization (FAO), Rome, Italy. 

Onema, J.M.K., Mazvimavi, D., Love, D., Mul, M.L., 2006.  Effects of selected dams on river flows of Insiza River, Zimbabwe. Phys. Chem. Earth, 3, 870-875.

Oudin, L., Andréassian, V., Perrin, C., Michel, C., and Le Moine, N., 2008. Spatial proximity, physical similarity, regression and ungauged catchments: A comparison of regionalization approaches based on 913 French catchments. Water Resour. Res., 44 (3), W03413. 

Ontario Flow Assessment tools III, 2013. Ministry of Natural resources and Forestry. Available at \\http://www.giscoeapp.lrc.gov.on.ca/web/mnr/wrip/ofat/Viewer/viewer.html. Accessed June 2014.

Poff, N.L., Allan, J.D., Bain, M.B., Karr, J.R., Prestegaard, K.L., Richter, B.D., Sparks, R.E., Stromberg, J.C., 1997. The natural flow regime. BioScience 47 (11), 769-784.

Peters, D., Prowse, T., 2001. Regulation effects on the lower Peace river, Canada. Hydrol. Process., 15, 3181-3194.

Piggott, A.R. Moin, S. Southam, C., 2005. A revised approach to UKIH method for calculation of baseflow. Hydrol. Sci. J., 50 (5), 911-920

Poff, N.L., Bledsoe,B.P.,Cuhaciyan,C.O., 2006.Hydrologic variation with land use across the continguous United States: geomorphic and ecological consequences for stream ecosystem. Geomorph., 79,264-285

Poff, N.L., Olden, J.D., Merritt, D.M., Pepin, D.M., 2007. Homogenization of regional river dynamics by dams and global biodiversity implications. Proc. Natl. Acad. Sci. U.S.A 104, 5732-5737.

Piggott, A.R. and D.R. Sharpe., 2007. Geological interpretations of baseflow for southern Ontario 394-401. In Proceedings of the 60th Canadian Geotechnical and 8th Joint IAH-CNC and CGS Groundwater Specialty Conferences. Canadian Geotechnical Society and Canadian National Chapter of the International Association of Hydrogeologists. 

Pyron, M., Neumann, K. 2008. Hydrologic alterations in the Wabash River watershed, USA. River Res. Appl., 24, 1175-1184.

Perera, N., Gharabaghi, B., and Noehammer, P., 2009.  ``Stream Chloride Monitoring Program of City of Toronto: Implications of Road Salt Application''.   Water Quality Research Journal of Canada 44 (2: 132-140.

Perera, N., Gharabaghi, B., Noehammer, P., and Kilgour, B., 2010.  ``Road Salt Application in Highland Creek Watershed, Toronto, Ontario''.   Water Quality Research Journal of Canada, 45 (4): 451-461.

Poff, N.L., Richter, B.D., Arthington, A.H., Bunn, S.E., Naiman, R.J., Kendy, E., Acreman, M., Apse, C., Bledsoe, B.P., et al., 2010. The ecological limits of hydrologic alteration (ELOHA): a new framework for developing regional environmental flow standards. Freshw. Biol. 55, 147-170.

Poff , N.L., Zimmerman, J.K., 2010. Ecological responses to altered flow regimes: a literature review to inform the science and management of environmental flows. Freshw Biol 55:194-205.

Pandey, A., Prasad, R., Srivastava, J., Pandey S., 2012.  Retrieval of Soil Moisture by artificial neural networks using X-band ground based data. Russ. Agri. Sci., 38 (3): 230-233.

Piotrowski, A.P., Napiorkowski, J.J., 2011. Optimizing neural networks for river flow forecasting – evolutionary computation methods versus the Levenberg– Marquardt approach. J. Hydrol., 407 (1-4), 12-27.

Piotrowski, A.P., Napiorkowski, J.J., 2013. A comparison of methods to avoid overfitting in neural networks training in the case of catchment runoff modeling. J. Hydrol. 476, 97-111.

Perera, N., Gharabaghi, B., and Howard, K., 2013.  ``Groundwater Chloride Response in the Highland Creek Watershed Due to Road Salt Application: A Re-assessment after 20 Years''.   Journal of Hydrology 479: 159-168.

Pino, D., Ruiz-Bellet, J.L., Balasch, J.C., Romero-León, L., Tuset, J., Barriendos, M., Mazon, J., Castelltort, X.	2016. Meteorological and hydrological analysis of major floods in NE Iberian Peninsula.  J. Hydr.

Pumo, D., Caracciolo, D., Viola, F., Noto, L.V.2016. Climate change effects on the hydrological regime of small non-perennial river basins. Sci. Total Env., 	542: 76-92

Palla, A., Gnecco, I., la Barbera, P., Ivaldi, M., Caviglia, D.	 2016. An Integrated GIS Approach to Assess the Mini Hydropower Potential. Water Resou. Mangmt.,	1-18

Piqué, G., Batalla, R.J., Sabater, S., 2016. Hydrological characterization of dammed rivers in the NW Mediterranean region. Hydrol. Proc., 30(11):1691-1707

Renard, K.G., Foster, G.R., Weesies, G.A., McCool, D.K., Yoder, D.C., (coordinators) 1997. Predicting Soil Erosion by Water: A Guide to Conservation Planning With the Revised Universal Soil Loss Equation (RUSLE). Agriculture Handbook, vol. 703, U.S. Department of Agriculture, 404 pp.

Richter, B.D., Matthews, R., Harrison, D.L., Wigington, R., 2003. Ecologically sustainable water management: managing river flows for ecological integrity. Ecol. Appl., 13,  206-224.

Rojanamon, P., Chaisomphob, T., Rattanapitikon, W., 2007.  Regional flow duration model for the Salawin river basin of Thailand. ScienceAsia, 33 (4), 411-9. 

Ries, K. G., III, Guthrie, J. G., Rea, A. H., Steeves, P. A., and Stewart, D. W., 2008. StreamStats: A water resources web application, U.S. Geological Survey Fact Sheet 2008-3067, http://pubs.usgs.gov/fs/2008/3067/.

Rajaee, T., Mirbagheri, S.A., Nourani, V., Alikhani, A., 2010. Prediction of daily suspended sediment load using wavelet and neuro-fuzzy combined model. Int. J. Environ. Sci. Technol. 7 (1), 93-110. 

Rajaee, T., Nourani, V., Zounemat-Kermani, M., Kisi, O., 2011. River suspended sediment load prediction: application of ANN and wavelet conjunction model. J. Hydrol. Eng. 16, 613–627.  

Rajaee, T., 2011. Wavelet and ANN combination model for prediction of daily suspended sediment load in rivers. Sci. Total Environ. 409, 2917-2928. 

Roman, D.C., Vogel, R.M., Schwarz, G.E., 2012. Regional regression models of watershed suspended-sediment discharge for the Eastern United States. J. Hydrol. 472, 53–62.

Razan, J. I., Islam, R.S., Hasan, R., Hasan, S., and Islam, F., 2012. A comprehensive study of micro-hydropower plant and its potential in Bangladesh. Renew. Energy, 2012 (635396). 

Razavi, T., Coulibaly, P., 2013. Streamflow Prediction in ungauged basins: Review of Regionalization Methods. J. Hydrol. Eng., 18 (8), 958-975.

Rianna, M., Efstratiadis, A., Russo, F., Napolitano, F., and Koutsoyiannis, D., 2013. A Stochastic Index Method for Calculating Annual Flow Duration Curves in Intermittent Rivers. Irrig.  Drain., 62 (S2), 41-49.

Ren, S., Kingsford, R. T. , 2014. Modelling impacts of regulation on flows to the Lowbridgee floodplain of the Murrumbidgee River, Australia. J. Hydrol., 519, 1660-1667.

Razavian, A., Azizpour, H., Sullivan,J., Carlsson, S., 2014. CNN Features off-the-Shelf: An astounding baseline for recognition. In Proceedings of the IEEE Conference on Computer Vision and Pattern Recognition Workshops, 806-813. 

Rudra, R., Dickinson, W.T., Ahmed, S. I., Patel, P., Zhou, J., and Gharabaghi, B., 2015.  ``Changes in Rainfall Extremes in Ontario''.   International Journal of Environmental Research 9 (4): 1117-1372.

Razavi, T., Coulibaly, P., 2016: An evaluation of regionalization and watershed classification schemes for continuous daily streamflow prediction in ungauged watersheds, Canadian Water Resources Journal, DOI: 10.1080/07011784.2016.1184590

Searcy, J.K., 1959. Flow Duration Curves. Manual of Hydrology: Part 2. Low-Flow Techniques. Methods and Practices of the Geological Survey. Geological Survey Water-Supply Paper 1542-A.

Specht, D.F., 1991. A general regression neural network. IEEE Trans. Neural Netw., 2 (6), 568-576.

Singh, V. P., 1997. The use of  entropy in hydrology and water resources. Hydrol. Process., 11 (6), 587-626.

Sene, K.J., Farquharsin,F.A.K.1998. Sampling Errors for Water Resources Design: The Need for Improved Hydrometry in Developing Countries. Water Resour. Manag. 12: 121-138.

Syvitski, J.P., Morehead, M.D., Bahr, D.B., Mulder, T., 2000. Estimating fluvial sediment transport: the rating parameters. Water Resour. Res. 36 (9), 2747-2760.

Schreider, S. Y., Jakeman, A. J., Letcher, R. A., Nathan, R. J., Neal, B. P., and Beavis, S. G., 2002. Detecting changes in streamflow response to changes in non-climatic catchment conditions: Farm dam development in the Murray-Darling basin, Australia. J. Hydrol., 262(1-4), 84-98.

Sivaplan, M., Takeuchi, K., Franks, S.W., Gupta, V.K., Karambiri, H., et al., 2003. IAHS decade on predictions in ungauged basins (PUB), 2003–2012: shaping an exciting future for the hydrological sciences. Hydrol. Sci. J., 48(6), 857-880.

Srivastav, R. K., Sudheer, K. P., Chaubey I. A., 2007. Simplified approach to quantifying predictive and parametric uncertainty in artificial neural network hydrologic models. Water Resour. Res., 43 (10), W10407.

Santhi, C. Allen, P.M., Muttiah, R.S., Arnold, J.G., Tuppad, P., 2007. Regional estimation of baseflow for conterminous United States by hydrology landscape regions. J. Hydrol., 351 (1-2), 139-153.

Skoien, J.O., Bloschl, G., 2007. Spatiotemporal topological krigging of runoff time series. Water Resour. Res., 43 (9), W09419. 

Sadeghi, S.H.R., Mizuyama, T., Miyata, S., Gomi, T., Kosugi, K., Fukushima, T., Mizugaki, S., Onda, Y., 2007. Development, evaluation and interpretation of sediment rating curves for a Japanese small mountainous reforested watershed. Geoderma 144 (1-2), 198-211.

Schwartz, J., Dahle, M., Robinson, B., 2008. Concentration-duration-frequency curves for stream turbidity: possibilities for assessing biological impairment. J. Am. Water Resour. Assoc. 44 (4), 879–886Solomatine, D.  Shrestha, D., 2009. A novel method to estimate model uncertainty using machine learning techniques. Water Resour. Res., 45 (12), W00B11.

Schuol, J.,  Abbaspour,K.C.,  Yang, H.,  Srinivasan, R.,  Zehnder, A.J. 2008.  Modeling blue and green water availability in Africa.Water Resour. Res., 44 

Swain, J.B., JHa, R.,Patra, K.c. 2015. .Stream Flow Prediction in a Typical Ungauged Catchment Using GIUH Approach. Aquatic Procedia 4: 993-1000.

Saliha, A., Awulachew, S. Cullmann, J. and Hans-B., 2011.  Estimation of flow in ungauged catchments by coupling a hydrological model and neural networks: case study. Hydrol. Res., 42(5), 386-400.

Samuel, J., Coulibaly, P., Metcalfe, R.A., 2011. Estimation of continuous streamflow in Ontario ungauged basins: Comparison of regionalization methods. J. Hydrol. Eng., 16 (5), 447-459.

Singh, A., Imtiyaz, M., Isaac, R.K., Denis, D.M., 2011. Comparison of soil and water assessment tool (SWAT) and multilayer perceptron (MLP) artificial neural network for predicting sediment yield in the Nagwa agricultural watershed in Jharkhand, India. Agric. Water Manage. 104, 113-120.

Suen,J.P., 2011. Determining the ecological flow regime for existing reservoir operation. Wat. Resour. Manag. 25(3): 817-835.

Shu, C., Ouarda, T., 2012. Improved methods for daily streamflow estimates at ungauged sites. Water Resour. Res., 48 (2), WO2523.

Sang,Y.F., 2013. ``A review on the applications of wavelet transform in hydrology time series analysis''. Atmospheric Research 122: 8-15.

Sabouri, F., Gharabaghi, B., Mahboubi, A., McBean, E., 2013.  Impervious surfaces and sewer pipe effects on stormwater runoff temperature.  J. Hydrol., 502, 10-17. 

Singh V.P., 2013. ``Entropy Theory and Its Applications in Environmental and Water Engineering''; John Wiley: New York

Singh V.P., Byrd, A., Cui, H., 2014. ``Flow Duration Curve Using Entropy Theory''. Journal of Hydrological Engineering 19(7): 1340-1348

Sattar, AM., 2014a. Gene Expression models for the prediction of longitudinal dispersion coefficients in transitional and turbulent pipe flow. J. Pipeline Syst. Eng. Pract., ASCE  5 (1), 04013011.

Sattar, AM., 2014b. Gene Expression Models for Prediction of Dam Breach Parameters. J. Hydroin., 16(3), 550-571.

Srivastava, N., Hinton, G., Krizhevsky, A., Sutskever, I., Salakhutdinov, S., 2014. Dropout: a simple way to prevent neural networks from overfitting. J. Mach. Learn. Res., 15 (1): 1929-1958.

Sutskever, I., Vinyals, O.,V. Le.Q., 2014. Sequence to Sequence Learning with Neural Networks. In Advances in Neural Information Processing Systems 27, edited by Z. Ghahramani, M. Welling, C. Cortes, N. D. Lawrence, and K. Q. Weinberger, 3104-3112. Curran Associates, Inc.

Sattar, AM., Gharabaghi, B., 2015. Gene expression models for prediction of longitudinal dispersion coefficient in streams. J. Hydrol., 524, 587-596.

Szegedy, C., Wei L., Jia, W.,  Sermanet, P., Reed, S.,  Anguelov, D., et al., 2015. Going Deeper with Convolutions. In Proceedings of the IEEE Conference on Computer Vision and Pattern Recognition, 1-9.

Sattar, A., Gharabaghi, B., McBean, E., 2016. Predicting timing of watermain failure using gene expression models for infrastructure planning. Water Resour. Manag. DOI 10.1007/s11269-016-1241-x (in press).

Sattar, AM., 2016. Prediction of organic micropollutant removal in soil aquifer treatment system using GEP. J. Hydrol. Eng. DOI 10.1061/(ASCE)HE.1943-5584.0001372 (in press).

Sellami, H., Benabdallah, S., La Jeunesse, I., Vanclooster, M. 2016	. Climate models and hydrological parameter uncertainties in climate change impacts on monthly runoff and daily flow duration curve of a Mediterranean catchmentHydrological Sciences Journal, 1-15

Shi, H., Li, T., Wang, K., Zhang, A., Wang, G., Fu, X., 2016. Physically based simulation of the streamflow decrease caused by sediment-trapping dams in the middle Yellow River. Hydrol. Process., 30(5), 783-794.

Tayfur, G., 2002. Artificial neural networks for sheet sediment transport. Hydrol. Sci. J. 47 (6), 879-892.

Tayfur, G., Ozdemir, S., Singh, V.P., 2003. Fuzzy logic algorithm for runoff-induced sediment transport from bare soil surfaces. Adv. Water Resour. 26 (12), 1249-1256 

Tropsha, A., Gramatica, P., Gombar, V.K.2003. The importance of being earnest: validation is the absolute essential for successful application and interpretation of QSPR models. QSAR Comb. Sci., 22 (1), 69-77

Talebizadeh, M. Ayyoubzadeh, S.  Ghasemzadeh, M., 2010. Uncertainty analysis in sediment load modeling using ANN and SWAT model. Water Resour. Manag., 24 (9), 1747-1761.

Trenouth, W., Gharabaghi, B., Bradford, A., \& MacMillan, G. 2013.  Better management of construction sites to protect inland waters.   Inland Waters, 3 (2), 167-178.

Torgo, Luís, Paula Branco, Rita P. Ribeiro, and Bernhard Pfahringer. 2015. ``Resampling Strategies for Regression.'' Expert Systems 32 (3): 465-76.

Trenouth, W., Gharabaghi, B., 2015a.  Event-Based Soil Loss Models for Construction Sites. Journal of Hydrology 524, 780-788. 

Trenouth, W., Gharabaghi, B., 2015b.  Event-Based Design Tool for Construction Site Erosion and Sediment Cntrols.   Journal of Hydrology 528, 790-795.

The Theano Development Team, Al-Rfou, R., Alain, G.,  Almahairi, A.,  Angermueller, C.,  Bahdanau, D., Ballas, N., et al. 2016. Theano: A Python Framework for Fast Computation of Mathematical Expressions. arXiv [cs.SC]. arXiv. http://arxiv.org/abs/1605.02688.

Thompson, J., Sattar, A., Gharabaghi, B., and Warner, R., 2016.  ``Event-Based Total Suspended Sediment Particle Size Distribution Model''.  Journal of Hydrology, 536: 236-246.

USACE(US Army Corps of Engineers), 2016. Corps Map. National Inventory of Dams. $<$http://nid.usace.army.mil/cm\_apex/f?p=838:1:0::NO$>$ (accessed January, 2016).

USGS NWIS (United States Geological Survey National Water information System), 2016.,http://waterdata.usgs.gov/nwis$>$ (accessed January, 2016).

Vogel, R.M, Fennessey, N.M., 1994. Flow duration curves I: New interpretation and confidence intervals. J. Water Res. Plan. Manag., 120 (4), 485-504.

Vose, D., 1996. Quantitative Risk Analysis: A Guide to Monte Carlo simulation Modeling, John Wiley, New York, 317.

Vicente, S., Sergio M. Saz Sánchez, M. A and Cuadrat, J.M., 2003. Comparative analysis of interpolation methods in the middle Ebro Valley (Spain): Application to annual precipitation and temperature. Clim. Res., 24 (2), 161-180.

Vanoni, V., 2006. Sedimentation Engineering: Classic Edition, pp. i–xii, http ://dx.doi. org/10.1061/9780784408230.fm.

Vasiljevic, B., McBean, E., Gharabaghi, B., 2012. ``Trends in Rainfall Intensity for Stormwater Designs in Ontario''.   Journal of Water and Climate Change 3 (1): 1-10.

Walker, H. 1931.Studies in the History of the Statistical Method. Williams and Wilkins Co., Baltimore, MD, pp.24-25

Warnick, C.C., 1984. Hydropower Engineering. Prentice-Hall, Inc., Englewood Cliffs, New Jersey, pp. 59-73.

Wall, G.J., Coote, D.R., Pringle, E.A., Shelton, I.J. (Eds.), 2002. RUSLEFAC - Revised Universal Soil Loss Equation for Application in Canada: A Handbook for Estimating Soil Loss from Water Erosion in Canada. Research Branch, Agriculture and Agri-Food Canada, Ottawa. Contribution No. AAFC/AAC2244E, 117pp.

Westra S., Brown, C., Lall, U., and Sharma, A., 2007.  Modeling multivariable hydrological series: Principal component analysis or independent component analysis? Water Resour. Res. 43 (6), W06429.

Wei, X.H., Zhang, M.F., 2010. Quantifying streamflow change caused by forest disturbance at a large spatial scale: a single watershed study. Water Resour. Res., 46 (12), W12525.

Wu, C.L., Chau, K.W., 2011. Rainfall–runoff modeling using artificial neural network coupled with singular spectrum analysis. J. Hydrol., 399 (3-4): 394-409.

Wen, L., Rogers, K., Ling, J., Saintilan, N., 2011.The impacts of river regulation and water diversion on the hydrological drought characteristics in the Lower Murrumbidgee River, Australia. J. Hydrol., 405, 382-391.

Yang,D., Ye,B., Kane, D.L., 2004. Streamflow changes over Siberian Yenisei River basin. J.Hydrol., 296, 59-80.

Yu, Z.G., Leung, Y.,  Chen, Y.D.,Zhang, Q., Anh, V., Zhou, Y., 2014.''Multifractal analyses of daily rainfall time series in Pearl River basin of China''. Physica A: Statistical Mechanics and its Applications 405: 193-202.

Yoon, Y., Beighley, E., 2015. Simulating streamflow on regulated rivers using characteristic reservoir storage patterns derived from synthetic remote sensing data. Hydrol. Process., 29, 2014-2026.

Zhang, X., Vincent, L.A., Hogg, W.D., Niitsoo, A., 2000. ``Temperature and precipitation trends in Canada during the 20th century''. Atmosphere-Ocean 38, 395-429.

Zhang, X., Harvey, K.D., Hogg, W.D., Yuzyk, T., 2001. ``Trends in Canadian streamflow''. Water Resources Research 37, 987-998.

Zhang, X., Srinivasan, R., 2009. GIS-based spatial precipitation estimation: A comparison of geostatistical approaches. J. Am. Water Resour. Ass., 45 (4), 894-906.

Zhao, F., Zhang, L., Xu, Z., Scott, D.F., 2010. Evaluation of methods for estimating the effects of vegetation change and climate variability on streamflow. Water Resour. Res. 46 (3), W03505.

Zimmerman, J.K.H., Letcher, B.H., Nislow, K.H., Lutz, K.A., Magilligan, F.J., 2010. Determining the effects of dams on subdaily variation in river flows at a whole-basin scale. River Res. Appl., 26, 1246-60. 

Zhang,Q.,  Zhou,Y., Singh, V., P., Chen, X.,2012. The influence of dam and lakes on the Yangtze River streamflow: long-range correlation and complexity analyses. Hydrol. Process. , 26, 436-444.

Zhang, W., Wei, X., Zheng, J., Zhu, Y., Zhang, Y., 2012. Estimating suspended sediment loads in the Pearl River Delta region using sediment rating curves. Cont. Shelf Res. 38 (15), 35-46.

Zhao, G., Tian, P., Mu, X., Jiao, J., Wang, F., Gao, P., 2014. Quantifying the impact of climate variability and human activities on streamflow in the middle reaches of the Yellow River basin, China. J. Hydrol., 519, 387-398.

Zhang, L.-T., Li, Z.-B., Wang, H., Xiao, J.-B., 2016. Influence of intra-event-based flood regime on sediment flow behavior from a typical agro-catchment of the Chinese Loess Plateau. J. Hydr.,538:71-81.

\end{hangingpar}


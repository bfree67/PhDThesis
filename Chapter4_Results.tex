\chapter{Analysis of Results and Validation}
This chapter presents results generated using the methodologies in Chapter 3 and compares them to other results or expected results in order to measure their effectiveness in addressing the problem statements of Chapter 2.

\section{Initial air zone mapping}

\subsection{Air dispersion model results and histograms}

Virtual sources were placed at key areas to represent different risk areas within Kuwait (see Fig \ref{fig:Kuwait}).  The selected sites include industrial and urban areas as well as oil production and remote desert areas. These sites are distributed among the main industrial and living areas to reflect the main land uses in Kuwait- mainly industrial, residential, oil production and open desert.  The sites are listed in Table \ref{tb:virtualsites} and shown in Figure \ref{fig:virtual-locations}. A partial 10 km coastal buffer is also shown in Figure \ref{fig:virtual-locations} that incorporates two major roads running north/south (40 Highway) and east/west (6th Ring Road).
%
\begin{table}[H]
\centering
\caption{Virtual Site locations.}
\label{tb:virtualsites}
\resizebox{\columnwidth}{!}{%
\begin{tabular}{@{}lccl@{}}
\toprule
\textbf{Location}        & \textbf{Longitude} & \textbf{Latitude} & \multicolumn{1}{c}{\textbf{Comments}} \\ \midrule
Ahmadhi                  & 48.0915$^{o}$      & 28.9933$^{o}$     & Dense heavy industrial area           \\
Burgan Field             & 47.9644$^{o}$      & 28.9879$^{o}$     & Oil production operations             \\
Egalia                   & 48.1032$^{o}$      & 29.1393$^{o}$     & Dense residential area                \\
Jahra City               & 47.6671$^{o}$      & 29.3321$^{o}$     & Dense urban area                      \\
Jahra Inland             & 47.7378$^{o}$      & 29.2418$^{o}$     & Sparse population                     \\
Kuwait City              & 47.9874$^{o}$      & 29.3580$^{o}$     & Dense urban area                      \\
Kabd                     & 47.7366$^{o}$      & 29.1967$^{o}$     & Sparse population                     \\
Kuwait Intl Airport      & 47.971$^{o}$       & 29.2440$^{o}$     & Sparse population                     \\
Salmiya                  & 48.069$^{o}$       & 29.3180$^{o}$     & Dense residential area                \\
Shuaibah Industrial Zone & 48.1428$^{o}$      & 28.9944$^{o}$     & Dense heavy industrial area           \\
Subhan                   & 48.0064$^{o}$      & 29.2113$^{o}$     & Dense heavy industrial area           \\ \bottomrule
\end{tabular}
} %end resize
\end{table}
%
%
\begin{figure}[H]
\includegraphics[width=\linewidth,keepaspectratio]{images/aqz4.png} 
\caption[Location of virtual sources]{Location of virtual sources for air dispersion analysis shown with partial 10 km buffer and main perimeter roads.}
\label{fig:virtual-locations}
\end{figure}
%

Figure \ref{fig:virtual-dispersions} displays examples of individual ground level concentration outputs from the virtual SO$_{2}$ sources described in Table \ref{tb:puffmodel}, while Figure \ref{fig:virtual-histogram} shows histograms of the SO$_{2}$ concentrations calculated by the model to show the distribution of concentrations within the model area.  Concentration color gradients were plotted on the same scale for comparison purposes.  The model is based only on runs accomplished with SO$_{2}$ over an annual average of 8,760 hours.  
%
\begin{figure}[H]
\includegraphics[width=\linewidth,keepaspectratio]{images/aqz5.png} 
\caption[Ground level air dispersion patterns]{Ground level air dispersion patterns from Inland and Coastal virtual SO$_{2}$ sources averaged over 8,760 hours (annual); (a) Jahra Inland dispersion (Inland) - Year 2010; (b) Kuwait City dispersion (Coastal) - Year 2009}
\label{fig:virtual-dispersions}
\end{figure}
%
%
\begin{figure}[H]
\includegraphics[width=\linewidth,keepaspectratio]{images/aqz6.png} 
\caption[Histograms of ground level concentrations]{Histograms of ground level concentrations from Inland and Coastal virtual SO$_{2}$ sources averaged over 8,760 hours (annual); (a) Jahra Inland histogram (Inland) - Year 2010; (b) Kuwait City histogram (Coastal) - Year 2009.}
\label{fig:virtual-histogram}
\end{figure}
%

Different studies using ambient air measurements over time assumed normal distributions of measurements and included statistics such as the mean and standard deviation \citep{Henschel2013, Rivera2015,Suresh2005}.  For highly skewed datasets, traditional statistics cannot effectively describe the distribution.  As described in the introduction section, skewness and kurtosis best describe the different concentration values generated by the puff highly skewed dispersion model runs.  

Zones were identified visually at first in order to establish baseline results.  The calculated statistics for each site and annual run are shown in Table 3 along with distance to shore.  The runs properly classified as inland $(S < 3, K < 20)$ or coastal $(S > 3, K > 20)$ are classified as strong.   Most of the misclassified runs using the $S-K$ method are sites that are clearly situated in an inland or coastal area, but are not properly classified (a site in an inland area classified as coastal and vice versa).  The Shuaibah Industrial Zone - Year 2010, is located 2.1 km from the coast and is considered coastal due to its physical location and that 2 out of 3 trials classified the zone as coastal. Figure \ref{fig:SKplotKuwait} plots the Skewness vs. Kurtosis values from Table \ref{tb:skewdist}. The Subhan site is 9.3 km from the coast and near the border of a coastal zone and a production zone. The Kabd site is clearly inland (16.8 km from the coast), but shows many coastal wind pattern features – suggesting that LSB effect extends deep into this region of Kuwait due to the flat topography and relatively smooth surfaces existing as a result of limited development.
%
\begin{table}[H]
\centering
\caption{Skewness, Kurtosis and distance to shore statistics for virtual sites.}
\label{tb:skewdist}
\resizebox{\columnwidth}{!}{%
\begin{tabular}{@{}lccccc@{}}
\toprule
\textbf{Site} & \textbf{Zone} & \textbf{Year} & \textbf{Skewness} & \textbf{Kurtosis} & \textbf{Distance (km)} \\ \midrule
Ahmadhi & Coastal & 2009 & 4.18 & 30.98 & 7.1 \\
 &  & 2010 & 4 & 25.43 &  \\
 &  & 2011 & 4.37 & 30.13 &  \\
Burgan Field & Inland & 2009 & 1.84 & 8.32 & 19.2 \\
 &  & 2010 & 2.34 & 11.11 &  \\
 &  & 2011 & 2.74 & 13.47 &  \\
Egalia & Coastal & 2009 & 8.16 & 115.4 & 2.7 \\
 &  & 2010 & 7.33 & 94.55 &  \\
 &  & 2011 & 6.74 & 77.61 &  \\
Jahra City & Coastal & 2009 & 5.07 & 48.94 & 4.9 \\
 &  & 2010 & 4.38 & 37.47 &  \\
 &  & 2011 & 6.46 & 70.95 &  \\
Jahra Inland & Inland & 2009 & 1.74 & 6.52 & 12.7 \\
 &  & 2010 & 2.15 & 9.26 &  \\
 &  & 2011 & 2.29 & 9.31 &  \\
Kabd & Inland & 2009 & 2.89 & 12.96 & 16.8 \\
 &  & 2010 & 3.19 & 16.33 &  \\
 &  & 2011 & 3.55 & 18.85 &  \\
Kuwait City & Coastal & 2009 & 4.71 & 32.72 & 2.2 \\
 &  & 2010 & 4.45 & 30.18 &  \\
 &  & 2011 & 5.22 & 41.33 &  \\
Kuwait Intl Airport & Inland & 2009 & 1.88 & 9.98 & 11.9 \\
 &  & 2010 & 1.56 & 6.56 &  \\
 &  & 2011 & 2.77 & 16.23 &  \\
Salmiya & Coastal & 2009 & 5.19 & 40.12 & 2.2 \\
 &  & 2010 & 5.76 & 48.53 &  \\
 &  & 2011 & 9.17 & 123.69 &  \\
Shuaibah Industrial Zone & Coastal & 2009 & 3.08 & 22.59 & 2.1 \\
 &  & 2010 & 2.33 & 12.56 &  \\
 &  & 2011 & 3.29 & 22.87 &  \\
Subhan & Coastal & 2009 & 2.8 & 12.06 & 9.3 \\
 &  & 2010 & 3.02 & 13.07 &  \\
 &  & 2011 & 3.68 & 26.72 &  \\ \bottomrule
\end{tabular}
} %end resize
\end{table}
%
%
\begin{figure}[H]
\includegraphics[width=\linewidth,keepaspectratio]{images/aqz7.png} 
\caption{$S-K$ plot of virtual sources in Kuwait.}
\label{fig:SKplotKuwait}
\end{figure}
%

Regions are highlighted showing inland classification areas (orange) and coastal (blue). A linear trend line of the plotted points, $K$ = 11.901$S$ – 15.732, shows strong correlation ($R^{2}$ = 0.943).

To confirm the procedure, random points were modeled in the nearby State of Qatar. An inland point was taken at Abu Naklah Air Base (ANAB - Lat 25.1208$^{o}$ East/Long 51.3177$^{o}$ North) and a coastal point taken at Masaeed Industrial City (MIC - Lat 24.9911$^{o}$ East/Long 51.5506$^{o}$ North) for the same years (2009-2011) using the same virtual source parameters as shown in Table \ref{fig:virtual-locations}. The points are approximately 570 km from Kuwait International Airport as shown in Figure \ref{fig:qatarlocs}. 

%
\begin{figure}[H]
\includegraphics[width=\linewidth,keepaspectratio]{images/aqz8.png} 
\caption{Location of Qatar virtual sites in relation to Kuwait.}
\label{fig:qatarlocs}
\end{figure}
%

The methodology properly classified each location as inland and coastal.  The $S-K$ points are shown in Figure \ref{fig:SKplotQatar} along with a trend line using the parameters from Figure \ref{fig:SKplotKuwait}.

%
\begin{figure}[H]
\includegraphics[width=\linewidth,keepaspectratio]{images/aqz9.png} 
\caption{$S-K$ plot of virtual sources in Qatar.}
\label{fig:SKplotQatar}
\end{figure}
%

Figure \ref{fig:flowchart} shows a flowchart which summarizes the process to develop this coastal/inland determination.  The process requires at least 3 trial years to allow a majority winner-take-all classification if one year returns a misclassification.

%
\begin{figure}[H]
\includegraphics[width=\linewidth,keepaspectratio]{images/aqz10.png} 
\caption{Coastal/Inland determination process flow chart.}
\label{fig:flowchart}
\end{figure}
%

\subsection{Identifying Air Quality Zones for Kuwait}

 The delineation and categorization of the AQZs for Kuwait was completed to assign appropriate control technologies to local air emission processes in order to improve air quality in zones that did not meet air quality standards and prevent deterioration in zones that did \citep{Carr2012}. This approach assumes that areas with predominantly coastal effect winds have different exposure risks than areas with predominantly inland winds, and that they should therefore be managed differently.  In Kuwait, most of the population, industries, power generation, and downstream hydrocarbon processes are located within 10 km of the coast.  The Bay of Kuwait is a relatively shallow body of water that never exceeds a depth of 20 meters.  Seasonal surface water temperatures vary from 12.4 degrees Celsius to 34.8 degrees Celsius \citep{Al-Mutairi2014}.  The presence of this body of water provides a significant source of energy that impacts local weather patterns \citep{Mizak2007,Panin2005}. 

After coastal effect limits were estimated using the $S-K$ method, zones were further defined by incorporating existing roads and expressways.  Where practical, these physical features were taken into account to allow for easier management.  As a result, the Coastal / Inland boundaries follow a main Kuwaiti ring road up to the northeast end of Kuwait Bay.  Municipal district boundaries were also used to promote familiarity when transitional areas were close to estimated boundaries.  The final AQZs are shown in Figure \ref{fig:kuwaitzones}.

%
\begin{figure}[H]
\includegraphics[width=\linewidth,keepaspectratio]{images/aqz11.png} 
\caption{Kuwait Air Quality Zones.}
\label{fig:kuwaitzones}
\end{figure}
%

The pink area in Figure \ref{fig:kuwaitzones} shows the buffer area from Figure \ref{fig:virtual-locations} and the dark border show final zone borders. Statistics for the air quality zones are shown in Table \ref{tb:aqzstats}.

\begin{sidewaystable}  %table rotated 90 degrees
\begin{table}[H]
\centering
\caption{Air Quality Zone Statistics.}
\label{tb:aqzstats}
\resizebox{\columnwidth}{!}{%
\begin{tabular}{@{}clccccc@{}}
\toprule
\textbf{Item} & \textbf{Zone} & \textbf{Zone Type} & \textbf{Area (km$^{2}$)} & \textbf{\% of Total Land} & \textbf{Population} & \textbf{\% of Total Population} \\ \midrule
1 & Northern Coastal & Coastal & 365 & 2.10\% & 392,598 & 11.50\% \\
2 & Southern Coastal & Coastal & 796 & 4.60\% & 737,997 & 21.60\% \\
3 & Central Coastal & Coastal & 426 & 2.50\% & 2,280,773 & 66.70\% \\
4 & Bubiyan & Coastal & 847 & 4.90\% & NA &  \\
5 & Southern Inland & Inland & 2,700 & 15.60\% & NA &  \\
6 & Jahra Inland & Inland & 10,556 & 61.00\% & 9,205 & 0.30\% \\
7 & Wafra & Production & 269 & 1.60\% & NA &  \\
8 & West Kuwait & Production & 375 & 2.20\% & NA &  \\
9 & North Kuwait 1 & Production & 267 & 1.50\% & NA &  \\
10 & North Kuwait 2 & Production & 105 & 0.60\% & NA &  \\
11 & Burgan & Production & 592 & 3.40\% & NA &  \\
 & TOTAL &  & 17,298 & 100\% & 3,420,573 & 100\% \\ \bottomrule
\end{tabular}
} %end resize
\end{table}
\end{sidewaystable}

The Kuwaiti coastal air zones include 14.1\% of the land area and 99.8\% of the population.  The inland zones include 76.6\% of the land area and the production zones represent 9.3\%.  Descriptions of the individual zone types are provided below:

\textbf{Coastal Zones.} The four coastal zones also include Bubiyan Island and the smaller islands off the coast of Kuwait including Failaka and Kubar Islands.  While comprising only 9.2\% of the land area, coastal zones are home to 99.7\% of all people living and working in Kuwait. Over 88\% of people live in the Southern and Central zones alone.  With additional construction of residential areas and industrial facilities, these are the primary zones for air quality management. 

\textbf{Inland Zones.} The two inland zones are mostly farms, camping areas, military base and free range land comprising over 76\% of the land area in Kuwait. The area is sparsely populated (0.3\% of the population).  

\textbf{Production Zones.} Designated oil and gas fields operated by the national oil companies require different compliance activities due to their industrial processes and site security.  production zones were established to allow the oil industry to apply their own regulatory program under local regulatory guidance and supervision. The zones follow the fence lines of the secure areas and represent over 9.3\% of the land area with no residential population and a sparsely distributed work force. Additionally, larger production zones were found to have internal micro-climates compared to the neighboring inland zones during the modeling phase.  This is due in large part to over-grazing by herd animals outside the secured areas.  The resulting change in land cover clearly impacts albedo and surface heat capacity. Perimeters around production zones are fenced and provide a continuous boundary.

\subsection{Validation of Air Zone Mapping Methodology}

In addition to applying the method to a secondary area in Qatar, three different methodologies were used to test the validity of the model and the final AQZ selection. The first test was a comparison of prognostic weather data sets to determine if the air dispersion model component is robust. The second test used a support vector machine (SVM) to classify the data sets and compare to the less computationally intensive method we presented. The final test evaluated the regions identified as a result of the LSB classification against satellite based data.

\subsubsection{Prognostic weather data comparison between MM5 and WRF}

Another common option for prognostic weather data is the Weather research and forecasting (WRF) model also developed by NCAR \citep{Skamarock2008}. WRF has been used in the region for different climatology studies looking at wind forecasts for different applications \citep{Amjad2015}. Direct comparisons of the two models, MM5 and WRF show that the results are statistically similar, with topography playing a large role in predictive accuracy \citep{Gsella2014, Henmi2004}. Comparing ground level wind patterns generated by both the MM5 and WRF models over a common 100 x 100 km area using 4 km grid cells, we found that wind directions were very similar, with wind fields approximately 10\% higher for the MM5 results. When applied to the CALPUFF model, the lower WRF wind speeds provided higher ground level concentrations, however, the final results support our method using MM5. Wind roses for three locations are shown in Figure \ref{fig:12windcompare}.
 	 
%
\begin{figure}[H]
\includegraphics[width=\linewidth,keepaspectratio]{images/aqz12.png} 
\caption[Comparison of wind patterns using prognostic data]{Comparison of wind patterns using MM5 and WRF prognostic data.}
\label{fig:12windcompare}
\end{figure}
%

\subsubsection{Classification by SVM}

SVMs are a class of supervised learning based models that classify input data sets by mapping into high-dimensional feature spaces. While SVMs have been used since 1995 \citep{Cortes1995}, they have only relatively recently been applied to air pollution concentration predictions \citep{Lu2005, Luna2014, Moazami2016}. A full description of the SVM is beyond the scope of this paper but several excellent references and online tutorials are available. A linear SVM was trained with the data set in Table \ref{tb:skewdist} using the Classification Learner module in Matlab 2015a (www.mathworks.com) with a linear kernel function \citep{Yang2011}. The input data was standardized by the software prior to training. Inland classification was converted to a -1 while coastal classification was designated with a 1. The SVM was initially trained only with $S$ and $K$ results. The resulting confusion matrix could achieve an overall accuracy of 87.9\% - comparable to the results of graphing the $S$ and $K$ values.  

%
\begin{figure}[H]
\includegraphics[width=\linewidth,keepaspectratio]{images/aqz13.png} 
\caption{Confusion matrix of trained SVM with $S$ and $K$ inputs.}
\label{fig:13confusionSK}
\end{figure}
%

The use of the SVM allows additional features to be included so that multiple inputs can be used to improve accuracy. Adding the coastal distance improved overall accuracy to 97\% as shown in Figure \ref{fig:14confusionSKdist}. The trained SVM was tested using the Qatar data set and achieved 100\% accuracy. 

%
\begin{figure}[H]
\includegraphics[width=\linewidth,keepaspectratio]{images/aqz14.png} 
\caption{Confusion matrix of trained SVM with S, K and distance to coast inputs.}
\label{fig:14confusionSKdist}
\end{figure}
%
 
\subsubsection{Satellite validation of mixing areas}

In order to evaluate the effectiveness of the new zone boundaries, satellite derived data from NASA’s Ozone Monitoring Instrument (OMI) on the Aura satellite was obtained and mapped for NO$_{2}$ over Longitudes 46.125$^{o}$ – 49.125$^{o}$ East and Latitudes 28.375$^{o}$ – 30.375$^{o}$ North \citep{Boersma2011, Strawa2013}.  NO$_{2}$ was used to map actual concentrations due to the higher concentrations measured in Kuwait by the OMI sensor instead of SO$_{2}$.  This provided higher contrasts of readings for visual analysis. Ground monitoring stations have also confirmed that SO$_{2}$ concentrations are not significant while concentrations of NO$_{2}$ are \citep{Al-Awadhi2014}.

Data was downloaded from the Giovanni web based system \citep{Acker2007} and converted to Excel spreadsheets.  The data resolution from the OMI was limited to 0.25 degrees per cell, representing approximately 312 km$^{2}$.  In addition, the data represented the average measurement of a vertical column extending from the surface of the planet to the satellite sensor in a sun synchronous orbit approximately 705 km above sea level.  The satellite repeats its orbit every 16 days (OMI, 2012). 

Annual mean and standard deviations for NO$_{2}$ (measured in 1015 molecule/cm$^{2}$) for the Kuwait area were plotted as shown in Figure \ref{fig:15meanNO2} and Figure \ref{fig:16stdNO2} respectively.  Highest concentrations are located on the Central and Southern coasts corresponding to heavy urban development and industrial activities. The pattern roughly follows the central and southern coastal zones.

%
\begin{figure}[H]
\includegraphics[width=\linewidth,keepaspectratio]{images/aqz15.png} 
\caption{Annual mean concentration of NO$_{2}$ in Kuwait for 2011.}
\label{fig:15meanNO2}
\end{figure}
%
%
\begin{figure}[H]
\includegraphics[width=\linewidth,keepaspectratio]{images/aqz16.png} 
\caption[Annual Standard Deviation of NO$_{2}$ Concentration]{Annual Standard Deviation of NO$_{2}$ Concentration as percentage of annual concentration in Kuwait for 2011.}
\label{fig:16stdNO2}
\end{figure}
%

Annual average concentrations of NO$_{2}$ show that the highest concentrations are clustered on the coast. Other zones may contribute to the ambient concentration but exposure risks are predominately located in the Central and Southern Coastal zone areas.  Looking at the standard deviation of the annual average NO$_{2}$ concentrations shows that different areas vary at different rates throughout the year. Areas of high variance occur in the Central Coast and Southern Inland zones while the Southern Coastal zone stays relatively constant.  A zone with high ambient concentrations that does not fluctuate has higher exposure risk. 

\subsection{Air Quality Exposure Risk Scores}

To identify where high exposure risk areas are, a risk scoring method was developed that incorporated annual average concentration ($\mu$) and annual concentration variance ($\sigma^{2}$).  High values of annual concentrations and annual variance were defined as annual concentrations greater than one standard deviation from the mean ($> \mu + \sigma$) or approximately 1.64 times the mean. This value accounts for the highest 5\% of values. Low values were assumed to be less than one standard deviation from the mean ($< \mu - \sigma$) or approximately 0.836 times the mean.

The following assumptions of risk were used:
\begin{itemize}
\item Areas with high annual concentrations and low annual variation had the most exposure risk due to constant high concentration exposure throughout the year.
\item Areas with high annual average concentrations and high annual variations were also considered high-exposure risks.
\item Areas with low annual exposure concentrations and high annual variation had moderate exposure risk due to fluctuating concentration levels throughout the year.
\item Areas with low annual exposure concentrations and low annual variation had low risk.
\end{itemize}

Datasets used for the annual concentration and annual standard deviations were normalized by taking the average of each cell as shown in eq \ref{eq:normconcentrate}

\begin{equation}
\label{eq:normconcentrate}
CN_{i,j}=\frac{C_{i,j}}{\bar{C}}
\end{equation}

\noindent
where $CN_{i,j}$ is the normalized annual concentration value of the cell, $C_{i,j}$ is the annual concentration value for the cell, and $\bar{C}$ is the average concentration of all cells.  A similar procedure was done to normalize the standard deviation of each cell.  Risk values ranged from 0 (low) or 1 (high) for concentration and 0 (low) to 0.5 (high) for standard deviation as shown in eq 5. 

\begin{equation}
\label{eq:riskscore}
Risk = CN_{risk} + StDev_{risk}
\end{equation}

\noindent
where $CN_{risk}$ is given as

\begin{equation}
\label{eq:riskCN1}
CN_{risk} = \left\{\begin{matrix}
1, CN > 1.6\\ 
0, CN \leq 1.6
\end{matrix}\right.
\end{equation}

\noindent
and $StDev_{risk}$ is given as
\begin{equation}
\label{eq:riskCN2}
StDev_{risk} = \left\{\begin{matrix}
0.5, StDev > 16\%\\ 
0, StDev \leq 16\%
\end{matrix}\right.
\end{equation}

Risk scores are shown in Table \ref{tb:expriskscores} and Figure 17.  High and medium risk areas are predominantly in the coastal areas with some medium risk areas extending inland where heavy oil production takes place.

\begin{table}[H]
\centering
\caption{Exposure Risk Scores.}
\label{tb:expriskscores}
\begin{tabular}{@{}cc@{}}
\toprule
\textbf{Score} & \textbf{Risk} \\ \midrule
1.5 & High          \\
1   & Medium - High \\
0.5  & Medium        \\
0  & Low           \\ \bottomrule
\end{tabular}
\end{table}

%
\begin{figure}
%\includegraphics[width=\linewidth,height=22.1cm,keepaspectratio]{images/aqz17.png} 
\includegraphics[width=\textwidth,keepaspectratio]{images/aqz17.png} 
\caption[NO$_{2}$ Exposure risk areas]{NO$_{2}$ Exposure risk based on annual average concentration and standard deviation. Areas of high population are shown as blue dots.}
\label{fig:176riskNO2}
\end{figure}
%

The use of satellite data confirms the high exposure risk areas fall within identified AQZs.  These areas should have priority in regards to access of pollution control technology and exposure reduction programs.  Zones that do not have high exposure risks can be managed differently such as providing additional control technology for processes in zones near high exposure risk areas to prevent cross-zone transport of emissions. 
%%%%%%%%%%%%%%%%%%%%%%%%%%%%%%%%%%%%%%%%%%%%%%%%%%%%%%%%%%%%%%%%%%%%%%%%%%%%%%%%%%%%%%%%%%%%%%%%%%%%%%
%%%%%%%%%%%%%%%%% END OF SECTION%%%%%%%%%%%%%%%%%%%%%%%%%%%%%%%%%%%%%%%%%%%%%%%%%%%%%%%%%%%%%%%%%%%%%%
%%%%%%%%%%%%%%%%%%%%%%%%%%%%%%%%%%%%%%%%%%%%%%%%%%%%%%%%%%%%%%%%%%%%%%%%%%%%%%%%%%%%%%%%%%%%%%%%%%%%%%
\clearpage
\section{Zone classification}

\subsection{Single station case}
Models were prepared to compute the CDI values for 1 hour NO$_{2}$ and 8 hour O$_{3}$ from an air monitoring station located in a coastal urban area and used data collected between 2008 and 2010.  The particular station was located in a dense residential area on a government building approximately 300 m away from a major highway. The resulting PDFs estimated from the data sets and related air quality standards are shown in Table \ref{tb5:singlePDF}.
%
\begin{table}[H]
\centering
\caption{Representative PDFs for single station case}
\label{tb5:singlePDF}
\begin{tabular}{@{}cccc@{}}
\toprule
\textbf{Pollutant} & \textbf{Air Quality Standard}  & \textbf{Compliance} & \textbf{Exceedance} \\ 
 &($\mu g/m^{3}$) &\textbf{PDF distribution} & \textbf{PDF distribution} \\ \midrule
1 hr NO$_{2}$ & 200 & Gamma & Pareto \\
8 hr O$_{3}$ & 100 & Kumaraswamy & Pareto \\ \bottomrule
\end{tabular}
\end{table}
%
The PDF means and standard deviations (SD) were then used as input parameters for Normal distributions after applying sample modifications per eq \ref{eq5:cdfmu}.  A summary of the inputs (means and standard deviations) for each model are shown in Table \ref{tb6:singleInputs}.
%
\begin{table}[H]
\centering
\caption{Summary of Normal Distribution input parameters used to calculate CDFs for the single station case.} 
\label{tb6:singleInputs}
\begin{tabular}{@{}lccccc@{}}
\toprule
\textbf{} & \textbf{} & \multicolumn{2}{c}{\textbf{Compliance PDF}} & \multicolumn{2}{c}{\textbf{Exceedance PDF}} \\ \midrule
\textbf{Pollutant} & \textbf{Method} & \textbf{Mean} & \textbf{SD} & \textbf{Mean} & \textbf{SD} \\
1 hr NO$_{2}$ & 3 Strike & 1,838,596.79 & 5,991.89 & 2,427.80 & 299.6 \\
 & 99\% & 1,820,820.40 & 5,962.85 & 70,945.80 & 1,619.50 \\
8 hr O$_{3}$ & 3 Strike & 849,524.12 & 3,750.43 & 1,106.10 & 86.7 \\
 & 99\% & 841,310.53 & 3,732.25 & 32,323.80 & 468.5 \\ \bottomrule
 &  & \multicolumn{4}{c}{(in $\mu g-hr/m^{3}$)} \\ 
\end{tabular}
\end{table}

%
The CDIs computed through the MCA of each pollutant using samples drawn from Compliance and Exceedance distributions in Table \ref{tb6:singleInputs} are shown in Table \ref{tb7:singleCDI}.
%
\begin{table}[H]
\centering
\caption[Single station CDIs]{Single station CDIs (in $\mu g/kgBW-day$)}
\label{tb7:singleCDI}
\begin{tabular}{@{}lccc@{}}
\toprule
\textbf{Pollutant} & \textbf{Method} & \textbf{Mean} & \textbf{SD} \\ \midrule
1 hr NO$_{2}$ & 3 Strike & 16.04 & 0.05 \\
 & 99\% & 16.48 & 0.05 \\
 & Difference & 2.7\% &  \\
8 hr O$_{3}$ & 3 Strike & 7.41 & 0.03 \\
 & 99\% & 7.61 & 0.03 \\
 & Difference & 2.7\% &  \\ \bottomrule
\end{tabular}
\end{table} 
%
Statistical testing of the single station case for 1 hr NO$_{2}$ is shown in Table \ref{tb8:test1hrNO2}.
%
\begin{table}[H]
\centering
\caption{Statistical testing for single station CDI case for 1 hr NO$_{2}$}
\label{tb8:test1hrNO2}
\begin{tabular}{lcc}
\toprule
\textbf{Method} & \textbf{3 Strikes} & \textbf{99\%} \\ \midrule
COV & 0.003 & 0.003 \\
F-test & \multicolumn{2}{c}{1.059} \\
df & 26,279 & 26,279 \\
F$_{critical}$ ($p<0.05$) & \multicolumn{2}{c}{1.02} \\
Result ($F_{test} < F_{critical}$) & \multicolumn{2}{c}{FALSE} \\
t-test & \multicolumn{2}{c}{955.73} \\
dfs & \multicolumn{2}{c}{52,514} \\
t$_{critical}$ ($p<0.05$) & \multicolumn{2}{c}{1.664} \\
Result (t$_{test} < $t$_{criticial}$) & \multicolumn{2}{c}{FALSE} \\ \bottomrule
\end{tabular}
\end{table}
%
For the 1 hr NO$_{2}$, the COVs of both methods are much less than 1 indicating the samples most likely came from normal distributions. As predicted in the Methods section, the large number of samples caused the variance test to fail. Using a modified t-test with unequal variances and a computed degrees of freedom approximation, the test for means also fails, requiring us to reject the null hypothesis and accept that there are significant statistical differences between the two method for 1 hr NO$_{2}$ classification. Critical values for the F and t tests were calculated using the F-distribution and t-distribution quantile functions.

Statistical testing results for the single station case of 8 hr O$_{3}$ is shown in Table \ref{tb9:test1hrO3}.
%
\begin{table}[H]
\centering
\caption{Statistical testing for single station CDI case for 8 hr O$_{3}$}
\label{tb9:test1hrO3}
\begin{tabular}{lcc}
\toprule
\textbf{Method} & \textbf{3 Strikes} & \textbf{99\%} \\ \midrule
COV & 0.004 & 0.004 \\
F-test & \multicolumn{2}{c}{1.005} \\
df & 26,279 & 26,279 \\
F$_{critical}$ ($p<0.05$) & \multicolumn{2}{c}{1.02} \\
Result ($F_{test} < F_{critical}$) & \multicolumn{2}{c}{TRUE} \\
t-test & \multicolumn{2}{c}{701.7} \\
dfs & \multicolumn{2}{c}{52,560} \\
t$_{critical}$ ($p<0.05$) & \multicolumn{2}{c}{1.664} \\
Result (t$_{test} < $t$_{criticial}$) & \multicolumn{2}{c}{FALSE} \\ \bottomrule
\end{tabular} \\
\end{table}
%
As with the 1 hr NO$_{2}$, the 8 hr O$_{3}$ shows that the null hypothesis must be rejected, despite the passing of the variance test and being 2.7\% different. For single stations, the 3 Strike method offers less exposure over the same duration than the 99\% Rule method. A more graphical representation of the two CDIs is shown in Fig \ref{fig6:CDIdistributions} where the different CDI distributions for 8 hr O$_{3}$ are plotted together, showing the difference in means. This assumes that BW and IR are constant.
%  
\begin{figure}
%\includegraphics[width=\linewidth,height=22.1cm,keepaspectratio]{images/risk6.png}
\includegraphics[width=\textwidth,height=\textheight,keepaspectratio]{images/risk6.png}  
\caption{Comparison of CDI distributions for 8 hr O$_{3}$.}
\label{fig6:CDIdistributions}
\end{figure}
%

\subsection{Multiple station cases}
\subsubsection{Results for Multiple station case: 1 hr NO$_{2}$}

Three stations within a coastal zone of Kuwait were selected for comparison with three years of data from 2008-2010. In addition to the station used for the single case, two stations located in nearby residential areas were also used. In each case the stations were mounted on the roofs of government buildings. The steps described above in eq \ref{eq11:step1} through \ref{eq14:step3b} were used to determine the weighted \%NC probability of continuous success, p, for the binomial distributions used to estimate additional non-compliant days. A summary of initial station parameters is shown in Table \ref{tb10:multiParaNO2}, including number of compliant and NC hours, the binomial probability and the base PDF distribution forms used for the MCA.
% 
\begin{table}[H]
\centering
\caption{Parameters for multiple station case of 1 hr NO$_{2}$.}
\label{tb10:multiParaNO2}
\begin{tabular}{@{}lcccc@{}}
\toprule
\textbf{Parameters} & \textbf{AMS 1} & \textbf{AMS 2} & \textbf{AMS 3} & \textbf{Total} \\ \midrule
Compliant hrs & 25,780 & 26,230 & 25,684 & 77,694 \\
NC hrs & 524 & 74 & 620 & 1,218 \\
\%NC & 1.99\% & 0.28\% & 2.36\% & 4.63\% \\
p (from eq 12) & 43.0\% & 6.1\% & 50.9\% & 100\% \\
Compliant PDF distribution & Gamma & Weibull & Triangle &  \\
NC PDF distribution & Pareto & Kumaraswamy & Lognormal &  \\ \bottomrule
\end{tabular}
\end{table} 

The parameters of Table \ref{tb10:multiParaNO2} were used to define the normal distributions used in the MCA for each method as summarized in Table \ref{tb11:multiNO2}.
% 
\begin{table}[H]
\centering
\caption{Normal distribution inputs for multiple station case CDF’s of 1 hr NO$_{2}$.}
\label{tb11:multiNO2}
\begin{tabular}{@{}lccc@{}}
\toprule
 & \multicolumn{3}{c}{\textbf{3 Strike Method ($\mu g-hr/m^{3}$)}} \\ \midrule
 & AMS 1 & AMS 2 & AMS 3 \\
 & \multicolumn{3}{c}{Compliant Normal Distributions} \\
Mean & 1,838,597 & 1,001,779 & 1,751,400 \\
SD & 5,991.90 & 4,907.70 & 7640.7 \\
 & \multicolumn{3}{c}{NC Normal Distributions} \\
Mean & 2,427.80 & 2,024.86 & 2,447.45 \\
SD & 299.58 & 77.37 & 444.99 \\
 & \multicolumn{3}{c}{\textbf{99\% Rule Method ($\mu g-hr/m^{3}$)}} \\
 & AMS 1 & AMS 2 & AMS 3 \\
 & \multicolumn{3}{c}{Compliant Normal Distributions} \\
Mean & 1,802,162 & 974,065 & 1,705,873 \\
SD & 6,133.60 & 4,880.30 & 7,557.30 \\
 & \multicolumn{3}{c}{NC Normal Distributions} \\
Mean & 216,883.59 & 180,887.22 & 218,638.96 \\
SD & 2,831.51 & 731.23 & 4,205.88 \\ \bottomrule
\end{tabular}
\end{table}

The resulting descriptive statistics for the CDIs of each station after 10,000 iterations during the MCA are shown in Table \ref{tb12:CDIstatsNO2}. 
%
\begin{table}[H]
\centering
\caption{CDI statistics for multiple stations of 1 hr NO$_{2}$.}
\label{tb12:CDIstatsNO2}
\begin{tabular}{@{}cccc@{}}
\toprule
 & \textbf{AMS 1} & \textbf{AMS 2} & \textbf{AMS 3} \\ \midrule
 & \multicolumn{3}{c}{\textbf{Mean ($\mu g/kgBW-day$)}} \\
3 Strike Method & 16.03 & 8.73 & 15.27 \\
99\% Rule Method & 17.06 & 9.24 & 16.29 \\
Difference & 6.4\% & 5.8\% & 6.7\% \\
 & \multicolumn{3}{c}{\textbf{SD ($\mu g/kgBW-day$)}} \\
3 Strike Method & 0.05 & 0.04 & 0.07 \\
99\% Rule Method & 0.06 & 0.04 & 0.08 \\ \bottomrule
\end{tabular}
\end{table}

The statistical tests for the methods at each station are shown in Table \ref{tb13:stat-testsNO2} for 1 hr NO$_{2}$. As with the single station, the multiple stations pass the Normality and variance tests, but fails the mean tests. The Null hypothesis is thus rejected for the multiple stations case for 1 hr NO$_{2}$.
% 
\begin{table}[H]
\centering
\caption{Statistical tests for multiple stations of 1 hr NO$_{2}$.}
\label{tb13:stat-testsNO2}
\begin{tabular}{@{}rccc@{}}
\toprule
\multicolumn{1}{l}{} & \textbf{AMS 1} & \textbf{AMS 2} & \textbf{AMS 3} \\ \midrule
\multicolumn{1}{l}{Normality testing} &  &  &  \\
3 Strike Method COV & 0.003 & 0.005 & 0.004 \\
99\% Rule Method COV & 0.004 & 0.005 & 0.005 \\
\multicolumn{1}{l}{Variance testing} &  &  &  \\
$F-test$ & 0.75 & 0.947 & 0.769 \\
df & 26,279 & 26,279 & 26,279 \\
$Fcritical$ ($p<0.05$) & 1.02 & 1.02 & 1.02 \\
Result ($F_{test} < F_{critical}$) & TRUE & TRUE & TRUE \\
\multicolumn{1}{l}{Means testing} &  &  &  \\
$t-test$ & 3175 & 1919.3 & 2484.4 \\
df & 52,560 & 52,560 & 52,560 \\
$t_{critical}$ ($p<0.05$) & 1.644 & 1.644 & 1.644 \\
\multicolumn{1}{l}{Result $(t_{test} < t_{critical})$} & FALSE & FALSE & FALSE \\ \bottomrule
\end{tabular}
\end{table}

\subsubsection{Results for Multiple station case: 8 hr O$_{3}$}

The multiple station case for 8 hr O$_{3}$ follows the same procedure as the 1 hr NO$_{2}$. Table \ref{tb14:multiParamO3} shows the input parameters. Table \ref{tb15:normalinputs03} shows the Normal distribution inputs used for CDF in the MCA. Table \ref{tb16:CDI-O3} summarizes the input statistics of the computed CDIs and Table \ref{tb17:statmultiO3} shows the statistical test results.

% 
\begin{table}[H]
\centering
\caption{Parameters for multiple station case of 8 hr O$_{3}$}
\label{tb14:multiParamO3}
\resizebox{\columnwidth}{!}{%
\begin{tabular}{@{}rcccc@{}}
\toprule
\textbf{Parameters} & \textbf{AMS 1} & \textbf{AMS 2} & \textbf{AMS 3} & \textbf{Total} \\ \midrule
Compliant hrs & 25,048 & 24,653 & 25,151 & 74,852 \\
NC hrs & 699 & 566 & 72 & 1,337 \\
\%NC & 2.71\% & 2.24\% & 0.29\% & 5.24\% \\
p (from eq \ref{eq12:step2}) & 51.8\% & 42.8\% & 5.4\% &  \\
Compliant PDF distribution & Kumaraswamy & Kumaraswamy & Kumaraswamy &  \\
Exceedance PDF distribution & Pareto & Exponential & Pareto &  \\ \bottomrule
\end{tabular}
}% end resize
\end{table}
%
\begin{table}[H]
\centering
\caption{Normal distribution inputs for multiple station case CDF’s of 8 hr O$_{3}$.}
\label{tb15:normalinputs03}
\begin{tabular}{@{}rccc@{}}
\toprule
 & \multicolumn{3}{c}{\textbf{3 Strike Method ($\mu g-hr/m^{3}$)}} \\ \midrule
 & AMS 1 & AMS 2 & AMS 3 \\
 & \multicolumn{3}{c}{Compliant Normal Distributions} \\
Mean & 849,524.10 & 964,430.50 & 597,880.30 \\
SD & 3,750.40 & 3,529.20 & 2,465.70 \\
 & \multicolumn{3}{c}{NC Normal Distributions} \\
Mean & 844.1 & 775.1 & 483.5 \\
SD & 138 & 112.4 & 60.4 \\
 & \multicolumn{3}{c}{\textbf{99\% Rule Method ($\mu g-hr/m^{3}$)}} \\
 & AMS 1 & AMS 2 & AMS 3 \\
 & \multicolumn{3}{c}{Compliant Normal Distributions} \\
Mean & 825,548.10 & 988,490.08 & 722,209.16 \\
SD & 3,707.63 & 3,682.44 & 3,025.70 \\
 & \multicolumn{3}{c}{NC Normal Distributions} \\
Mean & 75,399.30 & 69,252.40 & 40,748.60 \\
SD & 1,299.10 & 1,058.30 & 368.3 \\ \bottomrule
\end{tabular}
\end{table}

% 
\begin{table}[H]
\centering
\caption{CDI statistics for multiple stations of 8 hr O$_{3}$.}
\label{tb16:CDI-O3}
\begin{tabular}{@{}lccc@{}}
\toprule
 & \textbf{AMS 1} & \textbf{AMS 2} & \textbf{AMS 3} \\ \midrule
 & \multicolumn{3}{c}{Mean ($\mu g/kgBW-day$)} \\
3 Strike Method & 7.4 & 8.4 & 5.21 \\
99\% Rule Method & 7.85 & 9.21 & 6.65 \\
Difference & 6.1\% & 9.6\% & 27.6\% \\
 & \multicolumn{3}{c}{SD ($\mu g/kgBW-day$)} \\
3 Strike Method & 0.033 & 0.031 & 0.021 \\
99\% Rule Method & 0.034 & 0.033 & 0.027 \\ \bottomrule
\end{tabular}
\end{table}

% 
\begin{table}[H]
\centering
\caption{Statistical tests for multiple stations of 8 hr O$_{3}$.}
\label{tb17:statmultiO3}
\begin{tabular}{@{}rccc@{}}
\toprule
 & \textbf{AMS 1} & \textbf{AMS 2} & \textbf{AMS 3} \\ \midrule
\textbf{Normality test} &  &  &  \\
3 Strike COV & 0.004 & 0.004 & 0.004 \\
99\% Rule COV & 0.004 & 0.004 & 0.004 \\
\textbf{Variance test} &  &  &  \\
$F-test$ & 0.905 & 0.844 & 0.655 \\
df & 26,279 & 26,279 & 26,279 \\
F$_{critical}$ ($p<0.05$) & 1.02 & 1.02 & 1.02 \\
Result ($F_{test} < F_{critical}$) & TRUE & TRUE & TRUE \\
\textbf{Means test} &  &  &  \\
$t-test$ & 1532.48 & 2900.33 & 6829.69 \\
df & 52,560 & 52,560 & 52,560 \\
$t_{critical}$ ($p<0.05$) & 1.664 & 1.664 & 1.664 \\
Result $(t_{test} < t_{critical})$ & FALSE & FALSE & FALSE \\ \bottomrule
\end{tabular}
\end{table}

The 8 hr O$_{3}$ passes the test for normality and variance, but fails the test for equal means. Again, the Null hypothesis is rejected, showing that there is a statistically significant difference between the CDI of the two methods.
%%%%%%%%%%%%%%%%%%%%%%%%%%%%%%%%%%%%%%%%%%%%%%%%%%%%%%%%%%%%%%%%%%%%%%%%%%%%%%%%%%%%%%%%%%%%%%%%%%%%%%
%%%%%%%%%%%%%%%%% END OF SECTION%%%%%%%%%%%%%%%%%%%%%%%%%%%%%%%%%%%%%%%%%%%%%%%%%%%%%%%%%%%%%%%%%%%%%%
%%%%%%%%%%%%%%%%%%%%%%%%%%%%%%%%%%%%%%%%%%%%%%%%%%%%%%%%%%%%%%%%%%%%%%%%%%%%%%%%%%%%%%%%%%%%%%%%%%%%%%
\clearpage

\section{Area source estimating}

Using the general form of annual emissions in Equation \ref{eq1}, the Emissions Factor PDFs from Table \ref{tb2:emissionfactors} were converted into Triangle distributions, and the parameters for total annual combustion in Equation 5 were used to form a Normal distribution to represent the total input feedstock.  An MCA was performed for 50,000 realizations using @RISK 7.5.1 with Latin Hypercube sampling (www.palisade.com) . The number of iterations was selected based on personal experience  in order to insure full coverage and is significantly more than the minimum amount needed to cover over 99\% of all possible scenarios.  If system resources and complexity required reduced iterations, methods exist to more precisely estimate the minimum number required for a desired confidence interval \citep{Bukaci2016}.

Results were divided by 106 to get units of metric tonnes, or 103 to get kilograms if the total was less than 1 tonne per year.  The average annual emissions for individual pollutants are shown in Table \ref{tb4:results} with Triangle distributions of the EFs and results shown in Fig \ref{figng3:pdfs}. 

%
\begin{table}[H]
\centering
\caption{MCA Results for Annual Emissions for $nargyla$ smoking in Kuwait.}
\label{tb4:results}
\begin{tabular}{@{}cccc@{}}
\toprule
 & \textbf{Pollutant} & \textbf{Annual Mean} & \textbf{Annual Std} \\ \midrule
\multirow{4}{*}{\textbf{tonnes}} & CO$_{2}$ & 1,352.88 & 53.32 \\
 & CO & 80.33 & 21.40 \\
 & NMOC & 5.14 & 0.62 \\
 & CH$_{4}$ & 3.77 & 0.30 \\ \midrule
\multirow{4}{*}{\textbf{kg}} & NOx & 744.30 & 176.64 \\
 & N$_{2}$O & 233.76 & 94.17 \\
 & NH$_{3}$ & 225.75 & 92.07 \\
 & PAH & 1.34 & 0.84 \\ \bottomrule 
\end{tabular}
\end{table}
% 
 
 	
%
\begin{figure}
%\includegraphics[width=\linewidth,height=22.1cm,keepaspectratio]{images/ng3.png}
\includegraphics[width=\textwidth,height=\textheight,keepaspectratio]{images/ng3.png}  
\caption[Results of Emission Factor PDFs and Annual Emissions]{Results of Emission Factor PDFs and Annual Emissions based on Triangle distributions of Emission Factors.}
\label{figng3:pdfs}
\end{figure}
%

\subsection{Sensitivity Analysis or Area Source Emissions}
While the annual mass consumption could assume a Normal distribution due to the Central Limit Theorem, the EFs were arbitrarily assumed to have Triangle distributions.  To evaluate the selection of distribution, different distributions were applied to the MCA and compared to the Triangle distribution results.  Additional distributions included the Uniform, Normal, and Pert. Of the four distributions compared, all used finite range parameters based on the Min, Max, and Expected values of Table \ref{tb2:emissionfactors}.  For the Normal distribution, the mean, $\mu$, was assumed to be the average of the Min and Max, while the standard deviation, $\sigma$, assumed that the Min/Max were 3$\sigma$ from the mean and was calculated as
%
\begin{equation}
\label{eq8}
\sigma = (\mu – Min) / 3
\end{equation}
%
The parameters used for the different EF distributions are shown in Table \ref{tb5:sensitivity}.
%
\begin{table}[H]
\centering
\caption{Sensitivity Analysis Parameters for EF Distributions g/kg}
\label{tb5:sensitivity}
\begin{tabular}{@{}lccccc@{}}
\toprule
\textbf{Pollutant} & \textbf{Min} & \textbf{Expected} & \textbf{Max} & \textbf{$\mu$} & \textbf{$\sigma$} \\ \midrule
CO$_{2}$ & 2155 & 2385 & 2567 & 2361 & 68.7 \\
CO & 35 & 189 & 198 & 116.5 & 27.2 \\
NMOC & 6 & 10 & 11 & 8.5 & 0.8 \\
CH$_{4}$ & 5.29 & 6.7 & 7.8 & 6.545 & 0.4 \\
NOx & 0.5 & 1.41 & 2 & 1.25 & 0.25 \\
N$_{2}$O & 0.118 & 0.24 & 0.87 & 0.494 & 0.13 \\
NH$_{3}$ & 0.0009 & 0.395 & 0.79 & 0.39545 & 0.13 \\
PAH & 0.0001 & 0.00045 & 0.0065 & 0.0033 & 0.001 \\ \bottomrule
\end{tabular}
\end{table}
%
Comparing the means of each distribution shows that changes in the average vary slightly among the different forms with most variation, represented by the sample standard deviation (s) being N$_{2}$O in Table \ref{tb6:compare}.  The Triangle and Pert distributions provide the most conservative average values assuming central tendencies apply to the distribution.
%
\begin{table}[]
\centering
\caption{Comparison of annual total emission means}
\label{tb6:compare}
\resizebox{\columnwidth}{!}{%
\begin{tabular}{@{}cccccccccc@{}}
\toprule
 & \multicolumn{5}{c}{\textbf{Distributions}} & \multicolumn{4}{c}{\textbf{Statistics}} \\ \midrule
\textbf{Units} & \textbf{Pollutant} & \textbf{Triangle} & \textbf{Uniform} & \textbf{Normal} & \textbf{Pert} & \textbf{$\bar\{x\}$} & \textbf{S} & \textbf{Lower 95\% CI} & \textbf{Upper 95\% CI} \\ \midrule
\multirow{4}{*}{\textbf{tonnes}} & CO$_{2}$ & 1352.9 & 1348.3 & 1348.3 & 1357.4 & 1351.7 & 4.37 & 1,343.2 & 1,360.3 \\
 & CO & 80.3 & 66.5 & 66.5 & 94.1 & 76.9 & 13.2 & 51.0 & 102.8 \\
 & NMOC & 5.1 & 4.9 & 4.9 & 5.4 & 5.1 & 0.3 & 4.5 & 5.6 \\
 & CH$_{4}$ & 3.8 & 3.7 & 3.7 & 3.8 & 3.8 & 0.03 & 3.7 & 3.8 \\ \midrule
\multirow{4}{*}{\textbf{kg}} & NOx & 744.3 & 713.8 & 713.8 & 774.8 & 736.7 & 29.2 & 679.5 & 793.8 \\
 & N$_{2}$O & 233.8 & 282.1 & 282.1 & 185.4 & 245.8 & 46.3 & 155.1 & 336.6 \\
 & NH$_{3}$ & 225.7 & 225.8 & 225.8 & 225.7 & 225.8 & 0.1 & 225.6 & 225.9 \\
 & PAH & 1.3 & 1.9 & 1.9 & 0.8 & 1.5 & 0.5 & 0.5 & 2.5 \\ \bottomrule
\end{tabular}
} %end resize
\end{table}
%
A similar review of the standard deviations of the different distributions shows that the NOx, N$_{2}$O, and NH$_{3}$ have large variances relative to their means as seen in Table \ref{tb7:stndev}.  In terms of distributions, the Uniform distribution offers the most variance with the Triangle distribution coming next. In cases like this where ranges of values have finite limits, using a Normal distribution or any of the t-distribution families is not recommended as possible MCA values could be out of physical range.
%
\begin{table}[]
\centering
\caption{Comparison of annual total emission standard deviations.}
\label{tb7:stndev}
\resizebox{\columnwidth}{!}{%
\begin{tabular}{@{}cccccccccc@{}}
\toprule
 & \multicolumn{5}{c}{\textbf{Distributions}} & \multicolumn{4}{c}{\textbf{Statistics}} \\ \midrule
\textbf{Units} & \textbf{Pollutant} & \textbf{Triangle} & \textbf{Uniform} & \textbf{Normal} & \textbf{Pert} & \textbf{$\bar{x}$} & \textbf{S} & \textbf{Lower 95\% CI} & \textbf{Upper 95\% CI} \\ \midrule
\multirow{4}{*}{\textbf{tonnes}} & CO$_{2}$ & 53.8 & 71.9 & 45.8 & 50.4 & 55.5 & 11.43 & 33.1 & 77.9 \\
 & CO & 21.4 & 26.9 & 15.6 & 14.3 & 19.5 & 5.8 & 8.1 & 30.9 \\
 & NMOC & 0.6 & 0.8 & 0.5 & 0.5 & 0.6 & 0.2 & 0.3 & 0.9 \\
 & CH$_{4}$ & 0.3 & 0.4 & 0.2 & 0.3 & 0.3 & 0.08 & 0.2 & 0.5 \\ \midrule
\multirow{4}{*}{\textbf{kg}} & NOx & 176.7 & 247.6 & 143.4 & 160.8 & 182.1 & 45.7 & 92.5 & 271.8 \\
 & N$_{2}$O & 94.2 & 124.1 & 71.6 & 72.6 & 90.6 & 24.7 & 42.3 & 138.9 \\
 & NH$_{3}$ & 92.1 & 130.2 & 75.5 & 85.3 & 95.8 & 23.9 & 48.9 & 142.7 \\
 & PAH & 0.840 & 1.055 & 0.572 & 0.556 & 0.8 & 0.2 & 0.3 & 1.2 \\ \bottomrule
\end{tabular}
} %end resize
\end{table}

While annual emission distributions are influenced by the form of the EF distribution, the influence is not significant. Ninety five percent (95\%) Confidence Intervals (CI) of the distribution means are shown in Tables \ref{tb6:compare}, with similar CI’s for the sample standard deviations in Table \ref{tb7:stndev}. In all case, the distribution statistics are within the bounds and therefore can accept a null hypothesis that distribution means are not significantly different and that distribution variances are not statistically significant ($p<0.05$).  Tables \ref{tb6:compare} and \ref{tb7:stndev} show that using a Triangular distribution provides a conservative value if the underlying distribution is not known and few data points are available to describe the sample population.  

\subsection{Comparison of Annual Results to National Emissions Inventory}
Applying the resulting means generated from these PDFs we can compare the estimated emissions from smoking $nargyla$ to the total emissions reported by Kuwait in its first national communication under the United Nations Framework Convention on Climate Change (UNFCCC) \citep{AlMudhhi2012}.  The report baselines Kuwait’s Greenhouse Gas (GHG) emissions for 1996 baseline year.  Table \ref{tb8:comparison} shows that the estimated emissions for smoking in 2016 are significant compared to the entire reported emissions of 1996. 

%
\begin{table}[]
\centering
\caption{Comparison of 2016 $nargyla$ emissions to 1996 total emissions in Gg}
\label{tb8:comparison}
\begin{tabular}{@{}cccc@{}}
\toprule
 & \textbf{Annual 2016} & \textbf{1996} & \textbf{Percent of} \\
\textbf{Pollutant} & \textbf{$nargyla$ emissions} & \textbf{total emissions} & \textbf{1996 emissions} \\ \midrule
CO$_{2}$ & 1,352.9 & 29,502 & 4.6\% \\
CO & 80.3 & 544 & 14.8\% \\
NMOC & 5.1 & 522 & 1.0\% \\
CH$_{4}$ & 3.8 & 129.19 & 2.9\% \\
NOx & 0.74 & 113 & 0.7\% \\
N$_{2}$O & 0.23 & 0.44 & 53.1\% \\ \bottomrule
\end{tabular}
\end{table}

This comparison may be unfair as the 1996 emissions were probably lower than emissions released in more recent years.  For one reason, the population in 1995 was reported as 1,575,570 (total Kuwaiti and non-Kuwaiti) while in 2016, the total population was reported as 4,132,415 \citep{CSB2017}.  If we assume that the 1996 $nargyla$ emissions would be a 38\% linear approximation of the 2016 emissions (the percentage of the 1996 population to the 2016 population – we can assume the same percentage of smokers in both populations), the adjusted comparison in Table \ref{tb9:adjusted} still shows that $nargyla$ smoking is large relative to other combustion processes, especially N$_{2}$O.
%
\begin{table}[]
\centering
\caption{Comparison of $nargyla$ emissions in Gg adjusted to 1996 population.}
\label{tb9:adjusted}
\begin{tabular}{@{}cccc@{}}
\toprule
 & \textbf{Annual 2016} & \textbf{1996} & \textbf{Percent of} \\
\textbf{Pollutant} & \textbf{$nargyla$ emissions} & \textbf{total emissions} & \textbf{1996 emissions} \\ \midrule
CO$_{2}$ & 514.1 & 29,502 & 1.7\% \\
CO & 30.5 & 544 & 5.6\% \\
NMOC & 2.0 & 522 & 0.4\% \\
CH$_{4}$ & 1.4 & 129.19 & 1.1\% \\
NOx & 0.28 & 113 & 0.3\% \\
N$_{2}$O & 0.09 & 0.44 & 20.2\% \\ \bottomrule
\end{tabular}
\end{table}
%%%%%%%%%%%%%%%%%%%%%%%%%%%%%%%%%%%%%%%%%%%%%%%%%%%%%%%%%%%%%%%%%%%%%%%%%%%%%%%%%%%%%%%%%%%%%%%%%%%%%%
%%%%%%%%%%%%%%%%% END OF SECTION%%%%%%%%%%%%%%%%%%%%%%%%%%%%%%%%%%%%%%%%%%%%%%%%%%%%%%%%%%%%%%%%%%%%%%
%%%%%%%%%%%%%%%%%%%%%%%%%%%%%%%%%%%%%%%%%%%%%%%%%%%%%%%%%%%%%%%%%%%%%%%%%%%%%%%%%%%%%%%%%%%%%%%%%%%%%%
\clearpage
\section{Traffic density estimations}

Data was collected at the west-bound intersection on the 7th Ring and 40 Highway known for heavy congestion during the periods of 0700 and 0800 on Wednesday, 20 December 2017 and the south-bound intersection of the 55 Airport Road and 4th Ring on Thursday, 4 January 2018 between 1300 and 1400 hrs as shown in \ref{fig:flights}. 

\begin{figure}[H]
\includegraphics[width=\linewidth,keepaspectratio]{images/flights.png} 
\caption[Flight locations]{Flight locations in north and south Kuwait.}
\label{fig:flights}
\end{figure}

Data collection began after traffic had stacked, usually 20-30 seconds into the red light period, and lasted for about 45 seconds. The UAS flew on the inside shoulder to prevent direct overflight of vehicles. Data collection consisted of two passes - the first pass collected images perpendicular to traffic, and the 2nd collected images at a 45 degree offset to the traffic to provide necessary angles for late processing as shown in Figure \ref{fig:mission}.

\begin{figure}[H]
\includegraphics[width=\linewidth,keepaspectratio]{images/mission.png} 
\caption{Mission profile for data collection.}
\label{fig:mission}
\end{figure}

The flights were processed using the Pix4Dmapper Pro software with a summary of  each mission shown in Table \ref{tb:flightdata}. A total of two flights were processed on 20 Dec mission and three flights on the 4 Jan mission. All flights were flown at 40 m AGL.

\begin{table}[H]
\centering
\caption{Summary of data collection flights and processed imagery}
\label{tb:flightdata}
\begin{tabular}{@{}cccc@{}}
\toprule
\textbf{Flight} & \textbf{Flight start} & \textbf{\# of pictures} & \textbf{\# of cloud points} \\ \midrule
20Dec17-1 & 0704 & 39 & 3,819,568 \\
20Dec17-2 & 0725 & 68 & 6,610,918 \\
4Jan18-1 & 1317 & 40 & 3,037,660 \\
4Jan18-2 & 1329 & 38 & 3,313,626 \\
4Jan18-3 & 1345 & 50 & 3,840,476 \\ \bottomrule
\end{tabular}
\end{table}


\subsection{Classifying vehicles types from UAS data}
Vehicles such as pick-ups, mini-vans, and vans were classified as SUVs while diesel fueled lorries and HGVs were classified as large buses and gasoline lorries as medium buses. HGVs included all vehicles with a tractor cab and attached trailer. The different vehicles observed on each flight is shown in Table \ref{tab:fleetcount}.

\begin{sidewaystable}  %%%%%%%%%% table rotated 90 degrees
\begin{table}[H]
\centering
\caption{Fleet composition from each flight.}
\label{tab:fleetcount}
\resizebox{\columnwidth}{!}{%
\begin{tabular}{@{}cccccccccc@{}}
\toprule
\multicolumn{4}{c}{\textbf{20 Dec 2017 counts}} & \textbf{} & \multicolumn{5}{c}{\textbf{4 Jan 2018 counts}} \\ 
Flight & 20Dec17-1 & 20Dec17-2 & Total &  & Flight & 4Jan18-1 & 4Jan18-2 & 4Jan18-3 & Total \\ \midrule
Car & 45 & 75 & 120 &  & Car & 60 & 56 & 49 & 165 \\
SUV & 55 & 93 & 148 &  & SUV & 40 & 61 & 59 & 160 \\
Bus, Med & 2 & 1 & 3 &  & Bus, med & 0 & 0 & 2 & 2 \\
Bus, Large & 9 & 5 & 14 &  & Bus, Large & 5 & 5 & 6 & 16 \\
 & 111 & 174 & 285 &  & Subtotal & 105 & 122 & 116 & 343 \\ \midrule
\multicolumn{4}{c}{\textbf{20 Dec 2017 percentages}} & \textbf{} & \multicolumn{5}{c}{\textbf{4 Jan 2018 percentages}} \\
Flight & 20Dec17-1 & 20Dec17-2 & Total &  & Flight & 4Jan18-1 & 4Jan18-2 & 4Jan18-3 & Total \\ \midrule
Car & 40.5\% & 43.1\% & 42.1\% &  & Car & 57.1\% & 45.9\% & 42.2\% & 48.1\% \\
SUV & 49.5\% & 53.4\% & 51.9\% &  & SUV & 38.1\% & 50.0\% & 50.9\% & 46.6\% \\
Bus, Med & 1.8\% & 0.6\% & 1.1\% &  & Bus, med & 0.0\% & 0.0\% & 1.7\% & 0.6\% \\
Bus, Large & 8.1\% & 2.9\% & 4.9\% &  & Bus, Large & 4.8\% & 4.1\% & 5.2\% & 4.7\% \\ \bottomrule
\end{tabular}
} %end resize
\end{table}
\end{sidewaystable}  %%%%%%%%%% table rotated 90 degrees

To test the null hypothesis, $H_{o}$ established in Table \ref{tab:vehhyp} for fleet compositions whereby the composite probabilities of each vehicle class from each flight group was $p_{1} - p_{2} = 0$, the Z statistic was calculated according to Equation \ref{eq:2zteststat} and compared to the two-tailed critical value for 95\% confidence of 1.96. If the Z statistic was less than the critical value, the null hypothesis could not be rejected and the probabilities of each vehicle class could be considered equivalent for those flight groups. Table \ref{tab:fleettest} showed that all classes failed to reject the null hypothesis, strengthening the assumption that fleet composition is similar at most intersections and at all times of congestion.

\begin{table}[H]
\centering
\caption{Statistical significance test results for fleet composition between sites.}
\label{tab:fleettest}
\begin{tabular}{@{}ccccc@{}}
\toprule
\textbf{} & \textbf{Car} & \textbf{SUV} & \textbf{Bus, Med} & \textbf{Bus, Large} \\ \midrule
Z-statistic & 1.004 & 0.927 & 0.058 & 0.032 \\
Z-statistic \textless 1.96? & TRUE & TRUE & TRUE & TRUE \\ \bottomrule
\end{tabular}
\end{table}

The results of the pooled data for each vehicle type in the fleet composition is shown in Table \ref{tab:pooledfleet}.

\begin{table}[H]
\centering
\caption{Fleet composition based on pooled observation data.}
\label{tab:pooledfleet}
\begin{tabular}{cccc}
\toprule
\textbf{Car} & \textbf{SUV} & \textbf{Bus, Med} & \textbf{Bus, Large} \\ \midrule
45.4\% & 49.0\% & 0.8\% & 4.8\%\\ \bottomrule
\end{tabular}
\end{table}

\subsection{Evaluating traffic density from UAS data}

Measurements of $\delta_{0}$ were taken from models processed using Pix4D DesktopPro software with  imagery collected from the flights in Table \ref{tb:flightdata}.  Results from each lane were assumed to be part of the same population and pooled as one data set. The first test of similarity to confirm the null hypothesis established in Table \ref{tab:vehhyp} for $\delta_{0}$'s of equal means and variances between samples, required flights from the same day show equal means and variances. Accepting this null hypothesis also implied that the underlying distribution of the data was Normal.  Descriptive statistics from each flight were calculated and shown in Table \ref{tab:normaltest} along with results from one-way ANOVA tests for each flight groups (20Dec2017 and 4Jan2018). Skewness values for normally distributed data is close to zero, while kurtosis values should be 3 \citep{NIST2013}. The one-way ANOVA results for the 20Dec2017 flights shows a p-value greater than the significance level of p=0.05, allowing the Null hypothesis to not be rejected and assuming the means of both flights are not statistically different. However, the p-Value for the 4Jan2018 flights is less than the significance level, requiring the Null to be rejected and assuming that the means for each flight vary significantly.

\begin{sidewaystable}  %%%%%%%%%% table rotated 90 degrees
\begin{table}[H]
\centering
\caption{Statistic and One-way ANOVA tests for flights.}
\label{tab:normaltest}
\resizebox{\columnwidth}{!}{%
\begin{tabular}{@{}cccccc@{}}
\toprule
\textbf{Statistic} & \textbf{20Dec17-1} & \textbf{20Dec17-2} & \textbf{4Jan18-1} & \textbf{4Jan18-2} & \multicolumn{1}{l}{\textbf{4Jan18-3}} \\ \midrule
Mean & 2.227 & 2.0546 & 2.0230 & 2.393 & 2.2201 \\
Std. Dev. & 1.132 & 0.9297 & 0.8507 & 1.272 & 0.9348 \\
Skewness & 1.9369 & 1.6885 & 1.3285 & 2.0740 & 1.1618 \\
Kurtosis & 9.8617 & 6.8944 & 5.0260 & 9.7338 & 4.5501 \\
Count & 108 & 171 & 102 & 119 & 113 \\
 &  &  &  &  & \multicolumn{1}{l}{} \\ \midrule
\textbf{20Dec17 Flights} & \textbf{Sum of squares} & \textbf{Deg of freedom} & \textbf{Mean squares} & \textbf{F-ratio} & \multicolumn{1}{l}{\textbf{p-value}} \\ \midrule
Between Variation & 1.969 & 1 & 1.969 & 1.920 & 0.167 \\
Within Variation & 284.125 & 277 & 1.026 &  &  \\
Total Variation & 286.095 & 278 &  &  &  \\
\multicolumn{1}{l}{} & \multicolumn{1}{l}{} & \multicolumn{1}{l}{} & \multicolumn{1}{l}{} & \multicolumn{1}{l}{} & \multicolumn{1}{l}{} \\ \midrule
\multicolumn{1}{l}{\textbf{4Jan18 Flights}} & \multicolumn{1}{l}{\textbf{Sum of squares}} & \multicolumn{1}{l}{\textbf{Deg of freedom}} & \multicolumn{1}{l}{\textbf{Mean squares}} & \multicolumn{1}{l}{\textbf{F-ratio}} & \multicolumn{1}{l}{\textbf{p-value}} \\ \midrule
Between Variation & 7.513 & 1 & 7.513 & 6.231 & 0.013 \\
Within Variation & 264.073 & 219 & 1.206 &  &  \\
Total Variation & 271.586 & 220 &  &  &  \\ \bottomrule
\end{tabular}
} %end resize
\end{table}
\end{sidewaystable}  %%%%%%%%%% table rotated 90 degrees

By transforming the $\delta_{s}$ data using the common logarithmic function (base 10), the data shows more Normal features as well as allowing the Null hypothesis to stand for both groups as shown in the skewness and kurtosis statistics and one-way ANOVA results in Table \ref{tab:logtest}. Flight 20Dec17-1 has a high kurtosis value, despite the transformation.

\begin{sidewaystable}  %%%%%%%%%% table rotated 90 degrees
\begin{table}[H]
\centering
\caption{Statistic and one-way ANOVA tests for transformed flights.}
\label{tab:logtest}
\resizebox{\columnwidth}{!}{%
\begin{tabular}{@{}cccccc@{}}
\toprule
\textbf{Statistic} & \textbf{20Dec17-1} & \textbf{20Dec17-2} & \textbf{4Jan18-1} & \textbf{4Jan18-2} & \textbf{4Jan18-3} \\ \midrule
Mean & 0.297 & 0.275 & 0.272 & 0.331 & 0.311 \\
Variance & 0.048 & 0.032 & 0.028 & 0.040 & 0.031 \\
Std. Dev. & 0.220 & 0.178 & 0.168 & 0.200 & 0.177 \\
Skewness & -0.737 & 0.229 & 0.323 & 0.415 & -0.003 \\
Kurtosis & 6.044 & 3.364 & 2.796 & 3.157 & 2.983 \\
Count & 108 & 171 & 102 & 119 & 113 \\
 &  &  &  &  &  \\ \midrule
\textbf{20Dec17 Flights} & \textbf{Sum of squares} & \textbf{Deg of freedom} & \textbf{Mean squares} & \textbf{F-ratio} & \textbf{p-value} \\ \midrule
Between Variation & 0.031 & 1 & 0.031 & 0.800 & 0.372 \\
Within Variation & 10.563 & 277 & 0.038 &  &  \\
Total Variation & 10.594 & 278 &  &  &  \\
 &  &  &  &  &  \\ \midrule
\textbf{4Jan18 Flights} & \textbf{Sum of squares} & \textbf{Deg of freedom} & \textbf{Mean squares} & \textbf{F-ratio} & \textbf{p-value} \\ \midrule
Between Variation & 0.189 & 2 & 0.095 & 2.830 & 0.060 \\
Within Variation & 11.057 & 331 & 0.033 &  &  \\
Total Variation & 11.246 & 333 &  &  &  \\ \bottomrule
\end{tabular}
} % end resize
\end{table}
\end{sidewaystable}  %%%%%%%%%% table rotated 90 degrees

In both groups, the p-values are greater than the significance value of p = 0.05 and thus the means within each group can be assumed to be not significantly different. The individual flights within each group can be pooled if transformed in order to compare group means. The results of a one-way ANOVA test between combined flight groups in Table \ref{tab:grouptest} shows that the p-value is greater than the significance value, allowing the Null to stand and providing stronger evidence that the means for different $\delta_{0}$ locations and time are equal if transformed.
%
\begin{sidewaystable}  %%%%%%%%%% table rotated 90 degrees
\begin{table}[H]
\centering
\caption{Statistics and one-way ANOVA test for transformed groups.}
\label{tab:grouptest}
\resizebox{\columnwidth}{!}{%
\begin{tabular}{@{}cccccc@{}}
\toprule
\textbf{Statistic} & \textbf{20-Dec-17} & \textbf{4-Jan-18} &  &  &  \\ \midrule
Mean & 0.284 & 0.306 &  &  &  \\
Variance & 0.038 & 0.034 &  &  &  \\
Std. Dev. & 0.195 & 0.184 &  &  &  \\
Skewness & -0.261 & 0.309 &  &  &  \\
Kurtosis & 4.957 & 3.100 &  &  &  \\
Count & 279 & 334 &  &  &  \\
 &  &  &  &  &  \\ \midrule
\textbf{one-way ANOVA table} & \textbf{Sum of squares} & \textbf{Deg of freedom} & \textbf{Mean squares} & \textbf{F-ratio} & \textbf{p-value} \\ \midrule
Between Variation & 0.078 & 1 & 0.078 & 2.180 & 0.140 \\
Within Variation & 21.840 & 611 & 0.036 &  &  \\
Total Variation & 21.917 & 612 &  &  &  \\ \bottomrule
\end{tabular}
} %end resize
\end{table}
\end{sidewaystable}  %%%%%%%%%% table rotated 90 degrees

The $\delta_{0}$ variable in an MCA model can be represented by taking the exponent of a Normal distribution with the parameters of the mean, $\mu$, and standard deviation, $\sigma$, given in Table \ref{tab:gapnormal}.

\begin{table}[H]
\centering
\caption{Parameters for $\delta_{0}$ transformed distribution}
\label{tab:gapnormal}
\begin{tabular}{@{}ccc@{}}
\toprule
 \textbf{$\mu$} & \textbf{$\sigma$} \\ \midrule
 0.2958 & 0.1892 \\ \bottomrule
\end{tabular}
\end{table}

A distribution of the transformed pooled groups is shown in Figure \ref{fig:ivgloggroup}a along with a Normal distribution, $N(\mu,\sigma)$, fitted from the transformed data. In Figure \ref{fig:ivgloggroup}b, the distribution from Figure \ref{fig:ivgloggroup}a is inverted back to real space using the equation

\begin{equation}
\label{inverse-10}
X = 10^{N(\mu,\sigma)}
\end{equation}

along with an MCA using 5,000 iterations. The resulting fitted distribution is lognormal with a mean of 2.2 m and variance of 1.0 m. 
 
\begin{figure}[H]
\includegraphics[width=\linewidth,keepaspectratio]{images/ivg-group-log.png} 
\caption[Distributions of combined groups of $\delta_{s}$ data.]{Distributions of combined groups of $\delta_{s}$ data showing a. transformed data with fitted Normal distribution and b. MCA re-transformed data using the parameters from Table \ref{tab:gapnormal}.}
\label{fig:ivgloggroup}
\end{figure}

The fitted lognormal distribution provides a good fit as shown in the Q-Q plot in Figure \ref{fig:qlognormal}, especially with  $\delta_{s}$ values less than 5 m (representing approximately 97.5\% of the possible results). Additionally, the MAE between the MCA generated data and calculated  results is 0.0126 m, compared to the 0.041 m MAE measurement error. The composite error is $\pm 0.054$m.

\begin{figure}[H]
%\includegraphics[width=\linewidth,keepaspectratio]{images/qlognormal.png} 
\includegraphics[width=\textwidth,height=\textheight,keepaspectratio]{images/qlognormal.png} 
\caption{Q-Q plot of Figure \ref{fig:ivgloggroup}b.}
\label{fig:qlognormal}
\end{figure}

\subsection{Evaluating vehicle spacing by type}
Applying Bartlett's test to the different groups to evaluate homogeneity of variances produced $p = 0.343$ ($p_{critical} = 0.05$). This  suggests that the groups have similar variances and thus may come from a common population despite have differing means \citep{NIST2013}.

To test the assumption that $\delta_{s}$ did not vary with vehicle types, results from vehicle types were compared to the pooled data set using a two-tailed t-test with unequal variance. Table \ref{tab:vehtest} shows the breakdown of different types as well as frequency of observations on each lane, mean of  $\delta_{s}$ for each lane, transformed (log) mean of  $\delta_{s}$ for each lane, and the resulting t-test result p-value and test result for all observations in the vehicle type class in each lane. Larger vehicles, represented by medium and large buses did not have enough frequency in some lanes and were therefore excluded from the test. 

\begin{table}[H]
\centering
\caption{Results of significance testing of individual vehicle types and pooled data.}
\label{tab:vehtest}
\resizebox{\columnwidth}{!}{%
\begin{tabular}{@{}ccccccc@{}}
\toprule
\textbf{Type} & \textbf{Lane} & \textbf{Count} & \textbf{Mean} & \textbf{LogMean} & \textbf{p-value} & \textbf{Result} \\ \midrule
Car & 1 & 87 & 1.87 & 0.53 & 0.01 & FALSE \\
 & 2 & 102 & 2.10 & 0.65 & 0.53 & TRUE \\
 & 3 & 92 & 2.08 & 0.63 & 0.31 & TRUE \\ \midrule
SUV & 1 & 85 & 2.39 & 0.77 & 0.06 & TRUE \\
 & 2 & 102 & 2.09 & 0.67 & 0.84 & TRUE \\
 & 3 & 111 & 2.44 & 0.79 & 0.02 & FALSE \\ \midrule
Bus, Med & 1 & 17 & 2.50 & 0.86 & 0.05 & TRUE \\
 & 2 & 6 & 2.37 & 0.78 & 0.60 & TRUE \\ \midrule
Bus, Large & 1 & 8 & 1.58 & 0.39 & 0.08 & TRUE \\ \bottomrule
\end{tabular}
}% end resize
\end{table}

The largest 2 classes, cars and SUVs showed no statistically significant variation to the pooled data at the 95\% level ($p>0.05$), except for the 1st lane for cars and the 3rd lane for SUVs. Cars in the first lane stacked closer than average, while SUVs in the 3rd lane stacked farther. Large buses had the shortest average $\delta_{s}$, possibly due to the driver position located closer to the bumper and higher vantage point. Future research should consider driver position as a factor for $\delta_{s}$ distances.


\subsection{Monte Carlo Analysis using field data}
Using the fleet composition and $\delta_{s}$ model parameters developed from the pooled data collected from the UAS flights, a vehicle density model was prepared for MCA. Palisade Software’s @RISK Version 7.5 Industrial Edition (www.palisade.com) was used to provide the Monte Carlo analysis (MCA) using a Latin Hypercube sample generator.  A total of 5,000 iterations were run at 5, 10, 15, and 20 km/h. It was assumed that fleet composition at different speeds will not vary within an observed road segment. The $\delta_{0}$ calculated based on the parameters of Table \ref{tab:gapnormal} can be assumed for speeds $\leq$ 5 km/h. For speeds $>$ 5 km/h, the lognormal distribution is kept and parameters linearly extrapolated by multiplying the mean by factor of the speed relative to 5 km/h.  

The model captures the total number and class of vehicles in one kilometer of road.  Lane changing was not considered. The average (mean) of the total vehicles in one road lane over a 1 km segment at different speeds are summarized in Table \ref{tab:meanvehdensity}.

\begin{table}[H]
\centering
\caption{Average estimated total vehicles in a 1 km single lane.}
\label{tab:meanvehdensity}
\begin{tabular}{@{}lccccc@{}}
\toprule
\textbf{Vehicle Type} & \textbf{Idle} & \textbf{5 km/h} & \textbf{10 km/h} & \textbf{15 km/h} & \textbf{20 km/h} \\ \midrule
Sedans & 76 & 62 & 52 & 40 & 18 \\
SUV's & 82 & 67 & 56 & 43 & 20 \\
Bus, Medium & 1 & 1 & 1 & 1 & 0 \\
Bus, Large & 8 & 7 & 5 & 4 & 2 \\
Total & 167 & 137 & 114 & 88 & 40 \\ \bottomrule
\end{tabular}
\end{table}

Figure \ref{fig6:estimatedobs} shows the results of all vehicles traveling at an average of 5 km/h.
 
%
\begin{figure}[H]
%\includegraphics[width=\textwidth,keepaspectratio]{images/vdense6.png}
\includegraphics[width=\textwidth,height=\textheight,keepaspectratio]{images/vdense9.png} 
\caption{Total estimated idling vehicles on a 1,000 m segment.}
\label{fig6:estimatedobs}
\end{figure}
%

Graphing the statistical mean for the total number of vehicles over different average speed yields a linear form as shown in Figure \ref{fig9:estimateavemix}.  Fitting the curve with a trend line provides very high correlation ($R^{2} = .995$) that can approximate the expected value at each speed.  Similar curves (mean of each speed) for each vehicle class were prepared and summarized in Table \ref{tb4:expectedvehicles}.  A curve for medium buses was not included because the expected value at each speed is 1 per km.

%
\begin{figure}[H]
%\includegraphics[width=\textwidth,keepaspectratio]{images/vdense9.png} 
\includegraphics[width=\textwidth,height=\textheight,keepaspectratio]{images/vdense6.png}
\caption[Total estimated vehicles on 1,000 m segment at average speed]{Total estimated vehicles on 1,000 m segment at average speed (km/h).}
\label{fig9:estimateavemix}
\end{figure}
% 

Table \ref{tb4:expectedvehicles} summarizes the expected number for vehicles by class in a 1,000 m segment based on speed, $s$, in km/h.
%
\begin{table}[H]
\centering
\caption[Expected numbers of vehicles in 1,000 m based on average speed]{Expected numbers of vehicles in 1,000 m based on average speed (km/h).}
\label{tb4:expectedvehicles}
\begin{tabular}{@{}cc@{}}
\toprule
\textbf{Vehicle class} & \textbf{Expected number of vehicles } \\ \midrule
Total & \# of Vehicles = $Integer(-6.2(s) + 165.6)$ \\
Sedans & \# of Sedans = $Integer (-2.82(s) + 75.2)$ \\
SUV's & \# of SUVs = $Integer (-3.04(s) + 81)$ \\
Bus, medium & \ 1 \\
Bus, large & \# of large buses = $Integer (-0.3(s) + 8.2)$ \\ \bottomrule
\end{tabular}
\end{table}
%
The results in Table \ref{tb4:expectedvehicles} are representative only of that section of road with the fleet composition in Table \ref{tab:pooledfleet}.  Different traffic profiles will have different densities and results.  Other factors that could affect the traffic profiles include location of road section, type of road, season, time of day, weather conditions and construction activities. 

%%%%%%%%%%%%%%%%%%%%%%%%%%%%%%%%%%%%%%%%%%%%%%%%%%%%%%%%%%%%%%%%%%%%%%%%%%%%%%%%%%%%%%%%%%%%%%%%%%%%%%
%%%%%%%%%%%%%%%%% END OF SECTION%%%%%%%%%%%%%%%%%%%%%%%%%%%%%%%%%%%%%%%%%%%%%%%%%%%%%%%%%%%%%%%%%%%%%%
%%%%%%%%%%%%%%%%%%%%%%%%%%%%%%%%%%%%%%%%%%%%%%%%%%%%%%%%%%%%%%%%%%%%%%%%%%%%%%%%%%%%%%%%%%%%%%%%%%%%%%
\clearpage
\section{Air pollution time series forecasting}

\subsection{Final parameter selection}
The model was trained on 80\% of the processed data and tested against 20\% of the total available data. Because of the Tensor formation for input, the actual number of samples provided for training and testing was based on the look ahead horizon and number of recurrent (look-back) units of the individual run. The farther out the prediction, the fewer samples were available because of the time shifting required. The total amount of samples available for training and testing could be calculated as total samples = $(16,035 - h)$ where $h$ is the prediction horizon (as an integer value $>$ 1). 

The number of training epochs was limited after reviewing training error values up to 20 epochs for look-ahead horizons of 24 hrs, 36 hrs, and 48 hrs as seen in Figure \ref{fig:horizon-loss}. An optimum number of 10 epochs was used for later model runs as it minimizes the training error without overfitting which begins to take place after 12 epochs, especially in Figure \ref{fig:horizon-loss}a. 
%
\begin{figure}[H]
\centering
%\includegraphics[width=.5\textwidth]{images/horizon-loss.png}  %assumes jpg extension
\includegraphics[width=\textwidth,height=\textheight,keepaspectratio]{images/horizon-loss.png}
\caption[Loss function errors of training and test data sets]{Loss function errors for training and test data sets for different horizons at (a) 24 hr, (b) 36 hr, and (c) 48 hr.}
\label{fig:horizon-loss}
\end{figure}
%

\subsection{Performance measures}

Final parameter selection and performance were measured by Mean Absolute Error (MAE) given by 
%
\begin{equation}
\label{eq:MAE}
MAE = \frac{1}{n}\sum^{n}_{i=1} \left | y_{obs_{i}}- y_{pred_{i}} \right |
\end{equation}
%
and Root Mean Square Error given by
%
\begin{equation}
\label{eq:RMSE}
RMSE = \sqrt{\frac{1}{n}\sum^{n}_{i=1} \left ( y_{obs_{i}}- y_{pred_{i}} \right )^{2}}
\end{equation}
%
MAE and RMSE are widely used measures of continuous variables with RMSE criticized for over-biasing towards large errors \citep{Chai2014, Willmott2005}. Both metrics were calculated for comparison; however MAE is used more often for descriptive analytics.

\subsection{Impact of features and parameters on results}

The network trained very well with all 25 input features from Table \ref{tb:parameters}. An example of the predicted results compared with the observed measurements (8 hr ave O$_{3}$) over a 24 hr horizon is shown in Figure \ref{fig:example24}.
%
\begin{figure}[H]
\centering
\includegraphics[width=\textwidth,height=\textheight,keepaspectratio]{images/example24.png}
\caption{Results of training an RNN with a 24 hr horizon.}
\label{fig:example24}
\end{figure}
%
The MAE for this scenario is 0.41 ppb during training and 0.37 ppb during testing. The residuals of this scenario are shown in Figure \ref{fig:normalresiduals} where they show a Normal distribution tendency (skewness = 0.411, kurtosis = 3.94, where Normal is 0 and 3, respectively) with a positive bias given by the distribution mean of 1.632 ppb. This is consistent with \ref{fig:example24} that shows the model slightly under-predicting.

%
\begin{figure}[H]
\centering
\includegraphics[width=\textwidth,height=\textheight,keepaspectratio]{images/normalresiduals.png}
\caption{Distribution of residual test errors for a 24 hr horizon network.}
\label{fig:normalresiduals}
\end{figure}
%

The results of the decision tree analysis in Figure \ref{fig:importance} showed that many features could be removed without impacting network performance. Features were removed based on the order of least importance in groups of 5 until the most prominent feature remained. The results in Figure \ref{fig:features} show that overall training error improves with fewer inputs. By removing input features, the system complexity is also reduced, allowing the network to train easier. While providing better training results, reducing feature inputs also makes it easier to overfit on the training data.
%
\begin{figure}[H]
\centering
%\includegraphics[width=.75\textwidth]{images/features.png}  %assumes jpg extension
\includegraphics[width=\textwidth,height=\textheight,keepaspectratio]{images/features.png}
\caption{Training Error associated with feature reduction on network prediction.}
\label{fig:features}
\end{figure}
%
Based on the training error curves in Figure \ref{fig:features}, the 5 feature data set was used for evaluation because it provided stable errors over the prediction horizons of interest. The features used were (in order of importance) 8 hr ave O$_{3}$ in $ppb$, 1 hr O$_{3}$ in $ppb$, SR, the cosine of WD, and $CH_{4}$.

Parameter sensitivity analysis was performed on a model with default values shown in Table \ref{tb:default-parameter}.
%

\begin{table}[H]
\centering
\caption{Default values for parameter sensitivity analysis.}
\label{tb:default-parameter}
\begin{tabular}{@{}lc@{}}
\toprule
\textbf{Parameter} & \textbf{Default Value} \\ \midrule
Input features & 26 \\
Prediction horizon (hours) & 24 \\
Look back nodes & 26 \\
Samples/batch & 72 \\
Dropout factor & 0.2 \\ \bottomrule
\end{tabular}
\end{table}
%
All other parameters were held constant as an individual parameter was varied. The prediction horizon value of 24 hrs was held constant throughout all runs shown in  Figure \ref{fig:parameters}. In all cases, the error measurements, MAE and RMSE showed similar forms, despite the RMSE having a consistently higher value, as expected. The prediction horizon of the model using the 5 feature data set is shown in Figure \ref{fig:predictionhrs}.

%
\begin{figure}[H]
\centering
%\includegraphics[width=.75\textwidth]{images/predictionhrs.png}  %assumes jpg extension
\includegraphics[width=\textwidth,height=\textheight,keepaspectratio]{images/predictionhrs.png}
\caption{Prediction horizons using 5 features and default parameters.}
\label{fig:predictionhrs}
\end{figure}
%
As the prediction extends further into the future ($>$ 80 hrs), the training error climbs rapidly, while the test error appears to level off. The model began to overfit by this point and predictions past that range were considered to be unreliable.

The results in  Figure \ref{fig:parameters}a show training error for different prediction horizons over several parameters. The parameter that influences the model performance the most is the number of look-back nodes in relation to the prediction horizon. A horizon value of + 2 provides the lowest errors, while adding additional nodes increases the model complexity and makes training more challenging. 

Samples/batches in Figure \ref{fig:parameters}b show relatively little error variance until many samples are included ($>$75 samples/batch). While more samples per batch are preferred to reduce training time, too many create bias in the loss function as the overall average of each sample reduces chances for updates.

The number of recurrent, or look back nodes, in relation to the prediction horizon was considered in  Figure \ref{fig:parameters}b. Both training and test results are minimized at 26, the horizon value + 2. This was consistent with other horizon prediction values such as 36 and 48. As more look back nodes are added, the error increases.

Finally, the use of dropout is recommended to improve generalization of the model and reduce overfitting \citep{Gal2016}. For this model, dropout was applied only between the output of the LSTM layer and the FF output layer. The error shows reasonably good optimization at around 0.2. The errors level out at higher rates at around 0.35. The default values were the optimum parameters based on the results shown in Figure \ref{fig:parameters}.
%
\begin{figure}[H]
\centering
%\includegraphics[width=.75\textwidth]{images/parameters.png}  %assumes jpg extension
\includegraphics[width=\textwidth,height=\textheight,keepaspectratio]{images/parameters.png}
\caption[Impact of model parameters on training error]{Impact of (a) batch samples, (b) Look back nodes, and (c) dropout factor parameters on training errors in the model.}
\label{fig:parameters}
\end{figure}
%

An LSTM model has many more variables that can be optimized compared to other models. The RNN used in this study with 25 input features had 16,485 update-able parameters, of which 16,432 were in the LSTM layer alone. As a comparison, an FFNN with 3 hidden layers (5 layers total), bias on all layers, and the same 25 feature inputs, had only 2,107 parameters. Nonlinearities are further introduced during the training phase in which the derivative of the activation function for each layer is used to influence the weight updates. It is therefore difficult to fully explain the mechanisms driving the output results of complex DL models. In order to insure the model is working, the output must be compared with known results and parameters adjusted to optimize performance.

\subsection{Comparison to previous studies}
Previous studies mentioned in Section 1 used RMSE and other error measurement methods than MAE. The 3 studies that used RMSE were compared to our results with an LSTM network. Luna et al. (2014) used SVMs and FFNNs \citep{Luna2014}. Feng et al. (2011) used a a multi-layered system that included an SVM and a genetic algorithm stabilized FFNN \citep{Feng2011}. Wang and Lu used an FFNN with a particle swarm optimization \citep{Wang2006}. All studies, except Gomez (2003), used PCA to pre-process the data. Comparing the results of the RNN to these previous studies gives an initial impression  that the RNN has an order of magnitude improvement over the best FFNN or SVM models as shown in Table \ref{tb:compare}.
%

\begin{table}[H]
\centering
\caption{Comparison of LSTM RNN test data results to previously published results.}
\label{tb:compare}
\resizebox{\columnwidth}{!}{%
\begin{tabular}{@{}lcccc@{}}
\toprule
\textbf{Source} &  \textbf{Prediction horizon} & \textbf{Results (RMSE)} & \textbf{LSTM (RMSE)} \\ \midrule
Luna et al. (2014) & 1 hr & 6.3 - 12.3 & 0.8 \\
Feng et al. (2011) &  12 hr & 5.5 - 86.9 & 1.5 \\
Wang and Lu (2006) & 24 hrs & 7.9 - 11.2 & 2.5 \\ 
Gomez et al. (2003)& 24 hrs & 6.9 - 9.9 & \\ \bottomrule
\end{tabular}
} %end resize
\end{table}
%
The results cannot however be directly compared because they were based on different data sets. While the other studies used complex and hybrid architectures along with complicated pre-processing, the RNN model pre-processing was very simple after the features were prioritized using a decision tree. The RNN and LSTM are themselves complex algorithms with many internal parameters that undergo training and updates.

\subsection{Comparison to different models}

In order to evaluate how the RNN model compared to other forecasting models using the same data set, a comparison was made for a 24 hr prediction using a FFNN with three hidden layers, and an ARIMA model. The FFNN was built with the Keras library using $relu$ activation functions in the hidden layers and a $sigmoid$ function for the output. The inputs included all the parameters from Table 3 and the model was allowed to train for 1,000 epochs. The number of nodes in the hidden layer was based on the estimated \# of hidden nodes = ($SF$ * \# of input nodes) + \# of output nodes where $SF$ is a scaling factor between 0.5 and 1 \citep{Papaleonidas2013}. For an $SF$ of 0.75, the number of nodes is 20 nodes in each hidden layer. An MSE loss function and Nadam optimizer were also used to build the model. The ARIMA model was built using the \textbf{auto.arima} function in R (ver 3.4.3) \citep{Hyndman2013} and fitted on the 8 hr ave O$_{3}$ only. The formatting parameters, $p$, $d$, and $q$ were calculated as 3, 0, and 0, respectively. The results are shown in Figure \ref{fig:diffmodels}.

%
\begin{figure}[H]
\centering
\includegraphics[width=\textwidth,height=\textheight,keepaspectratio]{images/modelcompare.png}
\caption{Comparison of different model forecasts over a 24 hr period}
\label{fig:diffmodels}
\end{figure}
%

The continuous nature of the RNN and ARIMA predicted curves are in contrast to the FFNN predicted results due to the lack of memory inherent within the FFNN algorithm. While not able to predict as the time series as well as the RNN, the ARIMA model allows the previous result to influence the next time step. The FFNN model requires all necessary information to be provided within the input sample and network weights, allowing for better prediction results than the ARIMA, but a non-continuous form. 

%---------------------------------------------------------------
%------------------------End of Chapter----------------------
\bigskip
\begin{center}
END OF CHAPTER
\end{center}
